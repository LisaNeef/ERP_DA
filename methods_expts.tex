%-------------------experiments-------------------------
DART \citep{Anderson2009} is an open-source, community tool that provides ensemble-based data assimilation that is independent of a particular model and observation set.
Interfaces exist between DART and both CAM5 \citep{Raeder2012} and WACCM \citep{Pedatella2013}.

%-----CAM5 experiments
Experiments 1-4 (Table \ref{tab:expts} are performed using CAM5.  
A reference CAM5 simulation, run from January 2008 through February 2009 with observed sea surface temperatures as a boundary condition, constitutes the ``truth'' from which we generate a set of synthetic observations for assimilation. 
We generated an 80-member CAM5 ensemble by selecting the 1 January 2009 restart files from an 80-year CAM simulation, and assimilating 6-hourly synthetic temperature and wind ``radiosonde'' observations on a grid, amounting to roughly 27000 observations per day. 
This yields a tightly-constrained starting ensemble for experiments 1-4, which all start on 1 January 2009.

As a control experiment, the 80-member CAM5 ensemble is integrated forward over January and February 2009 without data assimilation. 
In the second experiment, the ensemble is run forward as before, but assimilating the synthetic angular momentum observations every 12 hours.
In Experiment 3, we continue to assimilate the grid of radiosonde observations, but only observing temperature.
In Experiment 4, we assimilate both the 6-hourly radiosonde temperatures and the three 12-hourly global angular momentum components.
We only integrated these latter two experiments for 31 and 17 days, respectively, because this amount of integration time was found to be sufficient to \hl{compare the error reduction in the two experiments.} 


%-----WACCM experiments
In Section \ref{sec:evaluation_variable} we examine how angular momentum observations can be used to evaluate the constrait imposed by other observations in a data assimilation system.  
Here we compare four DART-WACCM ensemble simulations (E5-E8), where the amount of (real) observations assimilated is increased for each experiment. 
In E5, a 40-member DART-WACCM ensemble is run forward from through October 2009 with no assimilation.  
The same initial ensemble is used for E6-E7, but now assimilating the meteorological observations that are also used in the NCEP/NCAR reanalysis (i.e. winds and temperatures from radiosondes and aircraft \citep{Saha2010}), as well as refractivity profiles simulating observations by the COSMIC GPS Radio-Occultation instrument \citep{Anthes2008}.
In E6, the assimilation is limited to the 30S-30N tropical band, while in E7 observations are assimilated globally. 
E8 is similar to E7, but additionally assimilates temperature profiles from the Sounding of the Atmosphere Using Broadband Emission Radiometry experiment \citep[SABER]{Russell2009}, from 20-100k.
E8 was performed by \hl{CITE NICK'S PAPER} for a different study and thus covers a different period, 6-25 November 2008.  
E6 is not entirely comparable to E5-7 because it uses a different starting ensemble and covers a different time period, but, as shall be seen below, the qualitative comparison between these experiments is sufficient to demonstrate the usefulness of Earth rotation parameters as an evaluation variable. 

