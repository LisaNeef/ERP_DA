%-------------------experiments-------------------------
DART \citep{Anderson2009} is an open-source, community tool that provides ensemble-based data assimilation that is independent of a particular model and observation set.
Interfaces have been developed between DART and both CAM \citep{Raeder2012} and WACCM \citep{Pedatella2013}.

%-----WACCM experiments
\subsubsection{DART-WACCM Experiments}
Section \ref{sec:evaluation_variable} uses experiments \WACCMNODA-\NCARFULL~to examine how Earth rotation parameters reflect the constraint imposed by other observations in a data assimilation system.  
Experiment \WACCMNODA~runs a 40-member ensemble of WACCM simulations forward for two months starting on 1 Oct 2009, with no assimilation. 
Experiments \WACCMTROPICS-\NCARFULL~steadily increase the number of observations assimilated. 
\WACCMTROPICS~and \WACCMGLOBAL~use the same initial ensemble as \WACCMNODA, and assimilate the meteorological observations that are also used in the NCEP/NCAR reanalysis, i.e. winds and temperatures from radiosondes and aircraft \citep{Saha2010}, as well as refractivity profiles measured via GPS radio-occultation by the COSMIC satellite \citep{Anthes2008}.
The observations extend from the surface to about 2 hPa. 
\WACCMTROPICS~assimilates the observations only in the 30S-30N tropical band, while in \WACCMGLOBAL~assimilates over the entire globe.

\NCARFULL ~is similar to \WACCMGLOBAL, but additionally assimilates temperature profiles from the Sounding of the Atmosphere Using Broadband Emission Radiometry experiment \citep[SABER]{Russell2009}, from 20-100km.
\NCARFULL ~was performed by \citet{Pedatella2014}, which covers a different period than \WACCMNODA-\WACCMTROPICS ~(1 Jan- 27 Feb 2009), and uses a different starting ensemble.  
As shall be seen below, the qualitative comparison between these experiments is sufficient to demonstrate the usefulness of Earth rotation parameters as an evaluation variable. 

Experiments \WACCMNODA-\NCARFULL~all use a Gaspari-Cohn function \citep{Gaspari1999} to localize observations, with a half-width of 0.2 radians in the horizontal, and 0.5 scale heights in the vertical.  
Experiments \WACCMNODA-\WACCMGLOBAL~carry the additional requirement that no analysis increment is allowed above 0.1 hPa, in order to prevent the generation of spurious gravity waves resulting from a large ensemble spread in the absense of upper level observations. 
All DART-WACCM experiments shown here employ adaptive inflation to inflate the prior ensemble \citep{Anderson2009}.  

%%---localization and inflation in Nick's 2014 paper: 
%Similar to Pedatella et al. [2013], the horizontal and vertical localizations are specified with a Gaspari-Cohn function [Gaspari and Cohn, 1999]
%with half widths of 0.2 radians and 0.5 in ln(p∕p0) coordinates, respectively. Spatially and temporally varying adaptive inflation is used to inflate the ensemble prior to the assimilation in order to prevent collapse of the model spread [Anderson, 2009].

%%---localization and inflation in Nick's 2013 paper: 
%The most notable change is that we specify the ver- tical localization using a Gaspari-Cohn function [Gaspari and Cohn, 1999] with a half width of 0.5 in ln(p/p0) coor- dinates, where p is pressure and p0 is the surface pressure. This differs from the vertical localization of 1000 hPa in p used by Raeder et al. [2012]. 

%-----CAM5 experiments
\subsubsection{DART-CAM Experiments}
Experiments \NODA-\ERPRST ~(Table \ref{tab:expts}) are performed using DART-CAM.  
A reference CAM simulation, run from Jan 2008 through Feb 2009 with observed sea surface temperatures as a boundary condition, constitutes the ``truth'' from which we generate a set of synthetic observations for assimilation. 
From this simulation, we generate synthetic observations of $\chi_1$, $\chi_2$, and $\chi_3$ every 24 hours, reflecting the observation frequency of the real Earth rotation data series published by the International Earth Rotation Services \citep{iers}.  
To roughly simulate reanalysis-type observations, we also generate 
6-hourly temperature observations on a uniform grid, with 1800 locations on the globe and on 15 levels ranging from the surface to 5 hPa (i.e. 27000 observations total at each time). 

An 80-member ensemble of CAM states was generated by selecting the 1 Jan restart files from an 80-year CAM simulation, and assimilating 6-hourly synthetic temperature and wind observations on a uniform grid (roughly 27000 observations per day) for an entire year, which yields a tightly-constrained starting ensemble on 1 Jan 2009.  
\NODA ~integrates the CAM ensemble forward over Jan and Feb 2009 without data assimilation. 
In \ERPALL, the ensemble is run forward as before, but assimilating the synthetic angular momentum observations every 12 hours.
\RST ~assimilates only the temperature observations.
\ERPRST ~assimilates both the temperature observations and the three global angular momentum components.
We only integrated these latter two experiments for 31 and 17 days, respectively, because this amount of integration time was found to be sufficient to compare the relative error reduction in the two experiments. 
