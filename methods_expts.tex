%-------------------experiments-------------------------
DART \citep{Anderson2009} is an open-source, community tool that provides ensemble-based data assimilation that is independent of a particular model and observation set.
Interfaces exist between DART and both CAM \citep{Raeder2012} and WACCM \citep{Pedatella2013}.

%-----WACCM experiments
Section \ref{sec:evaluation_variable} uses experiments \WACCMNODA-\NCARFULL~to examine how Earth rotation parameters can reflect the constraint imposed by other observations in a data assimilation system.  
Experiment \WACCMNODA~ runs a 40-member ensemble of WACCM simulations forward through October 2009, with no assimilation. 
Experiments \WACCMTROPICS-\NCARFULL ~steadily increase the number of observations assimilated. 
\WACCMTROPICS~ and \WACCMGLOBAL~ use the same initial ensemble as \WACCMNODA, and assimilate the meteorological observations that are also used in the NCEP/NCAR reanalysis, i.e. winds and temperatures from radiosondes and aircraft \citep{Saha2010}, as well as refractivity profiles measured via GPS radio-occultation by the COSMIC satellite \citep{Anthes2008}.
\WACCMTROPICS~assimilates the observations only in the 30S-30N tropical band, while in \WACCMGLOBAL~assimilates over the entire globe.
\hl{what is the vertical extent of these observations, and how do we localize?} 
\hl{also here describe how we generate the initial ensemble.}

\NCARFULL ~is similar to \WACCMGLOBAL, but additionally assimilates temperature profiles from the Sounding of the Atmosphere Using Broadband Emission Radiometry experiment \citep[SABER]{Russell2009}, from 20-100km.
\NCARFULL ~was performed by \citet{Pedatella2014}; this run covers a different period than \WACCMNODA-\WACCMTROPICS ~(1 Jan- 27 Feb 2009), and uses a different starting ensemble.  
As shall be seen below, the qualitative comparison between these experiments is sufficient to demonstrate the usefulness of Earth rotation parameters as an evaluation variable. 

%-----CAM5 experiments
Experiments \NODA-\ERPRST ~(Table \ref{tab:expts}) are performed using DART-CAM.  
A reference CAM simulation, run from Jan 2008 through Feb 2009 with observed sea surface temperatures as a boundary condition, constitutes the ``truth'' from which we generate a set of synthetic observations for assimilation. 
From this simulation, we generate synthetic observations of $\chi_1$, $\chi_2$, and $\chi_3$ every 24 hours, reflecting the observation frequency of the real Earth rotation data series published by the International Earth Rotation Services \citep{iers}.  
We also generate generic temperature observations on a regular grid every 6 hours, to roughly mimic the reanalysis-type observations that would conventionally be assimilated, as in \WACCMGLOBAL ~for example. 
\hl{Mention vertical extent of the obs here.}

An 80-member ensemble of CAM simulations was generated by selecting the 1 Jan restart files from an 80-year CAM simulation, and assimilating 6-hourly synthetic temperature and wind observations on a uniform grid (roughly 27000 observations per day) for an entire year, which yields a tightly-constrained starting ensemble on 1 Jan 2009.  
\NODA ~integrates the CAM ensemble forward over Jan and Feb 2009 without data assimilation. 
In \ERPALL, the ensemble is run forward as before, but assimilating the synthetic angular momentum observations every 12 hours.
\RST ~assimilates only the temperature observations.
\ERPRST ~assimilates both the temperature observations and the three global angular momentum components.
We only integrated these latter two experiments for 31 and 17 days, respectively, because this amount of integration time was found to be sufficient to compare the relative error reduction in the two experiments. 
