%\hl{AAM has been shown to be a useful variable because it sums up the atmosphere in just 3 variables that reflect a whole host of things.}
The global atmospheric angular momentum is a convenient variable to quantify the overall behaviour or atmospheric models, because it represents the integrated atmospheric mass and wind fields in terms of three variables, which are observed at high accuracy in the form of variations in the Earth's rate of rotation and polar orientation. 
%
%\hl{We show that it's also a really useful parameter for comparing the degree of constraint imposed by different DA setups.}
Because these Earth rotation parameters can be compared to the angular momentum of the atmosphere and are also independent of meteorological observations, they can be used to measure 
the magnitude of the constraint imposed by a data assimilation system.
%
%\hl{AAM is an appealing assimilation variable because it might constrain large scale structures -- but we illustrate here why that doesn't work.}
We test whether assimilating the atmospheric angular momentum can increase a model's agreement with reality. 
Using ensemble assimilation experiments with the Community Atmosphere Model 5 (CAM5), we show that  
assimilating the angular momentum can sometimes reduce the bias between the ensemble and the true state, but generally leads to filter divergence, with the ensemble clustering around a mean state that is farther from the truth than implied by the ensemble spread. 
This result suggests that observations that represent spatial or temporal averages are poor assimilation variables that lead to divergence of the assimilation system.  
