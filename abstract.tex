\hl{Revised argumentation}
\begin{enumerate}
	\item \hl{AAM has been shown to be a useful variable because it sums up the atmosphere in just 3 variables that reflect a whole host of things.}
	\item \hl{We show that it's also a really useful parameter for comparing the degree of constraint imposed by different DA setups.}
	\item \hl{AAM is an appealing assimilation variable because it might constrain large scale structures -- but we illustrate here why that doesn't work.}
\end{enumerate}

The exchange of angular momentum between the atmosphere and the solid Earth causes the Earth's rotational rate and polar orientation to vary in time. 
Consequently, these variations contain information about the angular momentum of the atmosphere, which in turn is why it has been suggested to constrain models by assimilating measured Earth rotation anomalies. 
We test whether the assimilation of Earth rotation data can improve a model state, by assimilating the atmosphere's angular momentum (which represents idealized Earth rotation parameters) into an ensemble of simulations of the Community Atmosphere Model.
We find that assimilating the angular momentum can sometimes reduce the bias between the ensemble and the true state, but generally leads to more rapid filter divergence, with the ensemble tightly clustering around a mean state that is farther from the truth than implied by the ensemble spread. 
Nevertheless, we show that Earth rotation parameters are a useful variable for evaluating the degree of constraint given by other variables in a data assimilation system.  
This result may also be applicable to other assimilation variables that represent spatial or temporal averages. 
