The exchange of angular momentum between the atmosphere and the solid Earth causes the Earth's rotational rate and polar orientation to vary in time. 
Measurements of these variations therefore contain information about the angular momentum of the atmosphere, which suggests that it may be possible to constrain atmosphere models by assimilating measured Earth rotation anomalies into a model. 
We test whether the assimilation of Earth rotation data can actually improve a model state, by assimilating idealized Earth rotation parameters (polar motion and length-of-day anomalies) into an ensemble of simulations of the Community Atmosphere Model, using Ensemble Adjustment Kalman Filter.
Adding the Earth rotation observations to the assimilation of conventional (localized) observations leads to more rapid filter divergence because \textcolor{alert}{FILL IN}. 
\textcolor{alert}{[probably need a closing message here.]}
