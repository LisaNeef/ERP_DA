The global atmospheric angular momentum represents the integrated atmospheric mass and wind fields in terms of three variables, which are observed at high accuracy in the form of variations in the Earth's rate of rotation and polar orientation. 
%
%\hl{AAM is an appealing assimilation variable because it might constrain large scale structures -- but we illustrate here why that doesn't work.}
We use the Community Atmosphere Model (CAM) within the Data Assimilation Research Testbed (DART) to test whether assimilating these parameters can increase a model's agreement with reality. 
We show that
assimilating the atmospheric angular momentum suffers from a high sampling error relative to a very small state-to-observation covariances, and therefore
generally leads to filter divergence.
This result suggests that observations that represent spatial or temporal averages are poor assimilation variables that lead to divergence of the assimilation system.  
%
%\hl{We show that it's also a really useful parameter for comparing the degree of constraint imposed by different DA setups.}
However, because the Earth rotation parameters can be compared to the angular momentum of the modeled atmosphere and are also independent of meteorological observations, they can be used to measure 
the magnitude of the constraint imposed by a data assimilation system.
%
