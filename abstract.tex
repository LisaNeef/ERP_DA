On timescales shorter than a few years, variability in the Earth's rotation and orientation comes primarily from the exchange of angular momentum between the atmosphere and the solid Earth. 
Measurements of Earth rotation variations therefore contain information about the angular momentum of the atmosphere. 
This suggests that it may be possible to constrain atmosphere models with measurements of Earth rotation, by assimilating measured anomalies in the rotation rate and the polar orientation into a model. 
We test this idea using idealized assimilation experiments where observations of atmospheric angular momentum are generated from a known state and assimilated into an ensemble of simulations using the Community Atmosphere Model and an Ensemble Adjustment Kalman Filter. 
Except for a few limited cases, the assimilation of atmospheric angular momentum causes the ensemble filter to diverge, becaise the constraint imposed by the observations it too weak to stop the growth of nonlinearities. 
Adding the atmospheric angular momentum observations to the assimilation of conventional (localized) observations leads to more rapid filter divergence. 
These results can be extanded to other cases where observations are assimilated that are an integral or average of the observed state. 
