% much shorter attempt at the introduction

% P1: general knowledge: Earth rotation data contain info about the AM of the Earth system, but it's unclear how to use this info  
Variations in the rotation and orientation of the Earth are measured regularly and at high accuracy using space geodesy \citep{Gross1992,iers}. 
These measurements contain information about the atmosphere, oceans, and continental hydrosphere, which all exchange angular momentum with the solid Earth, thereby causing both the Earth's rate of rotation and its polar orientation to fluctuate. 

% P2: previous work and gaps in it: others have shown that certain model adjustments rectify diffs between observed ERPs and modeled AM, most notably Saynish -- but they did not show that this actually results in a better model
The atmosphere and ocean dominate sub-decadal Earth rotation variations. 
We can therefore potentially use the difference between the observed Earth rotation anomalies, and modeled atmospheric and oceanic angular momenta, to correct models of the atmosphere and oceans. 
Indeed, previous studies have used Earth rotation data to evaluate the accuracy of, 
and identify biases in, atmosphere models \citep{Boer1990, Rosen2000}, ocean 
models \citep{Gross1996a} or reanalysis data \citep{Yu1999, Aoyama2000, 
Paek2012a},
%\citet{Ayoma2000} identified potential problems in reanalysis wind data due to discrepancies between predicted and observed LOD variations.
%\citet{Paek2012} used LOD observations to evaluate the accuracy of and identify biases in various reanalysis sets.
%\citet{Rosen2000} used observed LOD anomalies to evaluate the interannual variability simulated by AMIP models.
%\citet{Gross2006a} used observed polar motion to identify discrepancies in ocean tide models: [FILL IN]
In particular, \citet{Saynisch2010,Saynisch2011} and \citet{Saynisch2012} assimilated the residual between observed Earth rotation parameters and the estimated atmospheric --the residual being an estimate of the oceanic angular momentum-- into two ocean models of varying complexity. 
Thse studies showed assimilation of Earth rotation data can close the total momentum budget by adjusting the modeled oceanic angular momentum, while keeping in agreement with other ocean observations.  
\citet{Saynisch2010} and \citet{Saynisch2012} showed that the budget is mainly closed by changing large-scale model parameters such as the freshwater flux, while \citet{Saynisch2011} found that closing the equatorial angular momentum budget in particular required changes in the model's current field. 

These results suggest that an atmopspheric model might also benefit from the assimilation of Earth rotation data, especially since the atmosphere dominates subdecadal length-of-day variations and accounts for at least half of subdecadal polar motion variations. 
Since there is still considerable inaccuracy in the atmospheric angular momentum estimated from reanalysis data, adding Earth rotation data to the observations that are conventionally assimilated could potentially add valuable information to reanalysis fields, for example. 

However, the assimilation studies cited above also faced a major challenge in that they attempted to constrain large, global model fields with obsrvations of three global parameters, which presents a strongly undercontstrained optimization problem. 
Thus, while \citet{Saynisch2010,Saynisch2011} and \citet{Saynisch2012} were able to show interesting and physically-plausible adjustments to the models resulting from the comparison to Earth rotation variations, it is not known whether the solutions found by the data assimilation were unique and truly closer to the true state of the ocean.  

% P3: how our study closes the gap: We show how a model state is updated using obseved ER information and identif the problems thereing, using (1) idealized experiments where the true error is known, and (2) a comparison to more conventional obs.
This study examines the potential of observed Earth rotation variations to constrain a high-dimensional model, both in the presence and absence of conventional, spatially-localized observations. 
The main differences between this study and \citet{Saynisch2010,Saynisch2011} and \citet{Saynisch2012} are as follows: 
(2) We use an atmosphere rather than an ocean model.
(1) We generate synthetic observations from a model simulation that serves as the "true state". This allows us to measure the true error in the constrained model state. The use of synthetic observations also simuluates an idealized situation where the atmosphere accounts for 100\% of Earth rotationv variations, which means that our experiments do not require an estimate of the oceanic or hydrospheric contrubutions to Earth rotation variations.  


% P4: Organization of the paper
The paper is organized as follows.  
Section \ref{sec:methods} describes the data assimilation experiments, observations assimilated, assimilation method, and models used.
Section \ref{sec:assimilation_variable} examines to what extend observations of the global atmospheric angular momentum, and by extension Earth rotation parameters, can constrain the modeled atmosphere. 
In Section \ref{sec:evaluation_variable}, we examine the value of Earth rotation measurements as an evaluation tool.  
The results are summarized and discussed in Section \ref{sec:discussion}.

