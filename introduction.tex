
% CG zoomed in: earth rotation observations in particular have been found to deliver unique information about the atmosphere [examples...] -- it seems natural to use formal DA to really figure out what these obs have to say about our models, and in fact Saynisch et al have done this.
%-------------------------------------------
Anomalies in the Earth's rotation and orientation reflect changes in the angular momenta of the atmosphere, ocean, and continental hydrosphere, which are transferred to the solid Earth to conserve the total angular momentum of the Earth system. 
Because the atmosphere dominates angular momentum exchange on timescales shorter than a few years (the ocean playing a secondary, but  not negligible role), observations of Earth rotation and orientation anomalies contain unique information about the atmosphere.
For example, atmospheric modes of variability like the the Madden Jullian Oscillation \textcolor{alert}{[CITE]} or sudden stratospheric warmings, \citet{Neef2014}) are visible as modes in the timeseries of length-of-day and polar motion anomalies. 

It may therefore be possible to constrain atmosphere or ocean models with observations of Earth rotation, by requiring that the angular momentum of the modeled atmosphere (and/or ocean) agree with the observerd Earth rotation and orientation anomalies, using a data assimilation algorithm.
This could potentially constrain the large-scale model flow, or point outr important model biases. 
\cite{Saynisch2011,Saynisch2010} and \cite{Saynisch2012} assimilated measurements of the polar wobble and length-of-day changes into various ocean models, and found that large adjustments in the model's mass distribution and currents were needed to reconcile the modeled ocean angular momentum with the observations. 
They did not, however, show how much of this change was due to the assimilation of Earth rotation observations in particular, relative to more conventional (localized) observations, which were also assimilated. 
Those studies also did not show whether assimilation acually improved the modeled ocean state. 

% DC1: it's unclear whether integral observations can be useful constraints in the first place.
Earth rotation and orientation variations represent global angular momentum, which is an integral of the modeled mass and wind or current fields. 
Data assimilation algorithms generally adjust points in a model relative to local observations made nearby. 
When an observation is a function of the entire model state, the data assimilation step may not necessarily bring every point in the state closer to the truth, even if it fits the observed data. 


%% PP1: in this paper we evaluate whether ERPs are suited for data assimilation and show some of the difficulties associated with assimilating an integral quantity
Since assimilation algorithms are designed to fit observations, it is moreover impossible to judge whether an assimilation was successful by whether the analysis agrees wit the assimilated observations. 
It is more useful to use a model to simulate a "true" state, then generate observations from this state, and assimilate them back into a different realization of the same model. 
The resulting analysis (i.e. the wind or current, temperature, and pressure fields of the model) can then be compared to the known "true" state. 

In this study we generate and assimilatie synthetic observations of the atmospheric angular momentum using the Community Atmosphere Model 5 (CAM5) and the Data Assimilation Research Testbed (DART), a community facility for ensemble data assimilation.  
The assimilation experiments illustrate the difficulties of using an integral quality as a model constraintm which are described in Section **.
In Section ** we demonstrate how the modeled excitation of Earth rotation variations by the angular momentum of the atmosphere can serve as a measure of the assimilation accuracy, by implementing atmospheric angular momentum as an evaluated but unassimilated observation type. 
\textcolor{alert}{[Fill in the rest]}.
