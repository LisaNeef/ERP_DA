%%%%%%%%%%%%%%%%%%%%%%%%%%%%%%%%%%%%%%%%%%%%%%%%%%%%%%%%%%%%%%%%%%%%%%%%%%%%
% AGUtmpl.tex: this template file is for articles formatted with LaTeX2e,
% Modified March 2013
%
% This template includes commands and instructions
% given in the order necessary to produce a final output that will
% satisfy AGU requirements.
%
% PLEASE DO NOT USE YOUR OWN MACROS
% DO NOT USE \newcommand, \renewcommand, or \def.
%
% FOR FIGURES, DO NOT USE \psfrag or \subfigure.
%
%%%%%%%%%%%%%%%%%%%%%%%%%%%%%%%%%%%%%%%%%%%%%%%%%%%%%%%%%%%%%%%%%%%%%%%%%%%%
%
% All questions should be e-mailed to latex@agu.org.
%
%%%%%%%%%%%%%%%%%%%%%%%%%%%%%%%%%%%%%%%%%%%%%%%%%%%%%%%%%%%%%%%%%%%%%%%%%%%%
%
% Step 1: Set the \documentclass
%
% There are two options for article format: two column (default)
% and draft.
%
% PLEASE USE THE DRAFT OPTION TO SUBMIT YOUR PAPERS.
% The draft option produces double spaced output.
%
% Choose the journal abbreviation for the journal you are
% submitting to:

% jgrga JOURNAL OF GEOPHYSICAL RESEARCH
% gbc   GLOBAL BIOCHEMICAL CYCLES
% grl   GEOPHYSICAL RESEARCH LETTERS
% pal   PALEOCEANOGRAPHY
% ras   RADIO SCIENCE
% rog   REVIEWS OF GEOPHYSICS
% tec   TECTONICS
% wrr   WATER RESOURCES RESEARCH
% gc    GEOCHEMISTRY, GEOPHYSICS, GEOSYSTEMS
% sw    SPACE WEATHER
% ms    JAMES
%
%
%
% (If you are submitting to a journal other than jgrga,
% substitute the initials of the journal for "jgrga" below.)

\documentclass[draft,jgrga]{agutex}
%\documentclass[jgrga]{agutex}

% To create numbered lines:

% If you don't already have lineno.sty, you can download it from
% http://www.ctan.org/tex-archive/macros/latex/contrib/ednotes/
% (or search the internet for lineno.sty ctan), available at TeX Archive Network (CTAN).
% Take care that you always use the latest version.

% To activate the commands, uncomment \usepackage{lineno}
% and \linenumbers*[1]command, below:

 \usepackage{lineno}
% \linenumbers*[1]

%  To add line numbers to lines with equations:
%  \begin{linenomath*}
%  \begin{equation}
%  \end{equation}
%  \end{linenomath*}
%%%%%%%%%%%%%%%%%%%%%%%%%%%%%%%%%%%%%%%%%%%%%%%%%%%%%%%%%%%%%%%%%%%%%%%%%
% Figures and Tables
%
%
% DO NOT USE \psfrag or \subfigure commands.
%
%  Figures and tables should be placed AT THE END OF THE ARTICLE,
%  after the references.
%
%  Uncomment the following command to include .eps files
%  (comment out this line for draft format):
  \usepackage[dvips]{graphicx}
%
%  Uncomment the following command to allow illustrations to print
%   when using Draft:
  \setkeys{Gin}{draft=false}
%
% Substitute one of the following for [dvips] above
% if you are using a different driver program and want to
% proof your illustrations on your machine:
%
% [xdvi], [dvipdf], [dvipsone], [dviwindo], [emtex], [dviwin],
% [pctexps],  [pctexwin],  [pctexhp],  [pctex32], [truetex], [tcidvi],
% [oztex], [textures]
%
% See how to enter figures and tables at the end of the article, after
% references.
%
%  Other packages:
\usepackage{amsmath}
\usepackage[usenames]{color}
\usepackage{ulem}
\usepackage{url}
\usepackage{lscape}
%\usepackage[pdftex]{graphicx}
%\usepackage{epstopdf}
%\usepackage{pgfplots}
\usepackage{stmaryrd}

% define some colors
\definecolor{unsure}{RGB}{27,158,119}
\definecolor{alert}{RGB}{217,95,2}
\definecolor{reviewer}{RGB}{117,112,179}
%
%
%% ------------------------------------------------------------------------ %%
%
%  ENTER PREAMBLE
%
%% ------------------------------------------------------------------------ %%

% Author names in capital letters:
\authorrunninghead{NEEF AND MATTHES}

% Shorter version of title entered in capital letters:
\titlerunninghead{ASSIMILATION OF INTEGRALS}

%Corresponding author mailing address and e-mail address:
\authoraddr{Corresponding author: L.J. Neef,
%Helmholtz -Zentrum f{\"u}r Ozeanforschung (GEOMAR),
D{\"u}sternbrooker Weg 20,
D-24105 Kiel, Germany
(neef@geomar.de)}

\begin{document}
\bibliographystyle{agu08}

%% ------------------------------------------------------------------------ %%
%
%  TITLE
%
%% ------------------------------------------------------------------------ %%


\title{Assimilation of Earth Rotation Parameter observations to constrain and evaluate atmospheric models}
%
% e.g., \title{Terrestrial ring current:
% Origin, formation, and decay $\alpha\beta\Gamma\Delta$}
%

%% ------------------------------------------------------------------------ %%
%
%  AUTHORS AND AFFILIATIONS
%
%% ------------------------------------------------------------------------ %%


%Use \author{\altaffilmark{}} and \altaffiltext{}

% \altaffilmark will produce footnote;
% matching \altaffiltext will appear at bottom of page.

 \authors{Lisa Neef\altaffilmark{1} and
Katja Matthes \altaffilmark{1}}

\altaffiltext{1}{Ocean Circulation and Climate Dynamics - Marine Meteorology,
   Helmholtz Centre for Ocean Research Kiel (GEOMAR), Kiel, Germany.}


%% ------------------------------------------------------------------------ %%
%
%  ABSTRACT
%
%% ------------------------------------------------------------------------ %%

% >> Do NOT include any \begin...\end commands withinA
% >> the body of the abstract.

\begin{abstract}
Variability in the Earth's rate of rotation and the orientation of its rotational pole is primarily driven by changes in atmospheric angular momentum on timescales shorter than a few years.
This implies that observations of Earth rotation parameters contain information about the dynamics of the atmosphere.
It has therefore been proposed to assimilate these parameters into atmosphere and ocean models, as an additional observable constraint for prediction and model evaluation.
In this study we test this idea by performing perfect model experiments, assimilating Earth rotation parameters into the Community Atmosphere Model using an Ensemble Adjustment Kalman Filter.
This is tantamount to assimilating the global integrals of the wind and surface pressure fields. 
We show that the assimilation of Earth rotation parameters is only able to reduce error in a model ensemble \textcolor{unsure}{under very limited conditions}.
\textcolor{unsure}{Generally, the assimilation of Earth rotation paramters causes the ensemble filter to diverge because the constraint imposed by the observations is too weak to stop the growth of nonlinearities.}
It is also found that the  assimilation of Earth rotation parameters does not add value to the assimilation of more conventional, localized observations, but actually causes the filter to diverge from the true state because \textcolor{alert}{FILL IN REASON}.
However, including atmospheric angular momentum in an assimilation experiment as evaluated variables can provide a useful measure for the fidelity of an assimilation system where the true state is not known.  
The results found here could be extended to other cases where integral or averaged observations are assimilated.

\end{abstract}

%% ------------------------------------------------------------------------ %%
%
%  BEGIN ARTICLE
%
%% ------------------------------------------------------------------------ %%

% The body of the article must start with a \begin{article} command
%
% \end{article} must follow the references section, before the figures
%  and tables.

\begin{article}

%% ------------------------------------------------------------------------ %%
%
%  MATH ABBREVIATSIONS
%
%% ------------------------------------------------------------------------ %%
\newcommand{\degree}{\ensuremath{^\circ}}


%% ------------------------------------------------------------------------ %%
%
%  TEXT
%
%% ------------------------------------------------------------------------ %%

\section{Introduction}

% common ground -- DA is now accessible for climate models and enables us to use a variety of osbservations to evaluate and constrain models
The assimilation of data into numerical models of the atmosphere and ocean has now evolved from its original purpose in numerical weather prediction, to the application in climate modeling. 
Data assimilatiom methods and codes are now accessible non-experts, in the form of out-of-the-box assimilation toos tools such as the Data Assimilation Research Testbed \citep[DART]{Anderson2009} or the Parallel Data Assimilation Framework \citep[PDAF]{Nerger2013}.

% CG zoomed in: earth rotation observations in particular have been found to deliver unique information about the atmosphere [examples...] -- it seems natural to use formal DA to really figure out what these obs have to say about our models, and in fact Saynisch et al have done this.
The evolution of data assimilation algorithms and tools enables us to use various observations of the Earth system to constrain and evaluate climate models.
Space geodesy, i.e. the measurement of the figure, position, and geoid of the Earth from satellites, very long baseline interferometry, and satellite and lunar laser ranging, offers a unique source of information about the Earth system.  

In particular, it has been found that the angular momenta of the atmosphere and ocean can be observed in terms of anomalies in the Earth's polar orientation and rotation rate, the so-called Earth Rotation Parameters (ERPs).
For example, \citet{Neef2014} showed that sudden stratospheric warmings are typically preceded by a circa 60 milli-arcsecond wobble in the polar orientation, 3-4 weeks before the warming onset.
\textcolor{alert}{[Insert another example of how the atmosphere has footprints in the ERPs.]}

It makes sense to ask whether more clear connections between observed Earth rotation variations and atmosphere/ocean models can be drawn by formally assimilating these parameters into models. 
This was done by \cite{Saynisch2011,Saynisch2010,Saynisch2012}, who assimilated the observed ERP timeseris into various ocean model simulations, found that the observations implied that large adjustments in the model's mass distribution and currents are are necessary in order for the modeled ocean angular momentum to agree with observations.
However, those studies did not show how much of this change was due to the addition of the integral observations, nor whether or not the ERP observations actually added information to gridpoint observations that were also assimilated.


% DC1: it's unclear whether integral observations can be useful constraints in the first place.
It is not clear whether observations like the ERPs can indeed usefully constrain the modeled state, because the ERPs represent integrals of the modeled wind (or current) and mass fields, rather than local quantities.
While the prospect of assimilating non-gridpoint observations like ERPs is an exciting possibility, these observations represent integrals of the model state and their assimilation with standard algorithms is therefore not straightforward.


An assimilation algorithm looks for an adjustment to a model that brings it closer to the observations being assimilated. 
However, even though the resulting model adjustment may improve the fit to the assimilated observations, it will not necessarily bring the modeled state closer to the truth.


%% PP1: in this paper we evaluate whether ERPs are suited for data assimilation and show some of the difficulties associated with assimilating an integral quantity
It is not sufficient to evaluate the fit between the model and the observations, since the assimilation algorithm is designed to fit the observations. In so-called Observing System Simulation Experiments (OSSEs) a model is used to simulate a "true" state, from which observations are generated and assimilated back into either the same or another model. OSSEs make it possible to evaluate an observation configuration and/or an assimilation technique because the true state is known, and can be directly compared to the analysis model state.
In this study we perform OSSEs, using the community data assimilation facility DART, to evaluate whether ERPs are suited for data assimilation and show some of the difficulties associated with assimilating an integral quantity as a model constraint.


%% DC2: In realstic experiments where we don't know the truth, we need to evaluate using indepenent variables, which can be hard to find.

In non-idealized assimilation experiments, it is difficult to judge whether the assimilation of a given set of observations has actually succeeded in bringing the estimated state closer to the truth. 
Here it is important to compare the assimilated state with independent observations. However, these are often difficult to obtain because different observing systems often don't overlap sufficiently in time and space.

%% PP2: ERPs are actually really well suited for this, because they represent 3 easy numbers that capture the entire model state.
In Section ** we demonstrate how the modeled excitation of Earth rotation variations by the angular momentum of the atmosphere can be used as a measure of the assimilation efficacy, by implementing atmospheric angular momentum as an evaluated but unassimilated observation in  DART.



%-------------------------------------METHODS-----------------------------------------------------------------

\section{Methodology}
%-------------------MODEL-------------------------
\subsection{Atmospheric Model}
\label{sec:CAM}
To evaluate Earth rotation data as an assimilation variable, we assimilate synthetic observations into the Community Atmosphere Model 5 \citep[CAM5]{nealeetal2010}, which forms the atmospheric component of the Community Earth System Model. 
We have run CAM with a finite-volume dynamical core, $1.9^{\text{o}} \times 2.5^{\text{o}}$ horizontal resolution, and  30 hybrid-coordinate vertical levels, the highest of which is near 3 hPa.
The top three model levels (starting at about 14 hPa) constitute a ``sponge'' layer, where horizontal diffusion is applied to temperature, vorticity, and divergence in order to absorb vertically propagating planetary waves.  
The diffusion has been tuned in order to give a reasonable strength of the stratospheric polar night jets \citep{nealeetal2010}.

%-------------------DAS-------------------------


\subsection{Data Assimilation System}
\label{sec:DAS}
The Data Assimilation Research Testbed  \citep[DART]{Anderson2009, Raeder2012} provides an ensemble-based data assimilation interface for CAM (among many other atmosphere and ocean models). 
Our experiments use the Ensemble Adjustment Kalman Filter (EAKF) \citep{anderson2001}, which is one of several ensemble assimilation algorithms available within DART.
The following section outlines the basic algorithm of this ensemble filter.

To assimilate observations, the filter requires an ensemble of $N$ simulations to be advanced forward to the time at which a given observation becomes available. 
At this time, the filter unites two basic quantities: the observation itself, $y_{\text{obs}}$, and $N$ values of the observation that are predicted by the ensemble ($y_{b}^{n}$, where $b$ denotes the \textit{background} estimate and $n$ denotes an individual ensemble member).
Bayes' theorem states that the conditional probability distribution of the observation, given the prior ensemble estimate on the one handm, and the physical measurement on the other, is the product of their respective probability distributions.
The resulting joint probability distribution has an updated variance $\sigma_a^2$  (where $a$ denotes the update or \textit{analysis}):
\begin{eqnarray}
 \sigma_a^2 = 
\left[
\left( \sigma_b^2  \right)^{-1}+
\left( \sigma_{\text{obs}}^2  \right)^{-1}
\right]^{-1},
\label{eq:sigma_a}
\end{eqnarray}
where $\sigma_b^2$ is  the prior error variance of the observation implied by the ensemble, and $\sigma_{\text{obs}}^2$ is the error variance of the observation itself (i.e. the measurement error).
The ensemble mean of this joint probability distribution is given by
\begin{eqnarray}
\left< y_a \right> = \sigma_a^2 
\left(
\frac{\left< y_b \right>}{\sigma_b^2} +
\frac{y_{\text{obs}} }{\sigma_{\text{obs}}^2} 
\right).
\end{eqnarray}
This means that the observation value of each ensemble member should be updated to the following:
\begin{eqnarray}
 y_{a}^n = 
\left( \frac{\sigma_a^2}{\sigma_b^2}  \right)
\left(
y_{b}^n - \left< y_b  \right> \right)
+ \left< y_a \right>,
\end{eqnarray}
i.e. a linear transformation of the ensemble members about the updated mean such that (\ref{eq:sigma_a}) is satisfied by the ensemble \citep{andersoncollins2006}.

Practically, of course, we don't update the \textit{observation} implied by the ensemble, but rather the \textit{state} of each ensemble member (in our case the wind, surface pressure, and temperature fields). 
To update the state vector of each ensemble member, ${\bf x}^n$ with the information from the observation, we simply project the observation space update for each ensemble member ($\Delta y^n = y_{a}^n-y_{b}^n$) onto the individual state components $x_i$ via linear regression:
\begin{eqnarray}
 \Delta x_{i}^n = 
\left(
\frac{c_{x_iy}}{\sigma_b^2}
\right)
\Delta y^n,
\label{eq:state_update}
\end{eqnarray}
where $c_{x_iy}$ represents the prior covariance between the state component $x_i$ and the observation $y$.

%---cut---(it's not really relevant)
%Because the EAKF adjusts the ensemble based on the expected error reduction, it is called a stochastic ensemble filter.
%In contrast, so-called stochastic filters, such as the well-known Ensemble Kalman Filter \citep{evensen2003}, adjust the forecast ensemble first with perturbed upservations, and then compute the resulting error distribution from the ensemble.

Ensemble assimilation algorithms are novel because they estimate the covariance and variance terms in the above equations using an ensemble of model simulations, which means that these quantities can vary in time and space (according to the physical relationships simulated in the model), and are updated with new information whenever a new observation comes in.  
For example, the covariance between model state component $x_i$ and observation $y_j$ is given by
\begin{eqnarray}
c_{x_iy_j} &=& 
\left<
e_{x_i}^n 
e_{y_j}^n
\right>,
\label{eq:covariance} 
\end{eqnarray}
%
%\sigma_b^2 &=& 
%\left<
%\left( y_{b,n} - \left< y_b \right>   \right)^2
%\right>  
%\label{eq:sigma_b} \\
%
%\sigma_a^2 &=& 
%\left<
%\left( y_{a,n} - \left< y_a \right>   \right)^2
%\right>,  
%\label{eq:sigma_a} 
%
where the brackets $\left< \cdot \right>$ represent the ensemble mean and
$e_{x_i}^n = x_i^n - \left< x_i^n \right>   $ and $e_{y_j}^n = y_j^n - \left< y_j^n \right> $ are the deviations of each ensemble member (denoted by the superscript $n$) from the ensemble mean.

In a successful ensemble assimilation system, these terms should reflect the true error statistics of the model system.
If this is not the case, the ensemble filter will diverge, a condition where the uncertainty predicted by the ensemble underestimates the true error, which eventually leads to the rejection of new observations.
\textcolor{alert}{Note: in our experiments, this kind of filter divergence is not really the problem, so maybe rephrase what I said here, or cut it completely.}

%

%-------------------observations-------------------------

\subsection{Synthetic Observations}

To evaluate the impact of Earth rotation observations relative to more conventional (localized) observations in a data assimilation system, we generate synthetic observations from a reference model simulation that we call the "truth", then assimilate these observations into a an ensemble of model simulations spread around the truth.  

The truth run comes from a reference CAM simulation that starts on 1 Jan 2008 and uses observed sea surface temperatures as a boundary condition. 
We generate observations of this run from 1 January to 28 February 2009 of this simulation. 


\subsubsection{Atmospheric Angular Momentum Observations}
\label{sec:AAM}
\citet{barnesetal1983} derived the following three equations that describe variations in the three components of atmosphere angular momentum:  
\begin{eqnarray}
%X1
{\bf \chi}_{1}(t) &=& \frac{1}{\Omega \left( C_m-A_m \right)\left(1-\frac{k_2}{k_s}  \right)}
\left[ \left(1+k_l \right) \Omega \Delta {\bf I}_{13}(t) + \Delta h_1(t)  \right] \label{eq:X1} \\
%X2	
{\bf \chi}_{2}(t) &=& \frac{1}{\Omega \left( C_m-A_m \right)\left(1-\frac{k_2}{k_s}  \right)}
\left[ \left(1+k_l \right) \Omega \Delta {\bf I}_{23}(t) + \Delta h_2(t)  \right] \label{eq:X2} \\
%X3	
{\bf \chi}_{3}(t) &=& \frac{1}{\Omega C_m \left(1+\frac{4k_2}{3k_s}\frac{C-A}{C} \right)}
\left[ \left(1+k_l \right) \Omega \Delta {\bf I}_{33}(t) + \Delta h_{3}(t) \label{eq:X3} \right].
\end{eqnarray}
%
$\chi_1$ and $\chi_2$ represent the angular momentum vector components defined by the intersection of the equator with the $180\degree$W and $0\degree$ meridians, respectively, and $\chi_3$ reprents the axial component.
%
The constants 
$k_2 = 0.295$, 
$k_s = 0.938$, and 
$k_l = -0.301$
represent the rotational, secular and load Love numbers, which quantify the rotational deformation of the Earth.
$\Omega = 7.292115\times 10^{-5} \text{rad}/\text{s}$ is the average rotation rate  of the Earth. 
$C = 8.0365 \times 10^{37} \text{kg} \text{m}^2$ and $A = 8.0101 \times 10^{37} \text{kg} \text{m}^2$ represent the (3,3) and (1,1) components of the moment of inertia of the solid Earth, and $C_m = 7.1237\times 10^{37}  \text{kg} \text{m}^2$ and $A_m = 7.0999\times 10^{37} \text{kg} \text{m}^2$ are the corresponding moments of inertia of the mantle and crust only (they are used in the above equations to decouple the core and mean mantle motion).  



The ${\bf I}_{ij}$ terms represent the components of the atmospheric inertia tensor:
\begin{eqnarray}
  I_{13} &=& -\int R^2 \cos \phi \sin \phi \cos \lambda dM 
  \label{eq:I1}\\
  I_{23} &=& -\int R^2 \cos \phi \sin \phi \sin \lambda dM 
  \label{eq:I2}\\
  I_{33} &=&  \int R^2 \cos^2 \phi dM ,
  \label{eq:I3}
\end{eqnarray}
and the $h_i$ terms the relative angular momentum (due to wind) in each direction:
\begin{eqnarray}
  h_{1}  &=& -\int R \left[u \sin \phi \cos \lambda - v \sin \lambda \right] dM 
    \label{eq:h1}\\
  h_{2}  &=& -\int R \left[u \sin \phi \sin \lambda + v \cos \lambda \right] dM 
    \label{eq:h2}\\
  h_{3}  &=&  \int R u \cos \phi dM.
    \label{eq:h3}
\end{eqnarray}
%
$R = 6371.0$ km is the radius of the Earth, $\Omega = 7.292115\times 10^{-5} \text{rad}/\text{s}$ the average rotation rate, and $g = 9.81 \text{m}/\text{s}^2$ is the acceleration due to gravity.

Note that the axial terms [(\ref{eq:I3}) and (\ref{eq:h3})] weight all longitudes equally in the integration.
This means that mass anomalies usually cancel each other out in the global integral, which means that the axial wind excitation function $h_3$ dominates the axial angular momentum  \citep{barnesetal1983}.
%
Nearly the opposite is true for the equatorial angular momentum terms [(\ref{eq:I1})-(\ref{eq:I2}) and (\ref{eq:h1})-(\ref{eq:h2})], where zonally-asymmetric fields add up to a larger global integral.
For these functions, the mass terms ($I_1$ and $I_2$) are typically several factors larger than the wind terms ($h_1$ and $h_2$)  \citep{barnesetal1983}.

The longitudinal terms in the equatorial mass integrals [(\ref{eq:I1}) and (\ref{eq:I2})] means that $\chi_2$ is weighted more strongly over the continents while $\chi_1$ is weighted more strongly over the oceans.
Consequently, $\chi_2$ has a pronounced annual cycle due to the yearly appearance of the Siberian High \citep{dobslawetal2010}, and tends to show strong negative anomalies in the month preceding a sudden stratospheric warming \citep{Neef2014}.
$\chi_1$ has much weaker subseasonal to annual variations because surface pressure variations cause corresponding displacements of the ocean surface, which even out the angular momentum changes (the so-called ``inversted barometer'' effect, e.g. \citet{salsteinrosen1989}).

In reality, of course, we don't measure the atmospheric angular momentum but rather the variations in the Earth rotation parameters that variations in the angular momentum excite. 
To map these to equivalent angular momentum variations, one would rotate the inertial reference frame, convert observed length-of-day anomalies into corresponding axial angular momentum, and subtract out the excitation of each parameter that is due to other components of the Earth system.
For the purposes of this study, however, it is sufficient to assimilate the three angular momentum components directly, since we are performing perfect model experiments and therefore know the true state.
%---cut this part: since we assimilate the X's, we don't need to talk about polar motion
%The non-dimensional representation of the angular momentum in (\ref{eq:X1})-(\ref{eq:X3}) makes it easy to map angular momentum changes to equivalent changes in the polar orientation (or polar motion) and the rate of Earth's rotation. 
%Polar motion is measured in terms of two angles, $p_1$ and $p_2$, that represent the location of the Earth's rotational axis in an inertial, celestial reference frame that is fixed in space and defined relative to a group of stars (the so-called celestial ephemeris pole).
%\citet{barnesetal1983} and later \citet{Gross1992} showed that these vectors can be directly related to unit variations in the equatorial components of the Earth's angular momentum ($\chi_1$ and $\chi_2$) via a rotation into the inertial reference frame of the so-called Chandler wobble (a free nutation of the Earth of frequency 
%$\sigma_0 = 2\pi/ 433\text{d}$, which results from the oblateness of the Earth's figure):
%\begin{eqnarray}
%  p_1 + \frac{\dot{p_2}}{\sigma_0} &=& \chi_1 \\
%  -p_2 + \frac{\dot{p_1}}{\sigma_0} &=& \chi_2.
%\label{eq:X12_to_PM}
%\end{eqnarray}
%where the overdots represent time derivatives, $\sigma_0 = 2\pi/ 433\text{d}$ denotes the Chandler frequency.

Nevertheless, it is conceptually helpful to transform the axial angular momentum $\chi_3$ into corresponding anomalies in the length of a day ($\Delta$LOD), which is done using the following relationship:
\begin{eqnarray}
\Delta \text{LOD} &=& \Delta \chi_3 \times \text{LOD}_0 ,
\label{eq:X3_to_LOD}
\end{eqnarray}
where $\text{LOD}_0$ denotes the nominal length-of-day (86400s).  
%----cut----The length-of-day anomalies are observed as $\Delta \text{LOD}  \equiv \text{UT1} - \text{IAT}$ where UT1 denotes the universal time measured by geodetic techniques and IAT is a reference time based on atomic clock measurements.

For our experiments, observations of $\chi_1$, $\chi_2$, and $\chi_3$ are generated every 24 hours, reflecting the observation frequency of the real Earth rotation data series published by the International Earth Rotation Services \citep{iers}.  


\subsubsection{Idealized Radiosonde Observations}
\label{sec:radiosondes}

We compare the angular momentum observations described above to spatially localized observations that are similar to the types of observations used in  reanalysis \citep{Dee2005}.
We simulate the "conventional" observation grid simply using an idealized global grid (Fig. \ref{fig:RS}) of radiosonde measurements of temperature, which amounts to about 27000 observations per day.  



%-------------------experiments-------------------------


\subsection{Assimilation Experiments}
\label{sec:experiments}


\subsubsection{Ensemble Generation}

\textcolor{alert}{[Insert description of how we generated the true state.]}
All assimilation runs were performed with an 80-member ensemble.
The ensemble is generated by selecting the 1 January restart files from an 80-year simulation of CAM5, and then running a year of assimilation of the synthetic radiosonde observations (Section \ref{sec:radiosondes}).

\subsubsection{Experiment Overview}
% experiment types
Four assimilation experiments are performed, summarized in Table \ref{tab:expts}.
As a reference experiment, we have run the 80-member CAM ensemble forward for two months with no assimilation.
In the second experiment, the ensemble is run forward as before, but assimilating the three angular momentum components observed in the "true state" every 12 hours.
Borth of these experiments were integrated for two months, 1 January to 28 February 2009.  
In the third experiment, the temperature observations from the idealized radiosonde grid are assimilated every six hours. 
In the fourth experiment, we assimilate both the 6-hourly radiosonde temperatures and the three 12-hourly global angular momentum components.
These latter two experiments were only integrated for the first 31 and 17 days of this period, respectively, because this amount of integration time was found to be sufficient to give the results presented in this study.

\subsubsection{Output Diagnostics}
Our main evaluation diagnostics are the square error between the simulated true state and the ensemble mean
\begin{eqnarray}
E_M = \left(
\left< x_i \right>-x_{i,t}
\right)^2
\end{eqnarray}
and the ensemble variance about its mean:
\begin{eqnarray}
S = 
\frac{1}{N}
\sum_{n=1}^N
\left(
\left< x_{i,n} \right>-x_t
\right)^2,
\end{eqnarray}
where $x_{i,n}$ represents a component of the state vector in the ensemble,  $x_{i,t}$ the corresponding component in the truth, and $\left< \cdot \right>$ represents the ensemble mean, as in section \ref{sec:DAS}.

In order to evaluate whether the ensemble is an accurate representation of the true uncertainty, we  test whether the truth can be considered a sample of the probability distribution represented by the ensemble.
 \citet{Huntley2009} and \citet{Murphy1988} pointed out that this is the case when 
\begin{eqnarray}
\llbracket E_M \rrbracket = \frac{N+1}{N} \llbracket S \rrbracket, 
\label{eq:EvsS}
\end{eqnarray}
where the square brackets represent spatial averages. 

A second way to quantify whether the ensemble and the truth come from the same probability distribution is to compute rank histograms \citep[and references therein]{Hamill2001}
A rank histogram is generated by ordering the values of the ensemble at each point in the state space, and then finding the rank of some verification value on this list.
In our case, we simply use the true state as the verification, since it is known.
The rank of the verification is then counted up over many different assimilation times and state space points; we then generate a histogram of these ranks.
If the truth comes from the same PDF as the ensemble, the rank histogram should be nearly flat.  
If the truth is a frequent outlier of the PDF represented by the ensemble, the rank histogram will by concave.
A convex rank histogram indicates an ensemble whose spread is so large that the truth is usually the central rank. 



%-------------------------------------RESULTS 1- integral obs suck------------------------------------------------------
\section{AAM as an Assimilation Variable}

 
Figure \ref{fig:fit_to_ERPs} compares the modeled angular momentum functions for Experiments \NODA-\ERPRST (similar to \ref{fig:evalvariable_aam}, and again omitting $\chi_1$ for simplicity). 
The angular momentum components for each experiment are also compared to the ``true'' angular momentum in each case, which is assimilated in \ERPALL ~and \ERPRST.  

With no assimilation (\NODA, first column), the angular momentum functions again show how the ensemble spreads about the truth, and how the spread saturates after about one month.
If we assimilate the angular momentum functions alone (\ERPALL, second column), the ensemble predictably clusters close to the true angular momentum, and captures the day-to-day angular momentum variations. 
The difference between the first two columns indicates that assimilating the angular momentum has imposed some kind of constraint upon the wind, temperature, and surface pressure fields, though we do not yet know whether those fields have also moved closer to the true state. 

The ensemble clusters even more tightly around the truth when instead of the angular momentum functions we assimilate local temperature observations (\RST, third column), which is a much stronger constraint on the model fields. 
Finally, adding the angular momentum observations to the regularly-spaced temperature observations (\ERPRST, fourth column) slightly increases the agreement between the ensemble and the true state further.  
This suggests that the angular momentum observations may contain additional information that complemetns the information in the temperature observations, though the difference between \RST ~(assimilating only temperature) and \ERPRST ~(assimilating temperature and angular momentum) is small.  

\subsection{State-space diagnostics}
\label{sec:erpda}

To evaluate how the assimilation of the 3 AAM components changes the ensemble mean analysis state, we compare the assimilation of the three AAM components only (AAMDA) to the ensemble with no assimilation (NODA).
Figures \ref{fig:MSE_U} and \ref{fig:MSE_PS} both compare the mean square error (left column) to the ensemble variance scaled by $(N+1)/N$ (as in (\ref{eq:EvsS})), for experiment NODA (top row), experiment AAMDA (second row), and the difference between the two.
Figure \ref{fig:MSE_U} shows vertical profiles of mean square error or ensemble variance of the zonal wind, as a function of time (averaging meridionally and zonally). 
To focus on the surface signals, Figure \ref{fig:MSE_PS} shows the mean square error and ensemble variance of the surface pressure,  as a function of latitude and time (averaging vertically and zonally). 

% (*) error in NODA grows in the tropopsheric jets and at the highest model levels - the truth diverges from the ensemble mean where also the spread grows
Without assimilation (Fig. \ref{fig:MSE_U}a) the zonal wind in the true state diverges from that of the ensemble mean around the height of the extratropical jets (ca. 250hPa) and in the top four model levels (10hPa and above). 
The spread of the ensemble in this case (Fig. \ref{fig:MSE_U}b) looks similar but is slightly smaller. 
This implies that the true state probably has features that are averaged out in the ensemble mean, and the ensemble spread slightly underestimates the difference.

% (*) adding AAM obs reduces the spread in the 2 areas where it's largest
Assimilating observations of the three AAM components shows a similar mean error (Fig. \ref{fig:MSE_U}c) and ensemble variance (Fig. \ref{fig:MSE_U}d).
The difference between the two experiments (Fig. \ref{fig:MSE_U}e-f)  shows that the ensemble variance is reduced everywhere, and most strongly in the places where the initial spread was largest, the midlatitude jets and the highest levels. 
% (*) but: adding AAM really only reduces the error up high, and not consistently
However, the mean error is only reduced at the highest model levels, and over a few short pulses. 
Twice during the run time, the assimilation even increases the mean error. 

\textcolor{alert}{Also insert description of what happens at the surface, or 300 hPa or something, if relevant.}


\subsubsection{Rank histograms to diagnose filter divergence}

\textcolor{alert}{Here insert description of rank histograms before and after assimilation in various locations -- they will probably show that adding the integral observations increases filter divergence everywhere.}


\subsection{Evolution of the  Covariance Field}

Assimilation of the three global AAM components increases the error between the ensemble mean and the true state, especially as the data assimilation progresses, evenually causing the filter to diverge.
To understand why this happens, we can examine the ensemble-estimated covariance between local variables (in our case, winds or surface pressure) and the global AAM functions (\ref{eq:covariance}), which govern how and where each ensemble member is updated when an observation comes in. 

The covariance $c_{x_iy}$ is estimated for each state variable $x_i$ and each observation $y$ by the statistics of the model ensemble.  
A point in the model state can have a large covariance with the global AAM either if it has a large variance, or if it has a large correlation to the global AAM, or both.


Even when a given type of observation is unable to significantly constrain a modeled state by itself, it may still improve the assimilation of other, more standard observations.  
If angular momentum / Earth rotation parameter observations are assimilated in the presence of local observations, it is conceivable that an ensemble that is already clustered around the true state could perhaps be pushed even closer to the truth by adding the additional requirement that the global angular momentum should match observations. 
In this section, we compare the 80-member CAM ensembles assimilating synthetic temperature observations (column 3 of Fig. \ref{fig:fit_to_ERPs}) to an ensemble assimilating both temperature and additional angular momentum observations (column 4 of Fig. \ref{fig:fit_to_ERPs}).
Figure \ref{fig:fit_to_ERPs} shows that assimilating 
temperature observations alone is enough to achieve a good fit to the true angular momentum, but also that 
then adding observations of the angular momentum improves the fit further, which implies that  
the angular momentum observations add information to the ensemble. 
The question remains as to whether the information is interpreted correctly by the model.

% (a) what happens to the spread?  
Figure \ref{fig:added_value_MSE}a shows the increment (posterior-prior) in the ensemble variance in the zonal wind for both experiments, averaged globally. 
The increment in the ensemble spread is always negative [following (\ref{eq:sigma_a}), and adjusts from an initially weak value to a stronger steady-state value as the ensemble, which was at the initial time constrained by both wind and temperature observations, adjusts to being constrained only by temperature observations. 
During this early period, adding the angular momentum observations causes a slightly stronger adjustment of the ensemble variance, indicating that the angular momentum observations add some information to the state estimate. 

% (b) but the error increments look really different -- here we see sampling error  
However, the posterior-prior difference in the mean square error of the ensemble mean (Fig. \ref{fig:added_value_MSE}b) shows that the information imposed by the angular momentum observations is not imposed correctly: while posterior error is also consistently lower than prior error, adding the angular momentum oobservations mostly results in a weaker error reduction during the adjustment period. 
Once the ensemble variance has reached steady state (after about 9 Jan), the additional angular momentum observations do not add any additional certainty (Fig. \ref{fig:added_value_MSE}a), and only sometimes cause stronger global mean error reduction. 

% (c) what does the total mean square error show? -- adding the ERPs increases error at worst and does nothing at best
The absolute posterior mean square error (Figure \ref{fig:added_value_MSE}c) is consistently larger when the angular momentum observations are added: the additional observations increase the global mean error at worst (at the beginning of the assimilation), and have no impact at best (at steady-state). 
We therefore find no clear benefit of assimilating atmospheric angular momentum (or Earth rotation parameter) observations, either with or without the assimilation of conventional observations. 





\section{AAM as an evaluation variable}

%% uncomment if we do end up showing something about gpsro
GPS-RO measurements have been shown to add a high amount of information to numerical weather forecasts (Bonavita, 2013), due to their low systematic errors and high vertical resolution. It was shown by (Wang et al., 2013) that GPS-RO data give an unprecedented look into the vertical structure of the atmosphere because of their high vertical resolution, in this case showing a warming trend in the tropopause inversion layer that would not have been observed otherwise.





%------------------------------------summary and conclusions-------------------------
\section{Summary}  

In this study we tested the efficacy of assimilating Earth rotation parameters, which represent changes in the atmospheric angular momntum and therefore integrals of the atmospheric wind and pressure fields.  
This was done by performing perfect model assimilation experiments wherein the three components of the global atmospheric angular momentum were assimilate, with and without complementary meteorological observations. 

It may be assumed that the assimilation integral quantities can lead to an overall improvement of a modeled atmosphere or ocean state.
Indeed, this was done by \citet{Saynisch2010,Saynisch2011,Saynisch2012} for Earth rotation parameters and by \textcolor{alert}{[CITE other examples of integral constraints]}.
However, we found that an improved state estimate is difficult to achieve when integral-type observations are assimilated, \textcolor{unsure}{because the assimilation decreases ensemble spread without improving the accuracy of the ensemble.}


It was found that the assimilation of the three components of AAM is able to decrease state error in an ensemble of simulations, but only during the initial spin-up period when both  the spread of the ensemble and the true error are quite small
As both the error and spread grow, it becomes too easy to fit the observed AAM without actually bringing the ensemble closer together.

In Kalman filter-based data assimilation, the state is adjusted most strongly in regions where the covariance between the state components and each observation is highest, which means that it depends both on the model spread at each point in the state vector, as well as the correlation of that point to the observation.  
Thus points with a large ensemble spread recieve a stronger adjustment than points with a smaller spread, which may not always reflect the true situation (i.e. points with a large true error are not adjusted enough), but as long as the correlation is estimated correctly, the adjustment will be made in the right direction.  

\textcolor{unsure}{The correlations should be estimated accurately if the model and observation operator are perfect and the ensemble is large enough (three conditions that are guaranteed in our experiments).  
However, our experiments showed that the modeled correlations between state components and the AAM observations become less and less accurate as the assimiliation progresses. 
This means that the ensemble becomes less and less represenative of the true error in time, i.e. that the truth and the forecast ensemble are no longer members of the same probability distribution. 
 }


These results lead to the possible conclusion that angular momentum oscillations might better constrain the state when assimilated in conjunction with local state observations that keep the spread in the ensemble small. 
However, we found that in this case, the additional assimilation of angular momentum degrades the analysis.    
\textcolor{alert}{Fill in the reason for this when I really understand it.}

There exists a potential way to assimilate integral observations, based on the study of \citet{Dirren2005}, but applying this method is doubtful to offer improvements upon the assimilation of other available observations, such as GPS-RO measurements.  
\textcolor{alert}{Need to explain why I think so.}


We have found that the real usefulness of Earth rotation observations is that they can measure the 
fidelity of a data assimilation system because they offer a  way to quantify the global state error in terms of three simple components that all reflect different things.  



%%% End of body of article:

%%%%%%%%%%%%%%%%%%%%%%%%%%%%%%%%
%% Optional Appendix goes here
%
% \appendix resets counters and redefines section heads
% but doesn't print anything.
% After typing  \appendix
%
% \section{Here Is Appendix Title}
% will show
% Appendix A: Here Is Appendix Title
%
%%%%%%%%%%%%%%%%%%%%%%%%%%%%%%%%%%%%%%%%%%%%%%%%%%%%%%%%%%%%%%%%
%
% Optional Glossary or Notation section, goes here
%
%%%%%%%%%%%%%%
% Glossary is only allowed in Reviews of Geophysics
% \section*{Glossary}
% \paragraph{Term}
% Term Definition here
%
%%%%%%%%%%%%%%
% Notation -- End each entry with a period.
% \begin{notation}
% Term & definition.\\
% Second term & second definition.\\
% \end{notation}
%%%%%%%%%%%%%%%%%%%%%%%%%%%%%%%%%%%%%%%%%%%%%%%%%%%%%%%%%%%%%%%%
%
%  ACKNOWLEDGMENTS

\begin{acknowledgments}
The computations discussed in this study were performed at the Deutsches KlimaRechenZentrum (DKRZ) in Hamburg, Germany.
We are grateful to \hl{Nick Pedatella} for providing the data from the DART-WACCM experiments in Table \ref{tab:expts}, and to
Nancy Collins for help with porting DART to the DKRZ supercomputer.
This work has been performed within the Helmholtz-University Young Investigators Group NATHAN, funded by the Helmholtz-Association through the President's Initiative and Networking Fund, the GEOMAR Helmholtz Centre for Ocean Sciences Kiel, the Helmholtz Centre Potsdam GFZ German Research Centre for Geosciences, and Freie Universit\"at Berlin.


\end{acknowledgments}

%% ------------------------------------------------------------------------ %%
%%  REFERENCE LIST AND TEXT CITATIONS
%
%----temporary bibliography
\bibliography{everything}
%----temporary bibliography

% Either type in your references using
% \begin{thebibliography}{}
% \bibitem{}
% Text
% \end{thebibliography}
%
% Or,
%
% If you use BiBTeX for your references, please use the agufull08.bst file (available at % ftp://ftp.agu.org/journals/latex/journals/Manuscript-Preparation/) to produce your .bbl
% file and copy the contents into your paper here.
%
% Follow these steps:
% 1. Run LaTeX on your LaTeX file.
%
% 2. Make sure the bibliography style appears as \bibliographystyle{agufull08}. Run BiBTeX on your LaTeX
% file.
%
% 3. Open the new .bbl file containing the reference list and
%   copy all the contents into your LaTeX file here.
%
% 4. Comment out the old \bibliographystyle and \bibliography commands.
%
% 5. Run LaTeX on your new file before submitting.
%
% AGU does not want a .bib or a .bbl file. Please copy in the contents of your .bbl file here.

\begin{thebibliography}{}

%\providecommand{\natexlab}[1]{#1}
%\expandafter\ifx\csname urlstyle\endcsname\relax
%  \providecommand{\doi}[1]{doi:\discretionary{}{}{}#1}\else
%  \providecommand{\doi}{doi:\discretionary{}{}{}\begingroup
%  \urlstyle{rm}\Url}\fi
%
%\bibitem[{\textit{Atkinson and Sloan}(1991)}]{AtkinsonSloan}
%Atkinson, K., and I.~Sloan (1991), The numerical solution of first-kind
%  logarithmic-kernel integral equations on smooth open arcs, \textit{Math.
%  Comp.}, \textit{56}(193), 119--139.
%
%\bibitem[{\textit{Colton and Kress}(1983)}]{ColtonKress1}
%Colton, D., and R.~Kress (1983), \textit{Integral Equation Methods in
%  Scattering Theory}, John Wiley, New York.
%
%\bibitem[{\textit{Hsiao et~al.}(1991)\textit{Hsiao, Stephan, and
%  Wendland}}]{StephanHsiao}
%Hsiao, G.~C., E.~P. Stephan, and W.~L. Wendland (1991), On the {D}irichlet
%  problem in elasticity for a domain exterior to an arc, \textit{J. Comput.
%  Appl. Math.}, \textit{34}(1), 1--19.
%
%\bibitem[{\textit{Lu and Ando}(2012)}]{LuAndo}
%Lu, P., and M.~Ando (2012), Difference of scattering geometrical optics
%  components and line integrals of currents in modified edge representation,
%  \textit{Radio Sci.}, \textit{47},  RS3007, \doi{10.1029/2011RS004899}.

\end{thebibliography}

%Reference citation examples:

%...as shown by \textit{Kilby} [2008].
%...as shown by {\textit  {Lewin}} [1976], {\textit  {Carson}} [1986], {\textit  {Bartholdy and Billi}} [2002], and {\textit  {Rinaldi}} [2003].
%...has been shown [\textit{Kilby et al.}, 2008].
%...has been shown [{\textit  {Lewin}}, 1976; {\textit  {Carson}}, 1986; {\textit  {Bartholdy and Billi}}, 2002; {\textit  {Rinaldi}}, 2003].
%...has been shown [e.g., {\textit  {Lewin}}, 1976; {\textit  {Carson}}, 1986; {\textit  {Bartholdy and Billi}}, 2002; {\textit  {Rinaldi}}, 2003].

%...as shown by \citet{jskilby}.
%...as shown by \citet{lewin76}, \citet{carson86}, \citet{bartoldy02}, and \citet{rinaldi03}.
%...has been shown \citep{jskilbye}.
%...has been shown \citep{lewin76,carson86,bartoldy02,rinaldi03}.
%...has been shown \citep [e.g.,][]{lewin76,carson86,bartoldy02,rinaldi03}.
%
% Please use ONLY \citet and \citep for reference citations.
% DO NOT use other cite commands (e.g., \cite, \citeyear, \nocite, \citealp, etc.).

%% ------------------------------------------------------------------------ %%
%
%  END ARTICLE
%
%% ------------------------------------------------------------------------ %%
\end{article}
\newpage
%
%
%% Enter Figures and Tables here:
%
\begin{table}
\caption{Overview of assimilation experiments performed.}
\centering
\begin{tabular}{p{2cm}p{2cm}p{6cm}p{4cm}}
	Experiment& Model &  Assimilated Quantities  & Run dates \\
\hline
E1 & CAM	&	none &  1 Jan - 28 Feb 2009 \\
E2 & CAM &	$\chi_1$, $\chi_2$, $\Delta$LOD	& 1 Jan - 28 Feb 2009 \\
E3 & CAM &	Radiosonde temperatures	& 1 Jan -31 Jan 2009	\\
E4 & CAM &	Radiosonde temperatures, $\chi_1$, $\chi_2$, $\Delta$LOD	& 1 Jan - 17 Jan 2009\\
E5 & WACCM &	none   & 1-30 Oct 2009	\\
E6 & WACCM &	GPS-RO + NNRA$^b$ tropics only & 1-30 Oct 2009	\\
E7 & WACCM &	GPS-RO + NNRA$^b$ whole atmosphere  & 1-30 Oct 2009	\\
E8$^a$ & WACCM &	GPS-RO + NNRA$^b$ whole atmosphere +SABER & 1 Jan - 28 Feb 2009\\	
\hline
\end{tabular}
\tablenotetext{a}{Experiments performed by \citet{Pedatella2014}}
\tablenotetext{b}{Observations used in the NCEP/NCAR Reanalysis project \citet{Saha2010}, which include radiosonde and aircraft winds and temperatures, plus satellite drift winds.}
\label{tab:expts}
\end{table}
\clearpage

%-------comparison of the WACCM experiments in terms of true error ----
 \begin{figure}
	 \includegraphics[width=\textwidth]{Paper_figures/ERPDA_paper_evalvariable_state_space.pdf}
	 \caption{Global-average ensemble variance as a function of time, for experiments E5-E8, which all assimilated real observations into WACCM using DART.}
	 \label{fig:evalvariable_state}
\end{figure}

%-------comparison of the WACCM experiments in terms of aam ----
\begin{figure}
	 \includegraphics[width=\textwidth]{Paper_figures/ERPDA_paper_evalvariable_aam_space.pdf}
	 \caption{Comparison of the ensemble (gray) in experiments E5-E8, in terms of their axial angular momentum excitation functions $\chi_3$, scaled to equivalent length-of-day anomalies using (\ref{eq:X3_to_LOD}), and compared to the observed anomalous length-of-day for each case.}
	 \label{fig:evalvariable_aam}
\end{figure}

%------comparison of the DA runs in the observation space
\begin{figure}[p]
\includegraphics[width=\textwidth]{Paper_figures/ERPDA_paper_erpda_obs_space.pdf} 
 \caption{ The DART prior ensemble (gray) compared to the true state (green) in terms of assimilation time and each of the three angular momentum functions [(\ref{eq:X1})-(\ref{eq:X3}), (\ref{eq:X3_to_LOD})].  The columns compare the four experiments summarized in Table \ref{tab:expts}.  }
 \label{fig:fit_to_ERPs}
\end{figure}

%-----evolution of the covariances between local variables and the AAM observations 
 \begin{figure}
	 \includegraphics[width=\textwidth]{Paper_figures/ERPDA_paper_U_to_LOD_covariances_.pdf}
	 \caption{Evolution of the covariance zonal wind and the axial angular momentum component ($\chi_3$) on three dates in E2 (assimilating AAM). The top row shows values averaged over stratospheric levels, while the bottom row shows values averaged over tropospheric levels.}
 \label{fig:covariances}
\end{figure}

%-----evolution of the prior error and corresponding increments 
 \begin{figure}
	 \includegraphics[width=\textwidth]{Paper_figures/ERPDA_paper_U_priorerror_vs_increment_vs_ER_31jan.pdf}
	 \caption{Snapshots of the (a) bias (truth minus prior ensemble mean), (b) analysis increment (posterior minus prior ensemble mean), (c) posterior minus prior mean square error, and (d) posterior-minus prior ensemble variance on 31 January, i.e. after 1 month of assimilation, on the 320 hPa vertical leve, and all for assimilation in experiment E2 (assimilating AAM only). } 
 \label{fig:error_increments}
\end{figure}


%-----focus on the ensemble in two regions to show how it is moved away from the true state
 \begin{figure}
	 \includegraphics[width=\textwidth]{Paper_figures/ERPDA_paper_point_checks.pdf}
	 \caption{Comparison of the ensemble (gray) and its mean (black) to the true state (pink), for no assimilation [E1, (a) and (c)], and with assimilation of the three angular momentum components [E2, (b) and (d)]. The top row shows zonal wind averaged over the Atlantic jet stream, and the bottom row shows zonal wind averaged in the polar vortex (see text).}
	 \label{fig:point_checks}
\end{figure}




%------MSE increments in time (RST means error reduction) comparing RST and ERPSRT
 \begin{figure}
	 \includegraphics[width=\textwidth]{Paper_figures/ERPDA_paper_MSEincrement_comparison.pdf}
	 \caption{Vertical profiles of the difference in MSE between the posterior and prior states (where read means that assimilation reduces error) as a function of time, (a) assimilating a global grid of temperatures, and (b) additionally assimilating the three global angular momentum components.}
	 \label{fig:added_value_MSEincrement}
\end{figure}


%------MSE diff between ERPRST and RST (red is good)
 \begin{figure}
	 \includegraphics[width=\textwidth]{Paper_figures/ERPDA_paper_MSE_RST_vs_ERPRST.pdf}
	 \caption{Vertical profiles of the differenence in MSE between E4 (assimilating temperatures and global angular momentum) and E3 (assimilating temperatures only). Blue indicates that the error is higher when angular momentum observations are added.}
	 \label{fig:added_value_MSE}
\end{figure}



% DO NOT USE \psfrag or \subfigure commands.
%
% Figure captions go below the figure.
% Table titles go above tables; all other caption information
%  should be placed in footnotes below the table.
%
%----------------
% EXAMPLE FIGURE
%
% \begin{figure}
% \noindent\includegraphics[width=20pc]{samplefigure.eps}
% \caption{Caption text here}
% \label{figure_label}
% \end{figure}
%
% ---------------
% EXAMPLE TABLE
%
%\begin{table}
%\caption{Time of the Transition Between Phase 1 and Phase 2\tablenotemark{a}}
%\centering
%\begin{tabular}{l c}
%\hline
% Run  & Time (min)  \\
%\hline
%  $l1$  & 260   \\
%  $l2$  & 300   \\
%  $l3$  & 340   \\
%  $h1$  & 270   \\
%  $h2$  & 250   \\
%  $h3$  & 380   \\
%  $r1$  & 370   \\
%  $r2$  & 390   \\
%\hline
%\end{tabular}
%\tablenotetext{a}{Footnote text here.}
%\end{table}

% See below for how to make sideways figures or tables.

\end{document}

%%%%%%%%%%%%%%%%%%%%%%%%%%%%%%%%%%%%%%%%%%%%%%%%%%%%%%%%%%%%%%%

More Information and Advice:

%% ------------------------------------------------------------------------ %%
%
%  SECTION HEADS
%
%% ------------------------------------------------------------------------ %%

% Capitalize the first letter of each word (except for
% prepositions, conjunctions, and articles that are
% three or fewer letters).

% AGU follows standard outline style; therefore, there cannot be a section 1 without
% a section 2, or a section 2.3.1 without a section 2.3.2.
% Please make sure your section numbers are balanced.
% ---------------
% Level 1 head
%
% Use the \section{} command to identify level 1 heads;
% type the appropriate head wording between the curly
% brackets, as shown below.
%
%An example:
%\section{Level 1 Head: Introduction}
%
% ---------------
% Level 2 head
%
% Use the \subsection{} command to identify level 2 heads.
%An example:
%\subsection{Level 2 Head}
%
% ---------------
% Level 3 head
%
% Use the \subsubsection{} command to identify level 3 heads
%An example:
%\subsubsection{Level 3 Head}
%
%---------------
% Level 4 head
%
% Use the \subsubsubsection{} command to identify level 3 heads
% An example:
%\subsubsubsection{Level 4 Head} An example.
%
%% ------------------------------------------------------------------------ %%
%
%  IN-TEXT LISTS
%
%% ------------------------------------------------------------------------ %%
%
% Do not use bulleted lists; enumerated lists are okay.
% \begin{enumerate}
% \item
% \item
% \item
% \end{enumerate}
%
%% ------------------------------------------------------------------------ %%
%
%  EQUATIONS
%
%% ------------------------------------------------------------------------ %%

% Single-line equations are centered.
% Equation arrays will appear left-aligned.

Math coded inside display math mode \[ ...\]
 will not be numbered, e.g.,:
 \[ x^2=y^2 + z^2\]

 Math coded inside \begin{equation} and \end{equation} will
 be automatically numbered, e.g.,:
 \begin{equation}
 x^2=y^2 + z^2
 \end{equation}

% IF YOU HAVE MULTI-LINE EQUATIONS, PLEASE
% BREAK THE EQUATIONS INTO TWO OR MORE LINES
% OF SINGLE COLUMN WIDTH (20 pc, 8.3 cm)
% using double backslashes (\\).

% To create multiline equations, use the
% \begin{eqnarray} and \end{eqnarray} environment
% as demonstrated below.
\begin{eqnarray}
  x_{1} & = & (x - x_{0}) \cos \Theta \nonumber \\
        && + (y - y_{0}) \sin \Theta  \nonumber \\
  y_{1} & = & -(x - x_{0}) \sin \Theta \nonumber \\
        && + (y - y_{0}) \cos \Theta.
\end{eqnarray}

%If you don't want an equation number, use the star form:
%\begin{eqnarray*}...\end{eqnarray*}

% Break each line at a sign of operation
% (+, -, etc.) if possible, with the sign of operation
% on the new line.

% Indent second and subsequent lines to align with
% the first character following the equal sign on the
% first line.

% Use an \hspace{} command to insert horizontal space
% into your equation if necessary. Place an appropriate
% unit of measure between the curly braces, e.g.
% \hspace{1in}; you may have to experiment to achieve
% the correct amount of space.


%% ------------------------------------------------------------------------ %%
%
%  EQUATION NUMBERING: COUNTER
%
%% ------------------------------------------------------------------------ %%

% You may change equation numbering by resetting
% the equation counter or by explicitly numbering
% an equation.

% To explicitly number an equation, type \eqnum{}
% (with the desired number between the brackets)
% after the \begin{equation} or \begin{eqnarray}
% command.  The \eqnum{} command will affect only
% the equation it appears with; LaTeX will number
% any equations appearing later in the manuscript
% according to the equation counter.
%

% If you have a multiline equation that needs only
% one equation number, use a \nonumber command in
% front of the double backslashes (\\) as shown in
% the multiline equation above.

%% ------------------------------------------------------------------------ %%
%
%  SIDEWAYS FIGURE AND TABLE EXAMPLES
%
%% ------------------------------------------------------------------------ %%
%
% For tables and figures, add \usepackage{rotating} to the paper and add the rotating.sty file to the folder.
% AGU prefers the use of {sidewaystable} over {landscapetable} as it causes fewer problems.
%
% \begin{sidewaysfigure}
% \includegraphics[width=20pc]{samplefigure.eps}
% \caption{caption here}
% \label{label_here}
% \end{sidewaysfigure}
%
%
%
% \begin{sidewaystable}
% \caption{}
% \begin{tabular}
% Table layout here.
% \end{tabular}
% \end{sidewaystable}
%
%

