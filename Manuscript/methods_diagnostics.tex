Our main evaluation diagnostics are the square error between the simulated true state and the ensemble mean
\begin{eqnarray}
E_M = \left(
\left< x_i \right>-x_{i}^{t}
\right)^2
\end{eqnarray}
and the ensemble variance about its mean:
\begin{eqnarray}
S = 
\frac{1}{N-1}
\sum_{n=1}^N
\left(
	\left< x_{i}^{n} \right>-x_i^t
\right)^2,
\label{eq:spread}
\end{eqnarray}
where $x_{i}$ represents a component of the state vector in the ensemble,  $x_{i}^t$ and $x_{i}^n$ the corresponding component in the truth and individual ensemble members, respectively.

To evaluate whether the ensemble accurately represents the true uncertainty, we  test whether the truth can be considered a sample of the probability distribution represented by the ensemble.
 \citet{Huntley2009} and \citet{Murphy1988} pointed out that this is the case when 
\begin{eqnarray}
\llbracket E_M \rrbracket = \frac{N+1}{N} \llbracket S \rrbracket, 
\label{eq:EvsS}
\end{eqnarray}
where the square brackets represent spatial averages. 
 
%---cut---the rh doesn't really show any new information 
%A second way to quantify whether the ensemble and the truth come from the same probability distribution is via a rank histogram \citep[and references therein]{Hamill2001}.
%A rank histogram is generated by ordering the values of the ensemble at each point in the state space, and then finding the rank of some verification value on this list.
%In our case, we use the true state as the verification, since it is known,
%and then count this rank up all state space points in a histogram. 
%If the truth comes from the same distribution as the ensemble, the rank histogram should be nearly flat.  
%If the truth is a frequent outlier of this distribution, the rank histogram will by concave.
%A convex rank histogram indicates an ensemble whose spread is so large that the truth is usually the central rank. 

