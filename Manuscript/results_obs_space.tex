Figure \ref{fig:fit_to_ERPs} compares the angular momentum components $\chi_2$ and $\chi_3$ for the three DART-CAM ensembles, their means, and the synthetic observations.
%----cut--- Since the synthetic observations are of the angular momentum components themselves (rather than Earth rotation parameters), there is no effect of oceanic angular momentum in these experiments, and no unknown offset between modeled and observed $\chi_3$. 
%---cut----With no oceanic effects 
The ensemble fits to $\chi_1$ look qualitatively similar to $\chi_2$ for all experiments and have been omitted for simplicity. 

With no assimilation (``No DA'', first column), the angular momentum functions again show how the ensemble spreads about the truth, and how the spread saturates after about one month.
If we assimilate the angular momentum functions alone (``AAM DA'', second column), the ensemble predictably clusters close to the true angular momentum, and captures the day-to-day angular momentum variations. 
The difference between the first two columns indicates that assimilating the angular momentum has imposed some constraint upon the wind, temperature, and surface pressure fields, though, as will be shown below, this does not necessarily mean that those fields have also moved closer to the true state. 

The ensemble clusters even more tightly around the truth when instead of the angular momentum functions we assimilate local temperature observations (``Temp DA'', third column), which is a much stronger constraint on the model fields. 
Finally, adding the angular momentum observations to the regularly-spaced temperature observations (``Temp + AAM DA'', fourth column) slightly increases the agreement between the ensemble and the true state further.  
This suggests that the angular momentum observations may contain additional information that complements the information in the temperature observations, but it does not tell us whether the actual model fields have been improved, and if so, how. 
The next section will therefore examine how the update seen in the observation space translates into an increment in the state space. 
