%%%%%%%%%%%%%%%%%%%%%%%%%%%%%%%%%%%%%%%%%%%%%%%%%%%%%%%%%%%%%%%%%%%%%%%%%%%%
% AGUtmpl.tex: this template file is for articles formatted with LaTeX2e,
% Modified March 2013
%
% This template includes commands and instructions
% given in the order necessary to produce a final output that will
% satisfy AGU requirements.
%
% PLEASE DO NOT USE YOUR OWN MACROS
% DO NOT USE \newcommand, \renewcommand, or \def.
%
% FOR FIGURES, DO NOT USE \psfrag or \subfigure.
%
%%%%%%%%%%%%%%%%%%%%%%%%%%%%%%%%%%%%%%%%%%%%%%%%%%%%%%%%%%%%%%%%%%%%%%%%%%%%
%
% All questions should be e-mailed to latex@agu.org.
%
%%%%%%%%%%%%%%%%%%%%%%%%%%%%%%%%%%%%%%%%%%%%%%%%%%%%%%%%%%%%%%%%%%%%%%%%%%%%
%
% Step 1: Set the \documentclass
%
% There are two options for article format: two column (default)
% and draft.
%
% PLEASE USE THE DRAFT OPTION TO SUBMIT YOUR PAPERS.
% The draft option produces double spaced output.
%
% Choose the journal abbreviation for the journal you are
% submitting to:

% jgrga JOURNAL OF GEOPHYSICAL RESEARCH
% gbc   GLOBAL BIOCHEMICAL CYCLES
% grl   GEOPHYSICAL RESEARCH LETTERS
% pal   PALEOCEANOGRAPHY
% ras   RADIO SCIENCE
% rog   REVIEWS OF GEOPHYSICS
% tec   TECTONICS
% wrr   WATER RESOURCES RESEARCH
% gc    GEOCHEMISTRY, GEOPHYSICS, GEOSYSTEMS
% sw    SPACE WEATHER
% ms    JAMES
%
%
%
% (If you are submitting to a journal other than jgrga,
% substitute the initials of the journal for "jgrga" below.)

%\documentclass[draft,jgrga]{agutex}
\documentclass[jgrga]{agutex}

% To create numbered lines:

% If you don't already have lineno.sty, you can download it from
% http://www.ctan.org/tex-archive/macros/latex/contrib/ednotes/
% (or search the internet for lineno.sty ctan), available at TeX Archive Network (CTAN).
% Take care that you always use the latest version.

% To activate the commands, uncomment \usepackage{lineno}
% and \linenumbers*[1]command, below:

 \usepackage{lineno}
% \linenumbers*[1]

%  To add line numbers to lines with equations:
%  \begin{linenomath*}
%  \begin{equation}
%  \end{equation}
%  \end{linenomath*}
%%%%%%%%%%%%%%%%%%%%%%%%%%%%%%%%%%%%%%%%%%%%%%%%%%%%%%%%%%%%%%%%%%%%%%%%%
% Figures and Tables
%
%
% DO NOT USE \psfrag or \subfigure commands.
%
%  Figures and tables should be placed AT THE END OF THE ARTICLE,
%  after the references.
%
%  Uncomment the following command to include .eps files
%  (comment out this line for draft format):
  \usepackage[dvips]{graphicx}
%
%  Uncomment the following command to allow illustrations to print
%   when using Draft:
  \setkeys{Gin}{draft=false}
%
% Substitute one of the following for [dvips] above
% if you are using a different driver program and want to
% proof your illustrations on your machine:
%
% [xdvi], [dvipdf], [dvipsone], [dviwindo], [emtex], [dviwin],
% [pctexps],  [pctexwin],  [pctexhp],  [pctex32], [truetex], [tcidvi],
% [oztex], [textures]
%
% See how to enter figures and tables at the end of the article, after
% references.
%
%  Other packages:
\usepackage{amsmath}
\usepackage[usenames]{color}
\usepackage{ulem}
\usepackage{url}
\usepackage{lscape}
%\usepackage[pdftex]{graphicx}
%\usepackage{epstopdf}
%\usepackage{pgfplots}
%
%
%% ------------------------------------------------------------------------ %%
%
%  ENTER PREAMBLE
%
%% ------------------------------------------------------------------------ %%

% Author names in capital letters:
\authorrunninghead{NEEF AND MATTHES}

% Shorter version of title entered in capital letters:
\titlerunninghead{ASSIMILATION OF INTEGRALS}

%Corresponding author mailing address and e-mail address:
\authoraddr{Corresponding author: L.J. Neef,
%Helmholtz -Zentrum f{\"u}r Ozeanforschung (GEOMAR),
D{\"u}sternbrooker Weg 20,
D-24105 Kiel, Germany
(neef@geomar.de)}

\begin{document}
\bibliographystyle{agu08}

%% ------------------------------------------------------------------------ %%
%
%  TITLE
%
%% ------------------------------------------------------------------------ %%


\title{Assimilation of Earth Rotation Parameter observations to constrain and evaluate atmospheric models}
%
% e.g., \title{Terrestrial ring current:
% Origin, formation, and decay $\alpha\beta\Gamma\Delta$}
%

%% ------------------------------------------------------------------------ %%
%
%  AUTHORS AND AFFILIATIONS
%
%% ------------------------------------------------------------------------ %%


%Use \author{\altaffilmark{}} and \altaffiltext{}

% \altaffilmark will produce footnote;
% matching \altaffiltext will appear at bottom of page.

 \authors{Lisa Neef\altaffilmark{1} and
Katja Matthes \altaffilmark{1}}

\altaffiltext{1}{Ocean Circulation and Climate Dynamics - Marine Meteorology,
   Helmholtz Centre for Ocean Research Kiel (GEOMAR), Kiel, Germany.}


%% ------------------------------------------------------------------------ %%
%
%  ABSTRACT
%
%% ------------------------------------------------------------------------ %%

% >> Do NOT include any \begin...\end commands withinA
% >> the body of the abstract.



%% ------------------------------------------------------------------------ %%
%
%  BEGIN ARTICLE
%
%% ------------------------------------------------------------------------ %%

% The body of the article must start with a \begin{article} command
%
% \end{article} must follow the references section, before the figures
%  and tables.

\begin{article}

%% ------------------------------------------------------------------------ %%
%
%  MATH ABBREVIATSIONS
%
%% ------------------------------------------------------------------------ %%
\newcommand{\degree}{\ensuremath{^\circ}}


%% ------------------------------------------------------------------------ %%
%
%  TEXT
%
%% ------------------------------------------------------------------------ %%

\end{article}
\newpage
%
%
%% Enter Figures and Tables here:
%

\begin{table}
\caption{Overview of assimilation experiments performed.}
\centering
\begin{tabular}{ll}
Experiment name &  Assimilated Quantities  \\
\hline
NODA  &  none					\\
ERP  &  $p_1$, $p_2$, $\Delta$LOD	\\
RST  &  Radiosonde temperatures		\\
RST+ERP	 &  Radiosonde temperatures, $p_1$, $p_2$, $\Delta$LOD \\
\hline
\end{tabular}
%\tablenotetext{a}{Footnote text here.}
\label{tab:expts}
\end{table}

 \begin{figure}
\includegraphics[width=\textwidth]{../../Plots/ERP_DA/radiosonde_locations.eps} 
 \caption{  }
 \label{fig:RS}
\end{figure}

 \begin{figure}
\includegraphics[width=0.4\textwidth]{../../Plots/ERP_DA/NoDA_error_growthUlev.eps} 
%\includegraphics[width=0.4\textwidth]{../../Plots/ERP_DA/NoDA_error_growthPS.eps} 
 \caption{(a) Evolution of the RMS of the error (ensemble mean minus forecast) growth in zonal-mean zonal wind at 300 hPa.  (b) Evolution of the ensemble spread in the same field.  \textcolor{red}{To Do: remove the global error panel (it's redundant), make the axes nicer.}  }
 \label{fig:NODA}
\end{figure}


 \begin{figure}
\includegraphics[width=\textwidth]{../../Plots/ERP_DA/compare_ERPscompare_constraintsERP_LOD.eps} 
 \caption{ The DART ensemble (gray) compared to the true state (blue) and its observations (red) in terms of the length-of-day, one of the three ERPs.  Each panel shows one of the four assimilation experiments summarized in Table \ref{tab:expts}.  In each plot, the ensemble mean is shown as a thick gray line. \textcolor{red}{Take out the Radiosondes experiment and instead put in the ERP-only experiment here. Also add letter labels to the titles.  After these fixes decide which ERP is the most useful to show.} }
 \label{fig:fit_to_LOD}
\end{figure}

 \begin{figure}
\includegraphics[width=\textwidth]{../../Plots/ERP_DA/compare_ERPscompare_constraintsERP_PM1.eps} 
 \label{fig:fit_to_PM1}
\end{figure}

 \begin{figure}
\includegraphics[width=\textwidth]{../../Plots/ERP_DA/compare_ERPscompare_constraintsERP_PM2.eps} 
 \label{fig:fit_to_PM2}
\end{figure}




%---cut because this information is redundant with what is shown in the first figure
% \begin{figure}
%\includegraphics[width=0.9\textwidth]{../../Plots/ERP_DA/compare_ERPsERPDA_vs_NODA.eps} 
% \caption{Ensemble fit to the ERPs with and without assimilation -- we see that steps are taken in the assimilaton to rectify %prior disagreement with the ERPs.}
% \label{fig:fit_to_erps}
%\end{figure}


 \begin{figure}
\includegraphics[width=0.9\textwidth]{../../Plots/ERP_DA/error_red_in_space_and_time.eps} \\
 \caption{Error reduction in the surface pressure and zonal wind fields, along with the increment, as the ERP assimilation progresses in time.  We see that both ensemble spread and error are reduced somewhat, but mostly at the model top.}
 \label{fig:ERPDA}
\end{figure}

 \begin{figure}
\includegraphics[width=0.9\textwidth]{../../Plots/ERP_DA/global_ERPDA_rmse.eps} \\
 \caption{Global RMSE in the zonal wind at 10hPa and 300hPa (a), as well as the surface pressure (b), comparing ensembles with and without assimilation of ERPs.}
 \label{fig:global1}
\end{figure}



 \begin{figure}
\includegraphics[width=0.9\textwidth]{../../Plots/ERP_DA/increments_comparison.eps} \\
 \caption{Innovations in the zonal wind and surface pressure fields.  We see that it takes a while to rev up, and then most of the innovations are concentrated up high (why?)}
 \label{fig:innovations}
\end{figure}

% \begin{figure}
%\includegraphics[width=0.9\textwidth]{../../Plots/ERP_DA/ } \\
% \caption{Comparison of the ensemble in  in terms of NAO index, comparing NODA, ERPDA, RSDA, and ERPRSDA.  Here is the take-home message: ERPs have potential to be a constraint, but they don't add anything to conventional observations.}
% \label{fig:NAOi}
%\end{figure}


 \begin{figure}
\includegraphics[width=0.9\textwidth]{../../Plots/ERP_DA/ERP_complimentarity3.eps} \\
 \caption{(a) RMSE in zonal-mean zonal wind as a function of vertical level and time, in an experiment assimilating local temperatures along with global ERPs, relative to the temp-only experiment.  (b) The difference in the ensemble spread in the zonal mean zonal wind.\textcolor{red}{To Do: remove the left column of plots.  Also unify the color axis.}}
 \label{fig:added_value}
\end{figure}




% DO NOT USE \psfrag or \subfigure commands.
%
% Figure captions go below the figure.
% Table titles go above tables; all other caption information
%  should be placed in footnotes below the table.
%
%----------------
% EXAMPLE FIGURE
%
% \begin{figure}
% \noindent\includegraphics[width=20pc]{samplefigure.eps}
% \caption{Caption text here}
% \label{figure_label}
% \end{figure}
%
% ---------------
% EXAMPLE TABLE
%
%\begin{table}
%\caption{Time of the Transition Between Phase 1 and Phase 2\tablenotemark{a}}
%\centering
%\begin{tabular}{l c}
%\hline
% Run  & Time (min)  \\
%\hline
%  $l1$  & 260   \\
%  $l2$  & 300   \\
%  $l3$  & 340   \\
%  $h1$  & 270   \\
%  $h2$  & 250   \\
%  $h3$  & 380   \\
%  $r1$  & 370   \\
%  $r2$  & 390   \\
%\hline
%\end{tabular}
%\tablenotetext{a}{Footnote text here.}
%\end{table}

% See below for how to make sideways figures or tables.

\end{document}

%%%%%%%%%%%%%%%%%%%%%%%%%%%%%%%%%%%%%%%%%%%%%%%%%%%%%%%%%%%%%%%

More Information and Advice:

%% ------------------------------------------------------------------------ %%
%
%  SECTION HEADS
%
%% ------------------------------------------------------------------------ %%

% Capitalize the first letter of each word (except for
% prepositions, conjunctions, and articles that are
% three or fewer letters).

% AGU follows standard outline style; therefore, there cannot be a section 1 without
% a section 2, or a section 2.3.1 without a section 2.3.2.
% Please make sure your section numbers are balanced.
% ---------------
% Level 1 head
%
% Use the \section{} command to identify level 1 heads;
% type the appropriate head wording between the curly
% brackets, as shown below.
%
%An example:
%\section{Level 1 Head: Introduction}
%
% ---------------
% Level 2 head
%
% Use the \subsection{} command to identify level 2 heads.
%An example:
%\subsection{Level 2 Head}
%
% ---------------
% Level 3 head
%
% Use the \subsubsection{} command to identify level 3 heads
%An example:
%\subsubsection{Level 3 Head}
%
%---------------
% Level 4 head
%
% Use the \subsubsubsection{} command to identify level 3 heads
% An example:
%\subsubsubsection{Level 4 Head} An example.
%
%% ------------------------------------------------------------------------ %%
%
%  IN-TEXT LISTS
%
%% ------------------------------------------------------------------------ %%
%
% Do not use bulleted lists; enumerated lists are okay.
% \begin{enumerate}
% \item
% \item
% \item
% \end{enumerate}
%
%% ------------------------------------------------------------------------ %%
%
%  EQUATIONS
%
%% ------------------------------------------------------------------------ %%

% Single-line equations are centered.
% Equation arrays will appear left-aligned.

Math coded inside display math mode \[ ...\]
 will not be numbered, e.g.,:
 \[ x^2=y^2 + z^2\]

 Math coded inside \begin{equation} and \end{equation} will
 be automatically numbered, e.g.,:
 \begin{equation}
 x^2=y^2 + z^2
 \end{equation}

% IF YOU HAVE MULTI-LINE EQUATIONS, PLEASE
% BREAK THE EQUATIONS INTO TWO OR MORE LINES
% OF SINGLE COLUMN WIDTH (20 pc, 8.3 cm)
% using double backslashes (\\).

% To create multiline equations, use the
% \begin{eqnarray} and \end{eqnarray} environment
% as demonstrated below.
\begin{eqnarray}
  x_{1} & = & (x - x_{0}) \cos \Theta \nonumber \\
        && + (y - y_{0}) \sin \Theta  \nonumber \\
  y_{1} & = & -(x - x_{0}) \sin \Theta \nonumber \\
        && + (y - y_{0}) \cos \Theta.
\end{eqnarray}

%If you don't want an equation number, use the star form:
%\begin{eqnarray*}...\end{eqnarray*}

% Break each line at a sign of operation
% (+, -, etc.) if possible, with the sign of operation
% on the new line.

% Indent second and subsequent lines to align with
% the first character following the equal sign on the
% first line.

% Use an \hspace{} command to insert horizontal space
% into your equation if necessary. Place an appropriate
% unit of measure between the curly braces, e.g.
% \hspace{1in}; you may have to experiment to achieve
% the correct amount of space.


%% ------------------------------------------------------------------------ %%
%
%  EQUATION NUMBERING: COUNTER
%
%% ------------------------------------------------------------------------ %%

% You may change equation numbering by resetting
% the equation counter or by explicitly numbering
% an equation.

% To explicitly number an equation, type \eqnum{}
% (with the desired number between the brackets)
% after the \begin{equation} or \begin{eqnarray}
% command.  The \eqnum{} command will affect only
% the equation it appears with; LaTeX will number
% any equations appearing later in the manuscript
% according to the equation counter.
%

% If you have a multiline equation that needs only
% one equation number, use a \nonumber command in
% front of the double backslashes (\\) as shown in
% the multiline equation above.

%% ------------------------------------------------------------------------ %%
%
%  SIDEWAYS FIGURE AND TABLE EXAMPLES
%
%% ------------------------------------------------------------------------ %%
%
% For tables and figures, add \usepackage{rotating} to the paper and add the rotating.sty file to the folder.
% AGU prefers the use of {sidewaystable} over {landscapetable} as it causes fewer problems.
%
% \begin{sidewaysfigure}
% \includegraphics[width=20pc]{samplefigure.eps}
% \caption{caption here}
% \label{label_here}
% \end{sidewaysfigure}
%
%
%
% \begin{sidewaystable}
% \caption{}
% \begin{tabular}
% Table layout here.
% \end{tabular}
% \end{sidewaystable}
%
%

