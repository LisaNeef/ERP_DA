%
%  $Id: cam_se-fv.tex 6635 2013-12-04 23:16:42Z thoar $
%

Unconventional observations of the earth system can
be used in data assimilation research.
Changes in atmospheric angular momentum are reflected in observations
of earth rotation parameters (ERP),
which are highly accurate, but are an integration over the whole earth system.
The vector components of angular momentum of the CAM atmosphere 
can be defined as state variables,
which can be compared against observations of angular momentum.

The first test is a series of ''perfect model'' assimilations,
in which observations are taken from a free CAM run,
rather than from nature.
So the truth is known exactly, 
and can be matched by each member of the ensemble of CAM runs.
%Here no observational error is added to each observation to make them realistic.
these assimilations use combinations of (synthetic) observations 
of ERPs and radiosonde temperature and winds
at the locations of actual radiosondes from Jan 1-16, 2009.

% ERP_PM2_NoDA.eps                ERP_PM2_Radiosondes.eps
% ERP_PM2_RadiosondesTempOnly.eps ERP_PM2_RadiosondeTempERPs.eps  

\begin{figure}[h]
\begin{center}
\includegraphics[height=0.90\columnwidth]{./WACCM_ERP_pmo.pdf}
\caption{
Equatorial angular momentum component along the vector 
from the center of the earth to ($0^\circ$N,$0^\circ$E)
as a function of time for an ensemble of 80 CAM model states (light gray curves).
The units are given in terms of equivalent polar motion angle in milliseconds of arc (''mas'').
The observational error on the ERP observations (red circles) is 0 mas, 
so they lie on the true state (blue line).
%which comes from a ''perfect model'' free run of CAM.
The ensemble mean is shown by the dart gray line
These are ''prior'' values, which are derived from the model states 
at the end of each 24 hour forecast, but before the assimilation.  
a) Ensemble is not constrained by observations.
b) Assimilation of all temperature and wind observations, but no ERP.
c) Assimilation of only T
d) Assimilation of T and ERP, but no winds.
}
\end{center}
\label{fig:ang_mom}
\end{figure}

The initial ensemble starts with climatological spread (in $\chi_2$)
and in Fig.~\ref{fig:ang_mom}a it grows steadily because no observations are assimilated.
The ensemble mean happens to stay close to the truth,
but the confidence in this mean is low.
In Fig.~\ref{fig:ang_mom}b thousands of radiosonde observations constrain the ensemble 
to lie much closer to the truth,
even for this quantity which is not assimilated.
Fig.~\ref{fig:ang_mom}c shows that removing the wind observations from the assimilation
degrades the ensemble representation of the state.
Fig.~\ref{fig:ang_mom}d shows that introducing ERP observations to Fig. c restores
some of the accuracy (shown by the mean) and confidence (shown by the spread)
of the ensemble.

We can conclude that in the perfect model context
several ERP observations can provide 
%a large fraction of the
information about the true state of the atmosphere,
that is complementary to the wind and temperature observations. 
%as can be provided by thousands of wind observations.
% <lisa:>  I changed this text around a bit because the ERP observations don't really make up entirely for the missing wind observations -- they decrease the overall error in the state somewhat, but because the ERPs represent integrals, they wont constrain the wind/temp fields like the radiosonde obs will.  


Further development of this work is in preparation for publication: lneef@geomar.de.
