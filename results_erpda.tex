When an angular momentum observation is assimilated, the update on the state variables (\ref{eq:state_update}) is proportional to the covariance between each state component (i.e. the wind, temperature, or surface pressure at some point) and the global angular momentum component in question ($\chi_1$,$\chi_2$, or $\chi_3$). 
The filter computes these covariances from the ensemble using (\ref{eq:covariance}).

%--------covariance analysis---------------------

To see what the ensemble-estimated covariance $c_{x_i y_j}$ looks like when $y_j$ is one of the angular momentum components, we can abbreviate the angular momentum integrals [(\ref{eq:I1})-(\ref{eq:h3})] as
%
\begin{eqnarray}
y_j = \sum_{i=1}^M f_{i,j} x_i,
\end{eqnarray}
%
where $f_{i,j}$ represents the spatial weighting that is applied to state component $i$ for angular momentum component $j$, times the latitude, longitude, and mass increments in the integrals, and $M$ represents the number of variables in the model state.
Then the observation error terms in (\ref{eq:covariance}) become
\begin{eqnarray}
	e_{y_j}^n = y_j^n - \left< y_j^n \right>
	= \sum_{k=1}^{M}f_{k,j}e_{x_k}^n,
\end{eqnarray}
which means that the state-to-observation covariance term becomes
\begin{eqnarray}
	c_{x_i y_j} = \frac{1}{N-1}
	\sum_{n=1}^{N}
	\left[
	e_{x_i}^n
	\sum_{k=1}^{M} f_{k,j}e_{x_k}^n
	\right].
	\label{eq:state_to_obs_covariance}
\end{eqnarray}

%--------covariance analysis---------------------
For a given point $x_i$ in the model state, the covariance to angular momentum component $y_j$ is proportional to the ensemble spread at that point ($\frac{1}{N-1}\left[\sum_{n=1}^N \left( e^n_{x_i} \right)^2 \right]$) times the weighting function at that point ($f_{i,j}$), plus the sum of the covariances between $x_i$ and all other other points in the model state [$\frac{1}{N-1}\sum_{n=1}^N e_{x_i}^n e_{x_k}^n$], each multiplied by their respective weighting functions ($f_{k,j}$). 
In other words, a point will be updated by an angular momentum observation if it is located in a region that is weighted strongly for that particular angular momentum component, and/or if it happens to covary strongly with other points that are strongly weighted in the integrals.  

% covariance figure-----------------------------------
Figure \ref{fig:covariances} shows the covariance between the zonal wind field and the axial angular momentum ($\chi_3$) over the Northern Hemisphere for three days in Experiment 2, roughly dividing between the 
stratosphere (100-24 hPa) 
and
troposphere (surface to 100 hPa).
Note that the covariance pattern does not resemble the weighting function of the axial wind integral (\ref{eq:h3}, which follows the cosine of latitude and is therefore zonally symmetric and maximal in the tropics), and varies strongly in time. 
This implies that dynamic, ensemble-based covariance estimates should be more useful thatn a static estimate based on the weighting functions.

However, the covariances in both regions are on the order of $10^{-11}(\text{ms} \times \text{m/s})$.  
\hl{Considering the variances of the zonal wind (of order **) and the axial angular momentum (of order **), the corresponding state-to-observation correlations turn out to be on the order of 0.001.} 
The miniscule correlations are not surprising, given that each individual state component contributes only a small amount to the global angular momentum integral, but to 
correctly estimate them, the correlations need to be significantly larger than the sampling error that comes from using a finite ensemble, which 
scales as $1/\sqrt{N}$ \citep{Houtekamer1998}. 
This means that we would need upwards of $N=1000$ ensemble members to compute correlations that at least of the same order of magnitude as their sampling error, which is not computationally feasible. 

The only hope for attaining a reasonable correlation estimate from the 80-member model ensemble is if the covariance patterns have a large enough scale.  
Comparing the two rwos of Figure \ref{fig:covariances} shows that, while
the covariance patterns in the stratosphere and troposphere are similar, the covariance field in the stratosphere has larger spatial scales, in keeping with the larger spatial structures of stratospheric flow.
This suggests that the analysis increment due to assimilation of global AAM might be the most accurate in the stratosphere, where is more covariant with its neighbors.  

% increments versus error figure-----------------------------------
The top row of Figure \ref{fig:error_increments} shows the bias between the ensemble mean and the truth in the Northern Hemisphere, at the same time snapshots as the previous figure, and at the 320 hPa level, i.e. is roughly at the height of the tropospheric jets. 
The second row of Fig. \ref{fig:error_increments} shows the analysis increments (i.e. posterior-prior ensemble mean) at the same times and vertical level. 
We can see that, generally, the true error and the analysis increments have many similarities (e.g., on 28 Jan the prior ensemble mean zonal wind is too strong over the USA, and the analysis step increases the wind there), though 
the analysis increments are also an order of magnitude smaller than the true error, which reflects the miniscule state-to-observation correlations and reflects how the information from a single observation has to be spread over the entire state. 
However, it's also possible to find regions where the analysis increment opposes the prior error, e.g. over the northeastern Pacific on 11 Feb. 
Thus while the covariance model developed by the ensemble filter has a physical basis, its high degree of sampling error can easily result in spurious corrections. 

To answer the question of whether assimilating global angular momentum improves the modeled fields or not, 
Figure \ref{fig:ER} shows the difference in the mean square error (again in the zonal wind field), between experiment 1 (no assimilation) and experiment 2 (assimilating the angular momentum), at the same time snapshots as the previous figures. 
(Blue indicates that the assimilation of angular momentum has reduced the overall error, i.e. improved the state estimate.)
Figure \ref{fig:ER} again separates between the stratosphere (100-24 hPa, top row), and troposphere (1000-100 hPa, bottom row). 
Error reduction on the order to 10-20 m/s is evident at both upper and lower levels throughout the assimilation, though increases in error of a similar magnitude also appear early and dominate by February (last column).  
At the same time, the ensemble spread (not shown) consistently decreases to reflect the information ``gained'' from the observations. 
This suggests that we have a divergent data assimilation system, where the ensemble clusters closer together around a mean state that is, generally, farther from the truth than implied by the ensemble. 

% point checks figure-------------------------------------------------------
To illustrate this result on a local level, Figure \ref{fig:point_checks} compares the zonal wind in the ensemble and its mean to the true state, with no assimilation (left) and assimilation of angular momentum (right). 
The ensemble in each cases is shown as averages over two regions: the tropospheric extratropical jet (averaging between 30N-45N, between 300hPa-200hPa, \hl{and all longitudes}), and the Northern Hemisphere stratospheric polar vortex (averaging between 60N-90N, between 30-24 hPa, and all longitudes).
For both measures, the angular momentum assimilation over the first three weeks or so decreases the ensemble's spread about the mean, and gives the ensemble mean a bit more of the day-to-day variations of the true state (which average out in the ensemble with not assimilation), but in general, only the polar vortex winds, which showed a more large-scale covariance structure (Fig. \ref{fig:covariances_UA}) show improvement relative to the truth, and only for the first few weeks of assimilation. 

\hl{[maybe insert a sub-summary here?]}
