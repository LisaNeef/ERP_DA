When an angular momentum observation is assimilated, the update on the state variables (\ref{eq:state_update}) is proportional to the covariance between each state component (i.e. the wind, temperature, or surface pressure at some point) and the global angular momentum component in question ($\chi_1$,$\chi_2$, or $\chi_3$). 
The filter computes these covariances from the ensemble using (\ref{eq:covariance}).

%--------covariance analysis---------------------

To see what the ensemble-estimated covariance $c_{x_i y_j}$ looks like when $y_j$ is one of the angular momentum components, we can abbreviate the angular momentum integrals [(\ref{eq:I1})-(\ref{eq:h3})] as
%
\begin{eqnarray}
y_j = \sum_{i=1}^M f_{i,j} x_i,
\end{eqnarray}
%
where $f_{i,j}$ represents the spatial weighting that is applied to state component $i$ for angular momentum component $j$, times the latitude, longitude, and mass increments in the integrals, and $M$ represents the number of variables in the model state.
Then the observation error terms in (\ref{eq:covariance}) become
\begin{eqnarray}
	e_{y_j}^n = y_j^n - \left< y_j^n \right>
	= \sum_{k=1}^{M}f_{k,j}e_{x_k}^n,
\end{eqnarray}
which means that the state-to-observation covariance term becomes
\begin{eqnarray}
	c_{x_i y_j} = \frac{1}{N-1}
	\sum_{n=1}^{N}
	\left[
	e_{x_i}^n
	\sum_{k=1}^{M} f_{k,j}e_{x_k}^n
	\right].
	\label{eq:state_to_obs_covariance}
\end{eqnarray}

%--------covariance analysis---------------------
For a given point $x_i$ in the model state, the covariance to angular momentum component $y_j$ is proportional to the ensemble spread at that point ($\frac{1}{N-1}\left[\sum_{n=1}^N \left( e^n_{x_i} \right)^2 \right]$) times the weighting function at that point ($f_{i,j}$), plus the sum of the covariances between $x_i$ and all other other points in the model state [$\frac{1}{N-1}\sum_{n=1}^N e_{x_i}^n e_{x_k}^n$], each multiplied by their respective weighting functions ($f_{k,j}$). 
In other words, a point will be updated by an angular momentum observation if it is located in a region that is weighted strongly for that particular angular momentum component, and/or if it happens to covary strongly with other points that are strongly weighted in the integrals.  

% covariance figure-----------------------------------
Figure \ref{fig:covariances} shows the covariance between the zonal wind field and the axial angular momentum ($\chi_3$) over the Northern Hemisphere for three days in \ERPALL, roughly dividing between the 
stratosphere (100-24 hPa) 
and
troposphere (surface to 100 hPa).
Note that the covariance pattern does not resemble the weighting function of the axial wind integral (\ref{eq:h3}, which follows the cosine of latitude and is therefore zonally symmetric and maximal in the tropics), and varies strongly in time. 
This implies that dynamic, ensemble-based covariance estimates should be more useful thatn a static estimate based on the weighting functions.

However, the covariances in both regions are on the order of $10^{-11}\text{m/s}$.  
Considering the standard deviation of the zonal wind (of order 10 m/s) and the (dimensionless) axial angular momentum (of order $10^{-9}$), the corresponding state-to-observation correlations turn out to be on the order of $10^{-4}\text{m/s}$. 
The miniscule correlations are not surprising, given that each individual state component contributes only a small amount to the global angular momentum integral, but to 
correctly estimate them, the correlations need to be significantly larger than the sampling error that comes from using a finite ensemble, which 
scales as $1/\sqrt{N}$ \citep{Houtekamer1998}. 
This means that we would need upwards of $N=1000$ ensemble members to compute correlations that have at least the same order of magnitude as their sampling error, which is not computationally feasible. 

The only hope for attaining a reasonable correlation estimate from the 80-member model ensemble is if the covariance patterns have a large enough scale.  
Comparing the two rows of Figure \ref{fig:covariances} shows that, while
the covariance patterns in the stratosphere and troposphere are similar, the covariance field in the stratosphere has larger spatial scales, in keeping with the larger spatial structures of stratospheric flow.
This suggests that the analysis increment due to assimilation of global angular momentum might be the most accurate in the stratosphere, where an individual point is more covariant with its neighbors.  

% increments versus error figure-----------------------------------
Figure \ref{fig:error_increments} (a) shows the bias between the prior ensemble mean and the truth on 31 Jan 2009 in \ERPALL, i.e. after one month of assimilating the three angular momentum components. 
For simplicity, this figure shows a Northern Hemisphere-only slice at the 320 hPa level, i.e. roughly at the height of the subtropical jets.
We can compare this bias field to the analysis increment on that day (that is, the difference between the posterior and prior ensemble mean), which is shown in Fig. \ref{fig:error_increments}.
The true error and the analysis increment have some large-scale similarites, e.g. the North Atlantic jet stream winds are too weak relative to the truth (positive bias), and the assimilation increases wind in this region (positive increment). 
On the other hand, we can also find regions where the increment is in the opposite direction of the bias, e.g. over most of Asia. 
The analysis increments are also an order of magnitude smaller than the true error, which reflects how a single observation has to be spread over the entire state with 
miniscule state-to-observation correlations.

To see whether the assimilation increment has improved things or not, we compute the mean square error for each state vector component $i$:
\begin{eqnarray}
	E_i = \left(
		\left< x_i \right>-x_{i}^{t}
	\right)^2,
\end{eqnarray}
A sample map of the difference in the mean square error between the prior and posterior states for zonal wind (Fig. \ref{fig:error_increments}c) shows where the assimilation of global angular momentum increases or decreases the distance to the true state (here red shading means that the assimilation has reduction the bias relative to the truth).  
On this particular date, the assimilation has reduced the zonal wind bias over large swaths of North America, but with commensurate error increases over Asia. 

In order for a data assimilation to be considered reliable, it should not only 
decrease error but also reflect the correct error statistics in its covariance 
estimate. 
For ensemble filters, this means that the ensemble spread should be large 
enough that the true state can be considered  a sample of the probability 
distribution represented by the ensemble.
 \citet{Huntley2009} and \citet{Murphy1988} pointed out that this is the case when 
 \begin{eqnarray}
	 %---cut---\llbracket E_i \rrbracket = \frac{N+1}{N} \llbracket S_i \rrbracket, 
	 % (don't need the square brackets now that we show maps of scaled ensemble variance) 
	 E_i = \frac{N+1}{N} S_i. 
	 \label{eq:EvsS}
 \end{eqnarray}
In \ERPALL, assimilation decreases error only in some regions, but 
decreases the scaled ensemble spread everywhere (Fig. \ref{fig:error_increments}d), following (\ref{eq:y_a}) -(\ref{eq:state_update}).  
Thus, while the covariance model developed by the ensemble filter has a physical basis, its high degree of sampling error can also lead to analysis increments that make the state estimate worse, all while the ensemble spread decreases.  
This will ultimately lead to a divergent data assimilation system, where the ensemble clusters closer together around a mean state that is, generally, farther from the truth than implied by the ensemble. 


% point checks figure-------------------------------------------------------
To illustrate this result on a local level, Figure \ref{fig:point_checks} compares the zonal wind in the ensemble and its mean to the true state, with \NODA~ (no assimilation) and \ERPALL~ (assimilation of angular momentum). 
The ensemble in each cases is shown as averages over two regions: the tropospheric jet over the Atlantic (averaging between 30N-40N, 280W-360W, and 300hPa-200hPa), and the northern stratospheric polar vortex (averaging between 60N-90N, 30-24 hPa, and all longitudes).
For both measures, the angular momentum assimilation over the first three weeks or so clearly decreases the ensemble's spread about the mean, and gives the ensemble mean more realistic variations (which average out in the ensemble with no assimilation). 
However, only the polar vortex winds, which showed a more large-scale covariance structure (Fig. \ref{fig:covariances}) show improvement relative to the truth, and only for the first few weeks of assimilation. 

Thus we do see some improvement in the state estimate when assimilating the global angular momentum, which is perhaps a remarkable result given the high sampling error in the state-to-observation covariances. 
This improvement is nevertheless small, limited to regions with large-scale correlation patterns, and fades as the assimilation progresses and the ensemble spread increases.
A natural next question is to ask whether angular momentum observations could improve the state more if the ensemble spread is kept small by the assimilation of other, spatially localized, observations. 
This question will be treated in the next section.  
