When an angular momentum observation is assimilated, the update on the state variables (\ref{eq:state_update}) is proportional to the covariance between each state component (i.e. the wind, temperature, or surface pressure at some point) and the global angular momentum component in question ($\chi_1$,$\chi_2$, or $\chi_3$). 
The filter computes these covariances from the ensemble using (\ref{eq:covariance}).

%--------covariance analysis---------------------

To see what the ensemble-estimated covariance $c_{x_i y_j}$ looks like when $y_j$ is one of the angular momentum components, we can abbreviate the angular momentum integrals [(\ref{eq:I1})-(\ref{eq:h3})] as
%
\begin{eqnarray}
y_j = \sum_{i=1}^M f_{i,j} x_i,
\end{eqnarray}
%
where $f_{i,j}$ represents the spatial weighting that is applied to state component $i$ for angular momentum component $j$, times the latitude, longitude, and mass increments in the integrals, and $M$ represents the number of variables in the model state.
Then the observation error terms in (\ref{eq:covariance}) become
\begin{eqnarray}
	e_{y_j}^n = y_j^n - \left< y_j^n \right>
	= \sum_{k=1}^{M}f_{k,j}e_{x_k}^n,
\end{eqnarray}
which means that the state-to-observation covariance term becomes
\begin{eqnarray}
	c_{x_i y_j} = \frac{1}{N-1}
	\sum_{n=1}^{N}
	\left[
	e_{x_i}^n
	\sum_{k=1}^{M} f_{k,j}e_{x_k}^n
	\right].
	\label{eq:state_to_obs_covariance}
\end{eqnarray}

%--------covariance analysis---------------------
For a given point $x_i$ in the model state, the covariance to angular momentum component $y_j$ is proportional to the ensemble spread at that point ($\frac{1}{N-1}\left[\sum_{n=1}^N \left( e^n_{x_i} \right)^2 \right]$) times the weighting function at that point ($f_{i,j}$), plus the sum of the covariances between $x_i$ and all other other points in the model state [$\frac{1}{N-1}\sum_{n=1}^N e_{x_i}^n e_{x_k}^n$], each multiplied by their respective weighting functions ($f_{k,j}$). 
In other words, a point will be updated by an angular momentum observation if it is located in a region that is weighted strongly for that particular angular momentum component, and/or if it happens to covary strongly with other points that are.

% covariance figure-----------------------------------
The first row of Figure \ref{fig:covariances} shows, as an example, the covariance between the zonal wind field and the axial angular momentum ($\chi_3$), computed from the ensemble in Experiment 2, at 7-day intervals between 14 January and 11 February, and averaging over all vertical levels except for the highest three "sponge" levels of the model, which extend from about 14 to 3 hPa.  
The second row of this figure shows the corresponding correlations. 

Note that the covariance / correlation pattern varies strongly in time and is spatially heterogeneous; it does not resemble the weighting function of the axial wind integral (\ref{eq:h3}), which is maximal in the tropics.
The same is true for the covariance/correlation fields between the other state variables the other angular momentum components (not shown).
This implies that dynamic, ensemble-based covariance estimates should be useful for the assimilation of global angular momentum. 

However, the individual correlations are extremely small, on the order of 0.001. 
This is about two orders of magnitude smaller than the the sampling error due to a finite ensemble, which scales like $1/\sqrt{N}$ \citep{Houtekamer1998}.  

% increments versus error figure-----------------------------------
The covariances shown in Figure \ref{fig:covariances} determine how the ensemble filter maps the observation space increments in second column of Figure \ref{fig:fit_to_ERPs} onto the zonal wind field. 
The top row of Figure \ref{fig:error_increments} shows the bias between the ensemble mean and the truth in the Northerm Hemisphere, at the same time snapshots as the previous figure, and at 320 hPa. 
The second row of Fig. \ref{fig:error_increments} shows the analysis increments (i.e. posterior-prior ensemble mean) at the same times and vertical level. 
Throughout January, the true error and the analysis increments have similar patterns (e.g., on 21 Jan the prior ensemble mean zonal wind is too strong over the USA, and the analysis step increases the wind there), though 
the analysis increments are also an order of magnitude smaller than the true error, which makes sense, since the information from a single observation has to be spread over the entire state. 

As the assimilation progresses, the analysis increment agrees less and less with the prior error, and even sometime opposes it (e.g. over the northeastern Pacific on 11 Feb). 
Thus while the covariance model developed by the ensemble filter has a physical basis, it is highly prone to sampling error and seems to deteriorate in time. 

To answer the question of whether assimilating global angular momentum improves the modeled fields or not, 
Figure \ref{fig:ER} shows the difference in the mean square error (again in the zonal wind field), between experiment 1 (no assimlation) and experiment 2 (assimilating the angular momentum), at the same time snapshots as the previous figures. 
Figure \ref{fig:ER} also separates between the upper atmosphere (100-24 hPa, top row), and the lower atmosphere (1000-100 hPa, bottom row). 
Blue indicates that the assimilation of angular momentum has reduced the overall error, i.e. improved the state estimate.
Error reduction on the order to 10-20 m/s is evident at both upper and lower levels throughout the assimilation, though increases in error of a similar magnitude also appear early and grow in time. 
At the same time, the ensemble spread (not shown) consistently decreases to reflect the information ``gained'' from the observations. 
Thus suggests that we have a divergent data assimilation system, where the ensemble clusters closer together around a mean state that is, generally, farther from the truth than implied by the ensemble. 

% point checks figure-------------------------------------------------------
To illustrate this result on a local level, Figure \ref{fig:point_checks} compares the zonal wind in the ensemble and its mean to the true state, with no assimilation (left) and assimilation of angular momentum (right). 
The ensemble in each cases is shown as averages over two regions: the tropospheric extratropical jet (averaging between 30N-45N, between 300hPa-200hPa, and all longitudes), and the Northern Hemisphere stratospheric polar vortex (averaging between 60N-90N, between 30-24 hPa, and all longitudes).
For both measures, the angular momentum assimilation over the first three weeks or so decreases the ensemble's spread about the mean, and gives the ensemble mean a bit more of the day-to-day variations of the true state (which average out in the ensemble with not assimilation). 

In the first three weeks, assimilation clearly increases the agreement between the ensemble mean and the truth, especially in the North Polar Vortex. 
At later dates, the improvement is questionable. 

