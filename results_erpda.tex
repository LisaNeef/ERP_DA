\subsection{State-space diagnostics}
\label{sec:erpda}

To evaluate how the assimilation of the 3 AAM components changes the ensemble mean analysis state, we compare the assimilation of the three AAM components only (AAMDA) to the ensemble with no assimilation (NODA).
Figures \ref{fig:MSE_U} and \ref{fig:MSE_PS} both compare the mean square error (left column) to the ensemble variance scaled by $(N+1)/N$ (as in (\ref{eq:EvsS})), for experiment NODA (top row), experiment AAMDA (second row), and the difference between the two.
Figure \ref{fig:MSE_U} shows vertical profiles of mean square error or ensemble variance of the zonal wind, as a function of time (averaging meridionally and zonally). 
To focus on the surface signals, Figure \ref{fig:MSE_PS} shows the mean square error and ensemble variance of the surface pressure,  as a function of latitude and time (averaging vertically and zonally). 

% (*) error in NODA grows in the tropopsheric jets and at the highest model levels - the truth diverges from the ensemble mean where also the spread grows
Without assimilation (Fig. \ref{fig:MSE_U}a) the zonal wind in the true state diverges from that of the ensemble mean around the height of the extratropical jets (ca. 250hPa) and in the top four model levels (10hPa and above). 
The spread of the ensemble in this case (Fig. \ref{fig:MSE_U}b) looks similar but is slightly smaller. 
This implies that the true state probably has features that are averaged out in the ensemble mean, and the ensemble spread slightly underestimates the difference.

% (*) adding AAM obs reduces the spread in the 2 areas where it's largest
Assimilating observations of the three AAM components shows a similar mean error (Fig. \ref{fig:MSE_U}c) and ensemble variance (Fig. \ref{fig:MSE_U}d).
The difference between the two experiments (Fig. \ref{fig:MSE_U}e-f)  shows that the ensemble variance is reduced everywhere, and most strongly in the places where the initial spread was largest, the midlatitude jets and the highest levels. 
% (*) but: adding AAM really only reduces the error up high, and not consistently
However, the mean error is only reduced at the highest model levels, and over a few short pulses. 
Twice during the run time, the assimilation even increases the mean error. 

\textcolor{alert}{Also insert description of what happens at the surface, or 300 hPa or something, if relevant.}


\subsubsection{Rank histograms to diagnose filter divergence}

\textcolor{alert}{Here insert description of rank histograms before and after assimilation in various locations -- they will probably show that adding the integral observations increases filter divergence everywhere.}


\subsection{Evolution of the  Covariance Field}

Assimilation of the three global AAM components increases the error between the ensemble mean and the true state, especially as the data assimilation progresses, evenually causing the filter to diverge.
To understand why this happens, we can examine the ensemble-estimated covariance between local variables (in our case, winds or surface pressure) and the global AAM functions (\ref{eq:covariance}), which govern how and where each ensemble member is updated when an observation comes in. 

The covariance $c_{x_iy}$ is estimated for each state variable $x_i$ and each observation $y$ by the statistics of the model ensemble.  
A point in the model state can have a large covariance with the global AAM either if it has a large variance, or if it has a large correlation to the global AAM, or both.

