Having seen in Section \ref{sec:evaluation_variable} that Earth rotation parameters contain information about the atmospheric state, it seems natural to ask whether an atmosphere model could be constrained to the truth even more by assimilating these parameters. 
Experiments \NODA-\ERPRST ~address this question by assimilating synthetic observations of the angular momentum, as well as local temperature observations, using DART-CAM.
This analysis is similar to the studies of \citet{Saynisch2010,Saynisch2011} and \citet{Saynisch2012}, with two main differences:
(1) We use an atmosphere rather than an ocean model.
(2) We assimilate synthetic observations generated from a model simulation that serves as the ``true state''. This allows us to measure the true error.
The assimilation of synthetic observations in \ERPALL-\ERPRST also simuluates an idealized situation where the atmosphere accounts for 100\% of Earth rotation variations, which means that our experiments do not require an estimate of the oceanic or hydrospheric contributions to Earth rotation variations.

\subsection{Observation-space diagnosis}
\label{sec:obs_space}
Figure \ref{fig:fit_to_ERPs} compares the angular momentum of the ensembles in the three DART-CAM experiments (similar to \ref{fig:evalvariable_aam}, and again omitting $\chi_1$ for simplicity). 
The angular momentum components for each experiment are also compared to their mean and to the ``true'' angular momentum in each case.

With no assimilation (first column), the angular momentum functions again show how the ensemble spreads about the truth, and how the spread saturates after about one month.
If we assimilate the angular momentum functions alone (second column), the ensemble predictably clusters close to the true angular momentum, and captures the day-to-day angular momentum variations. 
The difference between the first two columns indicates that assimilating the angular momentum has imposed some kind of constraint upon the wind, temperature, and surface pressure fields, though, as will be shown below, this does not necessarily mean that those fields have also moved closer to the true state. 

The ensemble clusters even more tightly around the truth when instead of the angular momentum functions we assimilate local temperature observations (third column), which is a much stronger constraint on the model fields. 
Finally, adding the angular momentum observations to the regularly-spaced temperature observations (fourth column) slightly increases the agreement between the ensemble and the true state further.  
This suggests that the angular momentum observations may contain additional information that complements the information in the temperature observations, though the difference between assimilating only temperature (third column) and assimilating both temperature and angular momentum (fourth column) is small.  


\subsection{State Update from Angular Momentum Observations}
\label{sec:erpda}
\subsection{State-space diagnostics}
\label{sec:erpda}
\subsubsection{Error growth without assimilation}

Before examining the error in the ensemble when observations are assimilated, it is useful to examine how the ensemble in our assimilation system spreads relative to the true state when no observations are assimilated, because this shows us what information can actually be gained from assimilating.
Figure \ref{fig:NODA}(a) compares the  mean square error  between the ensemble mean and the true state (MSE hereafter) to the  variance of the ensemble after it has been scaled by $(N+1)/N$, as in (\ref{eq:EvsS}).
The first row shows MSE and ensemble variance for the zonal wind, as a function of vertical level and time, averaging zonally and meridionally.  
The second row of plots compares the MSE and the scaled ensemble spread in the surface pressure, over latitude and time, averaging zonally.  

For the zonal wind, both the error and the ensemble variance begin to grow first around 250 hPa, which is near the altitudes of the extratropical jets.
The growth of the scaled ensemble variance here is largely commensurate with the growth of the MSE.
\textcolor{unsure}{The relative agreement between the growth of the true error and the ensemble spread isn't surprising, since in our case the``truth'' is actually a realization of the same model that produced the ensemble.}

At the surface (Fig. \ref{fig:NODA}b-c) error growth happens first at high latitudes, especially in the Northern Hemisphere.
Here the scaled ensemble variance somewhat underestimates the MSE.
\textcolor{alert}{Do we have an explanation for this?  Or are they "close enough"? -- check after comparing to rank histograms.}

The error in the zonal wind field also becomes underestimated after abount a month, when the ensemble mean MSE shows large errors at the highest model levels (near the ``sponge'' layer, see Section \ref{sec:CAM}), which is not captured by the ensemble variance.
\textcolor{alert}{Again, need an explanation for this.}

As stated in the introduction, the goal of ensemble assimilation is to generate an ensemble that captures the true uncertainty in the estimated state, given the observations, which also means that the true state should, be a sample of the PDF represented by the ensemble.
To test whether our ensemble achieves this, we construct a rank histogram.

Figure \ref{fig:RH_basic}a shows the rank histogram for surface pressure in the NODA experiment, counting over all points in the model grid, and assimilation days 10-20. 
The histogram is mildly concave, indicating that the true state is somewhat more likely to be lower or higher than all ensemble members.
This means that the ensemble spread in this experiment is somewhat too small, \textcolor{unsure}{but by and large the truth can still be considered a sample of the ensemble}.  
\textcolor{alert}{Need to verify that the RH for zonal and meridional wind is qualitatively similar.}

In order for the assimilation experiments of the next section to be successful, they should achieve two goals: (1) to decrease the true error, and (2) to maintain an ensemble that is as or more representative of the true error as in the NoDA experiment.

\subsubsection{Assimilation of AAM}

Figure \ref{fig:RH_basic}b shows the surface pressure rank histogram for experiment AAM DA, i.e. after assimilating the three global excitation functions (\ref{eq:X1})-(\ref{eq:X3}).  
Clearly, the assimilation has made the rank histogram much more concave, which means that the truth is now a very frequent outlier of the PDF represented by the ensemble.
The rank histograms for the  wind fields, not shown here, are qualitatively similar.
This in turn suggests that the model ensemble has clustered more closely together in terms of the surface pressure, even while it is well spread out around the observed AAM (Fig. \ref{fig:fit_to_erps}, second column).

Figure \ref{fig:ERPALL_error_growth}a-b shows the MSE and scaled ensemble variance in the surface pressure, as in Fig. \ref{fig:NODA}b-c, but for the AAM DA experiment. 
Figure \ref{fig:ERPALL_error_growth}c-d shows the difference in MSE and scaled ensemble variance between AAM DA and No DA (positive values indicating larger values in AAM DA).  
When the AAM components are assimilated, MSE (Fig. \ref{fig:ERPALL_error_growth}a) grows more or less as in the No DA case (Fig. \ref{fig:NODA}c), but the difference between then (Fig. \ref{fig:ERPALL_error_growth}c) shows that the assimilation has decreased the error in some places, and increased it in others.
In contrast, the ensemble variance (Figure \ref{fig:ERPALL_error_growth}b,d) has decreased everywhere when the AAM components are assimilated.  




\subsection{Evolution of the  Covariance Field}

It was seen above that the ensemble filter is able to update the state more or less correctly at the beginning of the assimilation, actually makes the error between the ensemble mean and the true state worse later on in the assimilation period.
The degree to which each variable field is changed as the assimilation progresses depends on the covariances between local variables (in our case, winds or surface pressure) and the global AAM functions (\ref{eq:covariance}).
The covariance $c_{x_iy}$ is estimated for each state variable $x_i$ and each observation $y$ by the statistics of the model ensemble.  
A point in the model state can have a large covariance with the global AAM either if it has a large variance, or if it has a large correlation to the global AAM, or both.



\subsection{Assimilating AAM in the Presence of Conventional Observations}
\label{sec:added_value}
Even when a given type of observation is unable to significantly constrain a modeled state by itself, it may still improve the assimilation of other, more standard observations.  
%\cite{Pincus2011} showed that this is the case for cloud moisture and cloud fraction observations, which \hl{FILL IN}.
If angular momentum / Earth rotation parameter observations are assimilated in the presence of local observations, it is conceivable that an ensemble that is already clustered around the true state could perhaps be pushed even closer to the truth by adding the additional requirement that the global angular momentum should match observations. 
In this section, we compare experiments \RST~and \ERPRST, i.e. assimilating synthetic temperature observations with and without additional angular momentum observations.

Comparing these two experiments in terms of the fit to the angular momentum components (Fig. \ref{fig:fit_to_ERPs} columns 3-4), we see that the temperature observations alone already achieve a good fit to the true angular momentum, and that 
then adding observations of the angular momentum components further improves the fit.  
This means that the angular momentum observations add information to the ensemble, but the question remains as to whether the information is interpreted correctly by the model.

Figure \ref{fig:added_value_MSEincrement} shows zonal mean profiles of the increment (posterior minus prior) in the mean square error of zonal wind as a function of time, for 17 days of asssimilation in \RST~[assimilating temperatures, (a)] and \ERPRST~[assimilating temperatures and angular momentum, (b)].  
Both experiments show entirely negative values, meaning that the wind field is so well constrained that adding observations always pushes the ensemble mean closer to the truth. 
Adding the angular momentum observations in \ERPRST~(Fig. \ref{fig:added_value_MSEincrement}b) shows a weaker overall correction in the first week of assimilation, and then sometimes stronger corrections (e.g. around 200 hPa on 11 January). 
However, 
Figure \ref{fig:added_value_MSE} shows profiles of the difference in posterior zonal wind mean square error between \ERPRST~and \RST, and we can see that 
adding the angular momentum observations (\ERPRST) always maintains a larger zonal mean square error (short instances of negative values can be found, but they are small and disappear in the ensemble mean). 
At the same time, adding the angular momentum observations very slightly decreases the ensemble spread (not shown here) everywhere.  
This indicates that the ensemble filter tends to keep the ensemble from finding the true state by requiring it to fit the observed angular momentum.  



