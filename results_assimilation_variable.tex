This section evaluates whether an assimilation run can be improved by actually assimilating the Earth rotation data, using experiments \NODA-\ERPRST.
This analysis is similar to the studies of \citet{Saynisch2010,Saynisch2011} and \citet{Saynisch2012}, with two main differences:
(1) We use an atmosphere rather than an ocean model.
(2) We assimilate synthetic observations generated from a model simulation that serves as the ``true state''. This allows us to measure the true error.
The use of synthetic observations also simuluates an idealized situation where the atmosphere accounts for 100\% of Earth rotation variations, which means that our experiments do not require an estimate of the oceanic or hydrospheric contributions to Earth rotation variations.

\subsection{Observation-space diagnosis}
\label{sec:obs_space}
Figure \ref{fig:fit_to_ERPs} compares the modeled angular momentum functions for Experiments \NODA-\ERPRST (similar to \ref{fig:evalvariable_aam}, and again omitting $\chi_1$ for simplicity). 
The angular momentum components for each experiment are also compared to the ``true'' angular momentum in each case, which is assimilated in \ERPALL ~and \ERPRST.  

With no assimilation (\NODA, first column), the angular momentum functions again show how the ensemble spreads about the truth, and how the spread saturates after about one month.
If we assimilate the angular momentum functions alone (\ERPALL, second column), the ensemble predictably clusters close to the true angular momentum, and captures the day-to-day angular momentum variations. 
The difference between the first two columns indicates that assimilating the angular momentum has imposed some kind of constraint upon the wind, temperature, and surface pressure fields, though we do not yet know whether those fields have also moved closer to the true state. 

The ensemble clusters even more tightly around the truth when instead of the angular momentum functions we assimilate local temperature observations (\RST, third column), which is a much stronger constraint on the model fields. 
Finally, adding the angular momentum observations to the regularly-spaced temperature observations (\ERPRST, fourth column) slightly increases the agreement between the ensemble and the true state further.  
This suggests that the angular momentum observations may contain additional information that complemetns the information in the temperature observations, though the difference between \RST ~(assimilating only temperature) and \ERPRST ~(assimilating temperature and angular momentum) is small.  


\subsection{State Update from Angular Momentum Observations}
\label{sec:erpda}
\subsection{State-space diagnostics}
\label{sec:erpda}

To evaluate how the assimilation of the 3 AAM components changes the ensemble mean analysis state, we compare the assimilation of the three AAM components only (AAMDA) to the ensemble with no assimilation (NODA).
Figures \ref{fig:MSE_U} and \ref{fig:MSE_PS} both compare the mean square error (left column) to the ensemble variance scaled by $(N+1)/N$ (as in (\ref{eq:EvsS})), for experiment NODA (top row), experiment AAMDA (second row), and the difference between the two.
Figure \ref{fig:MSE_U} shows vertical profiles of mean square error or ensemble variance of the zonal wind, as a function of time (averaging meridionally and zonally). 
To focus on the surface signals, Figure \ref{fig:MSE_PS} shows the mean square error and ensemble variance of the surface pressure,  as a function of latitude and time (averaging vertically and zonally). 

% (*) error in NODA grows in the tropopsheric jets and at the highest model levels - the truth diverges from the ensemble mean where also the spread grows
Without assimilation (Fig. \ref{fig:MSE_U}a) the zonal wind in the true state diverges from that of the ensemble mean around the height of the extratropical jets (ca. 250hPa) and in the top four model levels (10hPa and above). 
The spread of the ensemble in this case (Fig. \ref{fig:MSE_U}b) looks similar but is slightly smaller. 
This implies that the true state probably has features that are averaged out in the ensemble mean, and the ensemble spread slightly underestimates the difference.

% (*) adding AAM obs reduces the spread in the 2 areas where it's largest
Assimilating observations of the three AAM components shows a similar mean error (Fig. \ref{fig:MSE_U}c) and ensemble variance (Fig. \ref{fig:MSE_U}d).
The difference between the two experiments (Fig. \ref{fig:MSE_U}e-f)  shows that the ensemble variance is reduced everywhere, and most strongly in the places where the initial spread was largest, the midlatitude jets and the highest levels. 
% (*) but: adding AAM really only reduces the error up high, and not consistently
However, the mean error is only reduced at the highest model levels, and over a few short pulses. 
Twice during the run time, the assimilation even increases the mean error. 

\textcolor{alert}{Also insert description of what happens at the surface, or 300 hPa or something, if relevant.}


\subsubsection{Rank histograms to diagnose filter divergence}

\textcolor{alert}{Here insert description of rank histograms before and after assimilation in various locations -- they will probably show that adding the integral observations increases filter divergence everywhere.}


\subsection{Evolution of the  Covariance Field}

Assimilation of the three global AAM components increases the error between the ensemble mean and the true state, especially as the data assimilation progresses, evenually causing the filter to diverge.
To understand why this happens, we can examine the ensemble-estimated covariance between local variables (in our case, winds or surface pressure) and the global AAM functions (\ref{eq:covariance}), which govern how and where each ensemble member is updated when an observation comes in. 

The covariance $c_{x_iy}$ is estimated for each state variable $x_i$ and each observation $y$ by the statistics of the model ensemble.  
A point in the model state can have a large covariance with the global AAM either if it has a large variance, or if it has a large correlation to the global AAM, or both.



\subsection{Assimilating AAM in the Presence of Conventional Observations}
\label{sec:added_value}
Even when a given type of observation is unable to significantly constrain a modeled state by itself, it may still improve the assimilation of other, more standard observations.  
If angular momentum / Earth rotation parameter observations are assimilated in the presence of local observations, it is conceivable that an ensemble that is already clustered around the true state could perhaps be pushed even closer to the truth by adding the additional requirement that the global angular momentum should match observations. 
In this section, we compare the 80-member CAM ensembles assimilating synthetic temperature observations (column 3 of Fig. \ref{fig:fit_to_ERPs}) to an ensemble assimilating both temperature and additional angular momentum observations (column 4 of Fig. \ref{fig:fit_to_ERPs}).
Figure \ref{fig:fit_to_ERPs} shows that assimilating 
temperature observations alone is enough to achieve a good fit to the true angular momentum, but also that 
then adding observations of the angular momentum improves the fit further, which implies that  
the angular momentum observations add information to the ensemble. 
The question remains as to whether the information is interpreted correctly by the model.

% (a) what happens to the spread?  
Figure \ref{fig:added_value_MSE}a shows the increment (posterior-prior) in the ensemble variance in the zonal wind for both experiments, averaged globally. 
The increment in the ensemble spread is always negative [following (\ref{eq:sigma_a}), and adjusts from an initially weak value to a stronger steady-state value as the ensemble, which was at the initial time constrained by both wind and temperature observations, adjusts to being constrained only by temperature observations. 
During this early period, adding the angular momentum observations causes a slightly stronger adjustment of the ensemble variance, indicating that the angular momentum observations add some information to the state estimate. 

% (b) but the error increments look really different -- here we see sampling error  
However, the posterior-prior difference in the mean square error of the ensemble mean (Fig. \ref{fig:added_value_MSE}b) shows that the information imposed by the angular momentum observations is not imposed correctly: while posterior error is also consistently lower than prior error, adding the angular momentum oobservations mostly results in a weaker error reduction during the adjustment period. 
Once the ensemble variance has reached steady state (after about 9 Jan), the additional angular momentum observations do not add any additional certainty (Fig. \ref{fig:added_value_MSE}a), and only sometimes cause stronger global mean error reduction. 

% (c) what does the total mean square error show? -- adding the ERPs increases error at worst and does nothing at best
The absolute posterior mean square error (Figure \ref{fig:added_value_MSE}c) is consistently larger when the angular momentum observations are added: the additional observations increase the global mean error at worst (at the beginning of the assimilation), and have no impact at best (at steady-state). 
We therefore find no clear benefit of assimilating atmospheric angular momentum (or Earth rotation parameter) observations, either with or without the assimilation of conventional observations. 



