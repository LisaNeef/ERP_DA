%%%%%%%%%%%%%%%%%%%%%%%%%%%%%%%%%%%%%%%%%%%%%%%%%%%%%%%%%%%%%%%%%%%%%%%%%%%%
% AGUtmpl.tex: this template file is for articles formatted with LaTeX2e,
% Modified March 2013
%
% This template includes commands and instructions
% given in the order necessary to produce a final output that will
% satisfy AGU requirements.
%
% PLEASE DO NOT USE YOUR OWN MACROS
% DO NOT USE \newcommand, \renewcommand, or \def.
%
% FOR FIGURES, DO NOT USE \psfrag or \subfigure.
%
%%%%%%%%%%%%%%%%%%%%%%%%%%%%%%%%%%%%%%%%%%%%%%%%%%%%%%%%%%%%%%%%%%%%%%%%%%%%
%
% All questions should be e-mailed to latex@agu.org.
%
%%%%%%%%%%%%%%%%%%%%%%%%%%%%%%%%%%%%%%%%%%%%%%%%%%%%%%%%%%%%%%%%%%%%%%%%%%%%
%
% Step 1: Set the \documentclass
%
% There are two options for article format: two column (default)
% and draft.
%
% PLEASE USE THE DRAFT OPTION TO SUBMIT YOUR PAPERS.
% The draft option produces double spaced output.
%
% Choose the journal abbreviation for the journal you are
% submitting to:

% jgrga JOURNAL OF GEOPHYSICAL RESEARCH
% gbc   GLOBAL BIOCHEMICAL CYCLES
% grl   GEOPHYSICAL RESEARCH LETTERS
% pal   PALEOCEANOGRAPHY
% ras   RADIO SCIENCE
% rog   REVIEWS OF GEOPHYSICS
% tec   TECTONICS
% wrr   WATER RESOURCES RESEARCH
% gc    GEOCHEMISTRY, GEOPHYSICS, GEOSYSTEMS
% sw    SPACE WEATHER
% ms    JAMES
%
%
%
% (If you are submitting to a journal other than jgrga,
% substitute the initials of the journal for "jgrga" below.)

\documentclass[draft,jgrga]{agutex}

% To create numbered lines:

% If you don't already have lineno.sty, you can download it from
% http://www.ctan.org/tex-archive/macros/latex/contrib/ednotes/
% (or search the internet for lineno.sty ctan), available at TeX Archive Network (CTAN).
% Take care that you always use the latest version.

% To activate the commands, uncomment \usepackage{lineno}
% and \linenumbers*[1]command, below:

% \usepackage{lineno}
% \linenumbers*[1]

%  To add line numbers to lines with equations:
%  \begin{linenomath*}
%  \begin{equation}
%  \end{equation}
%  \end{linenomath*}
%%%%%%%%%%%%%%%%%%%%%%%%%%%%%%%%%%%%%%%%%%%%%%%%%%%%%%%%%%%%%%%%%%%%%%%%%
% Figures and Tables
%
%
% DO NOT USE \psfrag or \subfigure commands.
%
%  Figures and tables should be placed AT THE END OF THE ARTICLE,
%  after the references.
%
%  Uncomment the following command to include .eps files
%  (comment out this line for draft format):
%  \usepackage[dvips]{graphicx}
%
%  Uncomment the following command to allow illustrations to print
%   when using Draft:
%  \setkeys{Gin}{draft=false}
%
% Substitute one of the following for [dvips] above
% if you are using a different driver program and want to
% proof your illustrations on your machine:
%
% [xdvi], [dvipdf], [dvipsone], [dviwindo], [emtex], [dviwin],
% [pctexps],  [pctexwin],  [pctexhp],  [pctex32], [truetex], [tcidvi],
% [oztex], [textures]
%
% See how to enter figures and tables at the end of the article, after
% references.
%
%  Other packages:
\usepackage{amsmath}
\usepackage[usenames]{color}
\usepackage{ulem}
\usepackage{url}
\usepackage{lscape}
\usepackage[pdftex]{graphicx}
%\usepackage{epstopdf}
%
%
%% ------------------------------------------------------------------------ %%
%
%  ENTER PREAMBLE
%
%% ------------------------------------------------------------------------ %%

% Author names in capital letters:
\authorrunninghead{NEEF AND MATTHES}

% Shorter version of title entered in capital letters:
\titlerunninghead{ASSIMILATION OF INTEGRALS}

%Corresponding author mailing address and e-mail address:
\authoraddr{Corresponding author: L.J. Neef,
%Helmholtz -Zentrum f{\"u}r Ozeanforschung (GEOMAR),
D{\"u}sternbrooker Weg 20,
D-24105 Kiel, Germany
(neef@geomar.de)}

\begin{document}
\bibliographystyle{agu08}

%% ------------------------------------------------------------------------ %%
%
%  TITLE
%
%% ------------------------------------------------------------------------ %%


\title{Successes and Failures of Assimilating Global Integrals with an Ensemble Filter}
%
% e.g., \title{Terrestrial ring current:
% Origin, formation, and decay $\alpha\beta\Gamma\Delta$}
%

%% ------------------------------------------------------------------------ %%
%
%  AUTHORS AND AFFILIATIONS
%
%% ------------------------------------------------------------------------ %%


%Use \author{\altaffilmark{}} and \altaffiltext{}

% \altaffilmark will produce footnote;
% matching \altaffiltext will appear at bottom of page.

 \authors{Lisa Neef\altaffilmark{1} and
Katja Matthes \altaffilmark{1}}

\altaffiltext{1}{Ocean Circulation and Climate Dynamics - Marine Meteorology,
   Helmholtz Centre for Ocean Research Kiel (GEOMAR), Kiel, Germany.}


%% ------------------------------------------------------------------------ %%
%
%  ABSTRACT
%
%% ------------------------------------------------------------------------ %%

% >> Do NOT include any \begin...\end commands within
% >> the body of the abstract.

\begin{abstract}
(Type abstract here)
\end{abstract}

%% ------------------------------------------------------------------------ %%
%
%  BEGIN ARTICLE
%
%% ------------------------------------------------------------------------ %%

% The body of the article must start with a \begin{article} command
%
% \end{article} must follow the references section, before the figures
%  and tables.

\begin{article}

%% ------------------------------------------------------------------------ %%
%
%  MATH ABBREVIATSIONS
%
%% ------------------------------------------------------------------------ %%
\newcommand{\degree}{\ensuremath{^\circ}}


%% ------------------------------------------------------------------------ %%
%
%  TEXT
%
%% ------------------------------------------------------------------------ %%

\section{Introduction}

 \textbf{General Issue}
\begin{itemize}
 \item Data assimilation is the process of inserting observations of the Earth system into dynamical models, by computing, at each observation time, the best estimate between what is predicted by the model on the one hand, and what is observed by measurement on the other.
 \item Data assimilation was first developed in the context of numerical weather prediction, where the emphasis is using data to compute the best possible model initial state, in order to improve forecasts.  
  \item Atmospheric data assimilation is now at the stage where implementation of the algorithms is accessible even to non-experts, via tools such as the Data Assimilation Research Testbed \citep[DART]{andersonetal2009} and the PDAF [\textcolor{red}{DEFINE and CITE}].   These tools make it possible to extend the algorithms that were originally developed for NWP to climate studies.  Using data assimilation, we can  constrain climate simulations  to reality, while still changing individual model parameters to meet the needs of an experimental design. 
\end{itemize}

 \textbf{scientific context and State-of-the-art }
\begin{itemize}
\item In the climate modeling context, data assimilation in some form is frequently used  to capture real events (e.g. the El Nino of a given year), while other factors in the modeling environment can be changed.  Assimilation can also be used to constrain  one part of the Earth system, and then explore the variability that ensues in other parts \textcolor{red}{[CITE EXAMPLE]}.
\item \textcolor{red}{(Say something about reanalysis.)}
\item But what observations should be used in such simulations?
\item In general, it is not obvious which variables need to be observed in order to adequately constrain the model.
\end{itemize}
%
 \textbf{Problems}
 \begin{itemize}
 \item It has been suggested to assimilation Earth rotation parameters, i.e. anomalies in the Earth's rotation rate and the orientation of its rotational pole -- have been suggested as a useful variable for data assimilation because they reflect changes in the angular momentum of the Earth system.  On subseasonal to interannual timescales, length-of-day anomalies (hereafter $\Delta$LOD) is essentially a measure of the angular momentum of the atmosphere [\textcolor{red}{CITE}], while polar motion (hereafter PM) reflects the combined angular momentum of the atmosphere, ocean, and hydrosphere.
 \item \citet{saynischetal2011a,saynischetal2011b} and \citet{saynischthomas2012} used advanced data assimilation  methods to fit
the excitation of $\Delta$LOD and PM in an ocean model to the observed variations of these paramters, by adjusting boundary parameters (atmospheric wind stresses and
freshwater flux). They found that strong adjustments must be made in boundary
parameters in order to close the observed discrepancy in the oceanic angular
momentum budget.
However, these studies  did not show  whether the assimilation of $\Delta$LOD and PM
actually brought the modeled ocean state closer to reality, nor whether the assimilation of these values added value to other variables that were also assimilated.
%\item \textcolor{red}{Mention \citet{dirrenhakim2005}, who showed that you can only update the mean state, and not deviations from it, when making observations of temporal averages.}
\item While the prospect of assimilating non-gridpoint observations is exciting an a potential new source for information about the Earth system, the assimilation of such observations is not straightforward and has several inherent difficulties.
\end{itemize}
%
%
 \textbf{Solution}
In this study we identify some of the major difficulties associated with assimilating average or integral observations, using the example of atmospheric angular momentum.  Instead, we show that the main value of global-integral observations is as a tool for evaluating the fidelity of an assimilation system consisting of more traditional, gridpoint observations.
%
%
 \textbf{Costs / benefits}: 
This result yields a handy tool for evaluating and quantifying the performance of data-constrained climate simulations. 
%
%

%-------------------------------------METHODS-----------------------------------------------------------------

\section{Data Assimilation System}

\subsection{Atmospheric Model}
Assimilation experiments are performed using the Community Atmosphere Model 5 \citep[CAM5 hereafter]{nealeetal2011}.
CAM5 is run with the finite-volume dynamical core at \textcolor{blue}{$2.5$ deg. horizontal resolution and 30 hybrid-coordinate vertical levels, with a top near 3 hPa}.
\textcolor{red}{How is CAM damped at the top?}
CAM5 forms the atmospheric component of the Community Earth System Model (CESM).

\subsection{Data Assimilation}
Assimilation is performed using an Ensemble Adjustment Kalman Filter (EAKF) within the Data Assimilation Research Testbed \citep[DART hereafter]{andersonetal2009, raederetal2012}).
DART is an open-source, community facility that makes ensemble data assimilation available for any model, using any observations.
It is available online at \url{www.image.ucar.edu/DAReS/DART}.

In ensemble filter assimilation algorithms such as the EAKF, covariances between components of the model state and the observations are estimated using an ensemble, which itself is updated with the observed information at each analysis time.

\subsection{Perfect-model experiments}
\label{sec:experiments}
% perfect model experiments
In this study we  perform so-called perfect-model experiments, wherein the ``truth'' is actually a realization of the model, and observations are generated from this realization with known error statistics.
Experiments are performed with an 80-member ensemble, which itself is generated from one full year of assimilation of the uniform grid of radiosonde observations described in Section \ref{sec:radiosondes}.

% observation generation
For our experiments, observations of $\chi_1$, $\chi_2$, and $\chi_3$ are generated every 24 hours.
This simulates the true observations of the ERPs ($p_1$, $p_2$, and $\Delta$LOD),
(\textcolor{blue}{$\Delta$LOD isn't really the observation .. it's UT1-UTC -- but what are they?})
which are available once daily (\textcolor{red}{CITE}).


% output diagnostics
Our main evaluation diagnostic is root mean square error (RMSE) between the ensemble mean the the true state.
To evaluate how much information is actually gained from assimilation in each case, the RMSE for each case will be shown relative to o a case where the 80-member initial ensemble is evolved forward in time without assimilation, hereafter known as the``No-DA'' run.


\subsection{Atmospheric Angular Momentum Observations}
\label{sec:AAM}
In this study we generate synthetic observations of atmospheric angular momentum (AAM) using the following non-dimensional equations derived by \citet{barnesetal1983}:
\begin{eqnarray}
{\bf \chi}_{1}(t) &=& \frac{1.608}{\Omega \left( C-A' \right)}
  \left[ 0.684 \Omega \Delta {\bf I}_{13}(t) + \Delta h_1(t)  \right] \label{eq:X1} \\
{\bf \chi}_{2}(t) &=& \frac{1.608}{\Omega \left( C-A' \right)}
  \left[ 0.684 \Omega \Delta {\bf I}_{23}(t) + \Delta h_2(t)  \right] \label{eq:X2}\\
{\bf \chi}_{3}(t) &=& \frac{0.997}{\Omega C_m}
  \left[ 0.750 \Omega \Delta {\bf I}_{33}(t) + \Delta h_{3}(t) \label{eq:X3} \right].
\end{eqnarray}
Here the ${\bf I}_{ij}$ terms represent components of the atmospheric inertia tensor:
\begin{eqnarray}
  I_{13} &=& -\int R^2 \cos \phi \sin \phi \cos \lambda dM 
  \label{eq:I1}\\
  I_{23} &=& -\int R^2 \cos \phi \sin \phi \sin \lambda dM 
  \label{eq:I2}\\
  I_{33} &=&  \int R^2 \cos^2 \phi dM ,
  \label{eq:I3}
\end{eqnarray}
and the $h_i$ terms he relative angular momentum (due to winds) in each direction:
\begin{eqnarray}
  h_{1}  &=& -\int R \left[u \sin \phi \cos \lambda - v \sin \lambda \right] dM 
    \label{eq:h1}\\
  h_{2}  &=& -\int R \left[u \sin \phi \sin \lambda + v \cos \lambda \right] dM 
    \label{eq:h2}\\
  h_{3}  &=&  \int R u \cos \phi dM.
    \label{eq:h3}
\end{eqnarray}
%
$R = 6371.0$ km is the radius of the Earth, $\Omega = 7.292115\times 10^{-5} \text{rad}/\text{s}$ the average rotation rate, and $g = 9.81 \text{m}/\text{s}^2$ is the acceleration due to gravity.
 $C = 8.0365 \times 10^{37} \text{kg} \text{m}^2$ and $A = 8.0101 \times 10^{37} \text{kg} \text{m}^2$ are the  axial and next-largest principal moments of inertia of the solid Earth, and $C_m = 7.1236 \times 10^{37} \text{kg} \text{m}^2$ is the principal inertia tensor component of the Earth's mantle \citep{gross2009}.

In reality, of course, atmospheric angular momentum can not be directly observed.
What can be directly observed, however, are the variations in Earth rotation parameters (ERPs) that are excited by the above angular momentum functions.
In fact, the non-dimensional representation of AAM in (\ref{eq:X1})-(\ref{eq:X3}) makes it easy to map angular momentum changes to equivalent changes in the polar orientation (or polar motion) and the rate of Earth's rotation. 
Polar motion is defined by two anomalies in the orientation of the rotational pole, $p_1$, defined as \textcolor{red}{FILL IN}, and $p_2$ defined as \textcolor{red}{FILL IN}.  
Their variation is related to variations in the equatorial AAM functions (\ref{eq:X1})-(\ref{eq:X2})  by a rotation of the reference frame:
\begin{eqnarray}
  p_1 + \frac{\dot{p_2}}{\sigma_0} &=& \chi_1 \\
  -p_2 + \frac{\dot{p_1}}{\sigma_0} &=& \chi_2,
\label{eq:X12_to_PM}
\end{eqnarray}
where $\sigma_0$ represents the Chandler frequency, an intrinsic wobble frequency of the solid Earth \textcolor{red}{(CITE)}.

The axial AAM function $\chi_3$ corresponds to unit changes in the rate of rotation of the Earth, and therefore also unit anomalies in the length of a day ($\Delta$LOD):
\begin{eqnarray}
\frac{\Delta \text{LOD}}{ \text{LOD}_0 } &=& \Delta \chi_3,
\end{eqnarray}
where $\text{LOD}_0$ denotes the sideral length-of-day \textcolor{red}{[DEFINE]}.  The length-of-day anaomlies are observed as $\Delta \text{LOD}  \equiv \text{UT1} - \text{IAT}$ where \textcolor{red}{DEFINE UT1 and IAT}.


Note that the zonal integration in the axial AAM terms [(\ref{eq:I3}) and (\ref{eq:h3})] weights all longitudes equally.
In practice, this means that mass anomalies tend to cancel one another out in the zonal integral, with the result that the axial mass excitation (\ref{eq:I3}) is usually several orders of magnitude smaller than the axial wind excitation (\ref{eq:h3}) \textcolor{red}{(CITE)}.

Nearly the opposite is true for the equatorial AAM terms [(\ref{eq:I1})-(\ref{eq:I2}) and (\ref{eq:h1})-(\ref{eq:h2})], where stronger zonal asymmetry in the wind and mass fields leads to a larger global integral.
For these functions, the mass terms [(\ref{eq:I1}) and (\ref{eq:I2})] typically outweigh the wind terms [(\ref{eq:h1}) and (\ref{eq:h2})] by a few factors \textcolor{red}{[CITE]}.


Of the polar motion angles, $p_1$ largely reflects surface pressure variations over the oceans, and $p_2$ over the continents.
The effect of short-timescale surface pressure variations over the ocean tend to be evened out by corresponding displacement of the ocean surface (the so-called ``inverse barometer'' effect), reducing the total angular momentum transfered to the Earth.  
Sub-annual $p_1$ variations therefore depend largely on other sources of AM.
\textcolor{red}{---this statement makes no sense.  Look at Henryk's paper and sort out what really happens with $p_1$}.
In contrast, $p_2$, which is strongly weighted over land, has a pronounced annual cycle due to the yearly appearance of the Siberian High (CITE) and tends to show strong negative anomalies in the two months preceding sudden stratospheric warmings \citep[in preparation]{neefmatthes2013sub}.


\subsection{Synthetic Radiosonde Observations}






%-------------------------------------RESULTS 1- integral obs suck------------------------------------------------------
\section{Can we constrain the state using global integral observations?}

\subsection{Prior and Posterior fit to Observed ERPs}
Figure \ref{fig:ERPs} shows the fit of the ensemble to the observed ERP excitation functions in each of the four main experiments outlined in Section \ref{sec:experiments}.
The case of no assimilation (Fig. \ref{fig:ERPs}(A)) starts out with the ensemble clustered closely around the true ERPs.
The ensemble spreads noticeably after about a week, and by the end of January, the ensemble mean lacks any of the short-timescale features of the true ERPs.

When ERPs are assimilated (Fig. \ref{fig:ERPs}(B)), the ensemble members naturally agree much more in their predicted ERPs.  
Thus the assimilation of ERPs is successful in the sense that the wind and surface pressure fields in each ensemble member are nudged enough to give each ensemble member an AAM vector that is close to what is observed.
However, the relative success in fitting the ERPs does not automatically imply that the state variables themselves are closer to the true state.  
This will be investigated in the following section.

An even greater contrast is seen when radiosonde observations are assimilated (Fig. \ref{fig:ERPs}(C)).
Here the true ERPs are matched closely for all ensemble members, for the entirety of the assimilation run, even though no ERPs are assimilated.
\textcolor{red}{is there any point in showing this case?  the fit is so perfect here that we don't really need ERPs.}

\subsection{Limited Error reduction in Wind and Pressure Fields}

A comparison between the NODA and ERPDA experiments in terms of the model state space is given in Figures \ref{fig:ERP_DA_U_p_time}-\ref{fig:ERP_DA_PS_lat_time}.
Fig. \ref{fig:ERP_DA_U_p_time}(a) shows the RMSE in zonal wind (see Section \ref{sec:experiments}), averaging over all latitudes and longitudes, as a function of height and time, in the NODA experiment.
As in the corresponding observation-space plots [Fig. \ref{fig:ERPs}(A)] the error begins to spread noticeably after about a week, and saturates after about a month.
Error growth is strongest in the tropospheric jets (around 300hPa) and near the model lid.

Fig. \ref{fig:ERP_DA_U_p_time}(b) shows the same but for assimilation of ERPs.  
Here the error growth is similar but slightly weaker, especially during the first few weeks.
The reduction of error between NODA and ERPALL is shown in Fig. \ref{fig:ERP_DA_U_p_time}(c).
It can be seen that assimilating ERPs reduces the error most of all at the top of the model, and visible error reduction lasts through about the end of February at which point the assimilation of ERPs actually \textit{increases} the error.

Fig. \ref{fig:ERP_DA_U_p_time}(d) shows the innovation (Prior-Posterior) at each analysis time.
The innovation, like the error reduction, is also strongest near the model top.
The innovation is generally strongest when the true error is largest, which means that the filter correctly chooses a strong adjustment at times when the error is largest.
However, while this results in a large error reduction early on in the assimilation period, it increases error towards the end of the assimilation period.

\textcolor{red}{[Insert explanation for why increments are concentrated up high.]}

Figure \ref{fig:ERP_DA_U_lat_time} examines what happens to the wind at 300hPa as the assimilation progresses. 
This figure is similar to Fig.~\ref{fig:ERP_DA_U_p_time}, but now showing 300hPa wind as a function of latitude and time.
Again, the grown of error is slower when ERPs are assimilated [Fig. \ref{fig:ERP_DA_U_lat_time}(b)] than with no assimilation [Fig. \ref{fig:ERP_DA_U_lat_time}(a)], and we see visible error reduction through January and February [Fig. \ref{fig:ERP_DA_U_lat_time}(c)].

Figure \ref{fig:ERP_DA_PS_lat_time} is similar to Fig. \ref{fig:ERP_DA_U_lat_time}, but showing surface pressure.
Here the error growth, and the error reduction brought about my the assimilation of the ERPs, are largely at latitudes about 50\degree N.
\textcolor{red}{Why is that?}

\subsection{Evolution of the Ensemble-Estimated Covariance Field}

\textcolor{red}{Describle how the covariances between local wind or mass variables, and global AAM, evolve in time as estimated by the ensemble.  It doesn't reach some sort of steady state but also doesn't seem to reflect reality?  (How do we know this?)}



%-------------------------------------RESULTS 1- integral obs suck------------------------------------------------------
\section{Integral observations as an evaluation tool}


\subsection{Dynamical implications of each ERP component} 
\textcolor{red}{In this section, write about how the different ERP components reflect different aspects of the model state.  How can we illustrate this with a figure?}


%%% End of body of article:

%%%%%%%%%%%%%%%%%%%%%%%%%%%%%%%%
%% Optional Appendix goes here
%
% \appendix resets counters and redefines section heads
% but doesn't print anything.
% After typing  \appendix
%
% \section{Here Is Appendix Title}
% will show
% Appendix A: Here Is Appendix Title
%
%%%%%%%%%%%%%%%%%%%%%%%%%%%%%%%%%%%%%%%%%%%%%%%%%%%%%%%%%%%%%%%%
%
% Optional Glossary or Notation section, goes here
%
%%%%%%%%%%%%%%
% Glossary is only allowed in Reviews of Geophysics
% \section*{Glossary}
% \paragraph{Term}
% Term Definition here
%
%%%%%%%%%%%%%%
% Notation -- End each entry with a period.
% \begin{notation}
% Term & definition.\\
% Second term & second definition.\\
% \end{notation}
%%%%%%%%%%%%%%%%%%%%%%%%%%%%%%%%%%%%%%%%%%%%%%%%%%%%%%%%%%%%%%%%
%
%  ACKNOWLEDGMENTS

\begin{acknowledgments}
(Text here)
\end{acknowledgments}

%% ------------------------------------------------------------------------ %%
%%  REFERENCE LIST AND TEXT CITATIONS
%
%----temporary bibliography
\bibliography{nathan,datass}
%----temporary bibliography

% Either type in your references using
% \begin{thebibliography}{}
% \bibitem{}
% Text
% \end{thebibliography}
%
% Or,
%
% If you use BiBTeX for your references, please use the agufull08.bst file (available at % ftp://ftp.agu.org/journals/latex/journals/Manuscript-Preparation/) to produce your .bbl
% file and copy the contents into your paper here.
%
% Follow these steps:
% 1. Run LaTeX on your LaTeX file.
%
% 2. Make sure the bibliography style appears as \bibliographystyle{agufull08}. Run BiBTeX on your LaTeX
% file.
%
% 3. Open the new .bbl file containing the reference list and
%   copy all the contents into your LaTeX file here.
%
% 4. Comment out the old \bibliographystyle and \bibliography commands.
%
% 5. Run LaTeX on your new file before submitting.
%
% AGU does not want a .bib or a .bbl file. Please copy in the contents of your .bbl file here.

\begin{thebibliography}{}

%\providecommand{\natexlab}[1]{#1}
%\expandafter\ifx\csname urlstyle\endcsname\relax
%  \providecommand{\doi}[1]{doi:\discretionary{}{}{}#1}\else
%  \providecommand{\doi}{doi:\discretionary{}{}{}\begingroup
%  \urlstyle{rm}\Url}\fi
%
%\bibitem[{\textit{Atkinson and Sloan}(1991)}]{AtkinsonSloan}
%Atkinson, K., and I.~Sloan (1991), The numerical solution of first-kind
%  logarithmic-kernel integral equations on smooth open arcs, \textit{Math.
%  Comp.}, \textit{56}(193), 119--139.
%
%\bibitem[{\textit{Colton and Kress}(1983)}]{ColtonKress1}
%Colton, D., and R.~Kress (1983), \textit{Integral Equation Methods in
%  Scattering Theory}, John Wiley, New York.
%
%\bibitem[{\textit{Hsiao et~al.}(1991)\textit{Hsiao, Stephan, and
%  Wendland}}]{StephanHsiao}
%Hsiao, G.~C., E.~P. Stephan, and W.~L. Wendland (1991), On the {D}irichlet
%  problem in elasticity for a domain exterior to an arc, \textit{J. Comput.
%  Appl. Math.}, \textit{34}(1), 1--19.
%
%\bibitem[{\textit{Lu and Ando}(2012)}]{LuAndo}
%Lu, P., and M.~Ando (2012), Difference of scattering geometrical optics
%  components and line integrals of currents in modified edge representation,
%  \textit{Radio Sci.}, \textit{47},  RS3007, \doi{10.1029/2011RS004899}.

\end{thebibliography}

%Reference citation examples:

%...as shown by \textit{Kilby} [2008].
%...as shown by {\textit  {Lewin}} [1976], {\textit  {Carson}} [1986], {\textit  {Bartholdy and Billi}} [2002], and {\textit  {Rinaldi}} [2003].
%...has been shown [\textit{Kilby et al.}, 2008].
%...has been shown [{\textit  {Lewin}}, 1976; {\textit  {Carson}}, 1986; {\textit  {Bartholdy and Billi}}, 2002; {\textit  {Rinaldi}}, 2003].
%...has been shown [e.g., {\textit  {Lewin}}, 1976; {\textit  {Carson}}, 1986; {\textit  {Bartholdy and Billi}}, 2002; {\textit  {Rinaldi}}, 2003].

%...as shown by \citet{jskilby}.
%...as shown by \citet{lewin76}, \citet{carson86}, \citet{bartoldy02}, and \citet{rinaldi03}.
%...has been shown \citep{jskilbye}.
%...has been shown \citep{lewin76,carson86,bartoldy02,rinaldi03}.
%...has been shown \citep [e.g.,][]{lewin76,carson86,bartoldy02,rinaldi03}.
%
% Please use ONLY \citet and \citep for reference citations.
% DO NOT use other cite commands (e.g., \cite, \citeyear, \nocite, \citealp, etc.).

%% ------------------------------------------------------------------------ %%
%
%  END ARTICLE
%
%% ------------------------------------------------------------------------ %%
\end{article}
\newpage
%
%
%% Enter Figures and Tables here:
%


 \begin{figure}
%\includegraphics[width=\textwidth]{../../Plots/ERP_DA/} 
 \caption{\textcolor{red}{Fit to ERP observations with No DA and assimilating ERPs -- we see that it's fairly easy to make the fit.}}
 \label{fig:ERPs}
\end{figure}

 \begin{figure}
%\includegraphics[width=\textwidth]{../../Plots/ERP_DA/} \\
 \caption{\textcolor{red}{Error reduction in U300 and PS fields as the assimilation progresses -- we start out with a pretty healthy error reduction, but then it goes away.}}
 \label{fig:ER_state_space}
\end{figure}

 \begin{figure}
%\includegraphics[width=\textwidth]{../../Plots/ERP_DA/} 
 \caption{\textcolor{red}{Ensemble versus truth in the NAO index in NoDA and ERPDA -- we don't gain much as far as large-scale variability goes.}}
 \label{fig:NAO}
\end{figure}

 \begin{figure}
%\includegraphics[width=\textwidth]{../../Plots/ERP_DA/} 
 \caption{\textcolor{red}{Evolution of the analysis increment (proportional to covariances) as the assimilation progresses. -- Nothing coherent emerges, and for some reason, observations up high get all the weight.}}
 \label{fig:increment}
\end{figure}

 \begin{figure}
%\includegraphics[width=\textwidth]{../../Plots/ERP_DA/} 
 \caption{\textcolor{red}{Evaluation of other data assimilation experiments (with localized observations) using ERPs.}}
 \label{fig:increment}
\end{figure}


% DO NOT USE \psfrag or \subfigure commands.
%
% Figure captions go below the figure.
% Table titles go above tables; all other caption information
%  should be placed in footnotes below the table.
%
%----------------
% EXAMPLE FIGURE
%
% \begin{figure}
% \noindent\includegraphics[width=20pc]{samplefigure.eps}
% \caption{Caption text here}
% \label{figure_label}
% \end{figure}
%
% ---------------
% EXAMPLE TABLE
%
%\begin{table}
%\caption{Time of the Transition Between Phase 1 and Phase 2\tablenotemark{a}}
%\centering
%\begin{tabular}{l c}
%\hline
% Run  & Time (min)  \\
%\hline
%  $l1$  & 260   \\
%  $l2$  & 300   \\
%  $l3$  & 340   \\
%  $h1$  & 270   \\
%  $h2$  & 250   \\
%  $h3$  & 380   \\
%  $r1$  & 370   \\
%  $r2$  & 390   \\
%\hline
%\end{tabular}
%\tablenotetext{a}{Footnote text here.}
%\end{table}

% See below for how to make sideways figures or tables.

\end{document}

%%%%%%%%%%%%%%%%%%%%%%%%%%%%%%%%%%%%%%%%%%%%%%%%%%%%%%%%%%%%%%%

More Information and Advice:

%% ------------------------------------------------------------------------ %%
%
%  SECTION HEADS
%
%% ------------------------------------------------------------------------ %%

% Capitalize the first letter of each word (except for
% prepositions, conjunctions, and articles that are
% three or fewer letters).

% AGU follows standard outline style; therefore, there cannot be a section 1 without
% a section 2, or a section 2.3.1 without a section 2.3.2.
% Please make sure your section numbers are balanced.
% ---------------
% Level 1 head
%
% Use the \section{} command to identify level 1 heads;
% type the appropriate head wording between the curly
% brackets, as shown below.
%
%An example:
%\section{Level 1 Head: Introduction}
%
% ---------------
% Level 2 head
%
% Use the \subsection{} command to identify level 2 heads.
%An example:
%\subsection{Level 2 Head}
%
% ---------------
% Level 3 head
%
% Use the \subsubsection{} command to identify level 3 heads
%An example:
%\subsubsection{Level 3 Head}
%
%---------------
% Level 4 head
%
% Use the \subsubsubsection{} command to identify level 3 heads
% An example:
%\subsubsubsection{Level 4 Head} An example.
%
%% ------------------------------------------------------------------------ %%
%
%  IN-TEXT LISTS
%
%% ------------------------------------------------------------------------ %%
%
% Do not use bulleted lists; enumerated lists are okay.
% \begin{enumerate}
% \item
% \item
% \item
% \end{enumerate}
%
%% ------------------------------------------------------------------------ %%
%
%  EQUATIONS
%
%% ------------------------------------------------------------------------ %%

% Single-line equations are centered.
% Equation arrays will appear left-aligned.

Math coded inside display math mode \[ ...\]
 will not be numbered, e.g.,:
 \[ x^2=y^2 + z^2\]

 Math coded inside \begin{equation} and \end{equation} will
 be automatically numbered, e.g.,:
 \begin{equation}
 x^2=y^2 + z^2
 \end{equation}

% IF YOU HAVE MULTI-LINE EQUATIONS, PLEASE
% BREAK THE EQUATIONS INTO TWO OR MORE LINES
% OF SINGLE COLUMN WIDTH (20 pc, 8.3 cm)
% using double backslashes (\\).

% To create multiline equations, use the
% \begin{eqnarray} and \end{eqnarray} environment
% as demonstrated below.
\begin{eqnarray}
  x_{1} & = & (x - x_{0}) \cos \Theta \nonumber \\
        && + (y - y_{0}) \sin \Theta  \nonumber \\
  y_{1} & = & -(x - x_{0}) \sin \Theta \nonumber \\
        && + (y - y_{0}) \cos \Theta.
\end{eqnarray}

%If you don't want an equation number, use the star form:
%\begin{eqnarray*}...\end{eqnarray*}

% Break each line at a sign of operation
% (+, -, etc.) if possible, with the sign of operation
% on the new line.

% Indent second and subsequent lines to align with
% the first character following the equal sign on the
% first line.

% Use an \hspace{} command to insert horizontal space
% into your equation if necessary. Place an appropriate
% unit of measure between the curly braces, e.g.
% \hspace{1in}; you may have to experiment to achieve
% the correct amount of space.


%% ------------------------------------------------------------------------ %%
%
%  EQUATION NUMBERING: COUNTER
%
%% ------------------------------------------------------------------------ %%

% You may change equation numbering by resetting
% the equation counter or by explicitly numbering
% an equation.

% To explicitly number an equation, type \eqnum{}
% (with the desired number between the brackets)
% after the \begin{equation} or \begin{eqnarray}
% command.  The \eqnum{} command will affect only
% the equation it appears with; LaTeX will number
% any equations appearing later in the manuscript
% according to the equation counter.
%

% If you have a multiline equation that needs only
% one equation number, use a \nonumber command in
% front of the double backslashes (\\) as shown in
% the multiline equation above.

%% ------------------------------------------------------------------------ %%
%
%  SIDEWAYS FIGURE AND TABLE EXAMPLES
%
%% ------------------------------------------------------------------------ %%
%
% For tables and figures, add \usepackage{rotating} to the paper and add the rotating.sty file to the folder.
% AGU prefers the use of {sidewaystable} over {landscapetable} as it causes fewer problems.
%
% \begin{sidewaysfigure}
% \includegraphics[width=20pc]{samplefigure.eps}
% \caption{caption here}
% \label{label_here}
% \end{sidewaysfigure}
%
%
%
% \begin{sidewaystable}
% \caption{}
% \begin{tabular}
% Table layout here.
% \end{tabular}
% \end{sidewaystable}
%
%

