%-------------------CAM-------------------------
The Community Atmosphere Model \citep[CAM]{nealeetal2010}, forms the atmospheric component of the NCAR Community Earth System Model (CESM). 
We have run CAM version 5.0 with a finite-volume dynamical core, $1.9^{\text{o}} \times 2.5^{\text{o}}$ horizontal resolution, and  30 hybrid-coordinate vertical levels, the highest of which is near 2 hPa.
The top three model levels (starting at about 14 hPa) constitute a ``sponge'' layer, where horizontal diffusion is applied to temperature, vorticity, and divergence in order to absorb vertically propagating planetary waves.  
The diffusion has been tuned in order to give a reasonable strength of the stratospheric polar night jets \citep{nealeetal2010}.

%-------------------WACCM-------------------------
The Whole Atmosphere Community Climate Model \citep[WACCM]{Marsh2013} extends the dynamical core of CAM up to $5 \times 10^{-6}$hPa, which is in the lower thermosphere, and includes additional chemical and physical processes for these regions. 
The experiments here use WACCM4, from the CESM distribution 1.1.1 (\NCARFULL) and 1.2.0 (\WACCMNODA, \WACCMTROPICS, and \WACCMGLOBAL). 
WACCM's three highest model levels (starting from about $2 \times 10^{-5}$hPa) have strong horizontal diffusion to absorb planetary waves. 
%% Kevin Reader's answer about the WACCM sponge layers: 
%The number of damped levels depends on the dissipation choice in the WACCM atm_in namelist variable div24del2flag. For value 2 (ldiv2 diffusion) there are 3 damped levels. For values 4 and 42 (ldiv4 and ldiv4+ldel2) there are 4 damped levels.
%In addition, the advection operators go to first order for the top 1/8 of the levels, which is 8 for the 66 level model.  I believe that this is a gentler
The DART-WACCM simulations in this study were also run at $1.9^{\text{o}} \times 2.5^{\text{o}}$ horizontal resolution, with 66 vertical levels. 
%dissipation, so I think that we should set highest_state_pressure_Pa so that only levels 1-3 (or 4) feel no observations.

