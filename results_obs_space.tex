
\subsection{Error reduction in observation space}

Let us first compare the experiments in Table \ref{tab:expts} in terms of the observed AAM components [(\ref{eq:X1})-(\ref{eq:X3})].
Figure \ref{fig:fit_to_ERPs} compares the prior ensemble of each AAM excitation function to the true state for each of the four experiments. 
With no assimilation (first column), the three AAM values of the ensemble spread rapidly about the truth, saturating after about a month.  

As the different observational constraints are applied (columns 2-4), the ensemble estimate of each AAM component comes closer to the truth. 
Assimilating the three global AAM components (E-AAM second column) constrains the ensemble almost as much as when local temperatures are assimilated (E-RST, third column), and assimilating both observation types together (E-ERPRST, fourth column) offers the strongest constraint.
This indicates that the EAKF is indeed able to adjust individual variable fields of each ensemble member in such a way that the global AAM agrees with observations. 
This suggests that the assimilation of the AAM components offers a useful constraint. 
Whether the error in the state variables (i.e. column 1 of Fig. \ref{fig:NODA}) is actually reduced will be investigated in Section \ref{sec:erpda}.
