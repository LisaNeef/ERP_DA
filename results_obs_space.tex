\hl{INSERT transition sentence from previous section.}

In Experiments 1-3, (Table \ref{tab:expts}), we assimilate synthetic observations of the three atmospheric angular momentum functions [(\ref{eq:X1})-(\ref{eq:X3})] from a CAM "nature run" into a 80-member CAM ensemble. 
To illustrate the constraint that these observations impose, Figure \ref{fig:fit_to_ERPs} compares the four CAM experiments in terms of their angular momentum functions $\chi_2$ and $\chi_3$ (as in Fig. \ref{fig:**} omitting $\chi_1$ for its similarity to $\chi_2$).  
In this figure we compare the ensemble to the synthetic "observations" of the angular momentum functions, rather than observed Earth rotation parameters. 

With no assimilation (\NODA, first column), the angular momentum functions again show how the ensemble spreads about the truth, and how the spread saturates after about one month.
When we assimilate the angular momentum functions themeselves (E2, second column), the ensemble predictably clusters close to the true angular momentum, and captures the day-to-day angular momentum variations. 
The difference between the first two columns indicates that assimilating the angular momentum has imposed some kind of constraint upon the wind, temperature, and surface pressure fields, though we do not yet know whether those fields have also moved closer to the true state. 

The ensemble clusters even more tightly around the truth when instead of the angular momentum functions we assimilate local temperature observations (E3, third column), which is a much stronger constraint on the model fields. 
Finally, adding the angular momentum observations to the regularly-spaced temperature observations (E4, fourth column) slightly increases the agreement between the ensemble and the true state further.  
This suggests that the angular momentum observations may contain additional information that complemetns the information in the temperature observations, though the difference between E3 (assimilating only temperature) and E4 (assimilating temperature and angular momentum) is small.  
