
\subsection{Error reduction in observation space}
Of the experiments outlined in the previous section, two include the three atmospheric angular momentum functions [(\ref{eq:X1})-(\ref{eq:X3})] as an assimilation variable. 
To compare the constraint imposed by assimilating this variable, Figure \ref{fig:fit_to_ERPs} shows the prior ensemble estimate of the three angular momentum components as a function of time, compared the the "true" values, for each of the four experiments.

With no assimilation (first column), the ensemble estimates of the three angular momentum components spread rapidly about the; the ensemble spread for each component saturates after about 30 days.
Applying the three different observational constraints (columns 2-4), the ensemble clusters closely around the truth.
Assimilating the three angular momentum components only (second column) constrains the ensemble almost as much as whenwe assimilate the global grid of radiosonde temperature measurements (third column).
Assimilating both observation types together (fourth column) constrains the global angular momentum even more than assimilating the radiosonde grid alone. 
Both results suggest that rectifying the misfit between the observed and predicted angular momentum holds a great deal of information about the modeled atmosphere. 
However, Figure \ref{fig:fit_to_ERPs} does not show whether adding the angular momentumm components to the assimilation actually reduces the error in the model state -- this is the subject of the next section.
