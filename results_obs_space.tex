In two of the eight experiments in Table \ref{tab:expts}, we assimilate the three atmospheric angular momentum functions [(\ref{eq:X1})-(\ref{eq:X3})] as additional observational constraints.
To illustrate the constraint that these observations impose, Figure \ref{fig:fit_to_ERPs} compares the four CAM experiments in terms of their angular momentum, in each case comparing the ensemble and the true state.

With no assimilation (first column), the angular momentum components show how the ensemble spreads about the truth, and how the spread saturates after about one month.
When we assimilate the angular momentum (second column), the ensemble predictably clusters close to the true angular momentum, and captures the day-to-day angular momentum variations. 
The difference between the first two columns indicates that assimilating the angular momentum has imposed some kind of constraint upon the wind, temperature, and surface pressure fields, though we do not yet know whether those fields have also moved closer to the true state. 

Not surprisingly, the ensemble clusters even more tightly around the truth when instead of global integrals of the wind and pressure fields we assimilate local radiosonde temperatures (column 3), which is a much stronger constraint on the model fields. 

Finally adding angular momentum observations to the radiosonde observations (column 4) increases the agreement between the ensemble and the truth even further -- this suggests that the angular momentum observations contain additional information not found in the global grid of radiosonde-type observations.  
Note, however, that the difference between assimilating conventional observations (column 3) and conventioanal plus AAM observations (column 4) is much smaller than \citet{Saynisch2010} found for the axial term and the ocean (see their fig. 2) and what \citet{Saynisch2011} found for the equatorial terms (their fig. 1).   
\hl{What does this imply for my work??}
