\hl{INSERT transition sentence from previous section.}

In Experiments 1-3, (Table \ref{tab:expts}), we assimilate synthetic observations of the three atmospheric angular momentum functions [(\ref{eq:X1})-(\ref{eq:X3})] from a CAM "nature run" into a 80-member CAM ensemble. 
To illustrate the constraint that these observations impose, Figure \ref{fig:fit_to_ERPs} compares the four CAM experiments in terms of their angular momentum, in each case comparing the ensemble and the true state.

With no assimilation (first column), the angular momentum components show how the ensemble spreads about the truth, and how the spread saturates after about one month.
When we assimilate the angular momentum (second column), the ensemble predictably clusters close to the true angular momentum, and captures the day-to-day angular momentum variations. 
The difference between the first two columns indicates that assimilating the angular momentum has imposed some kind of constraint upon the wind, temperature, and surface pressure fields, though we do not yet know whether those fields have also moved closer to the true state. 

Not surprisingly, the ensemble clusters even more tightly around the truth when instead of global integrals of the wind and pressure fields we assimilate local radiosonde temperatures (E3), which is a much stronger constraint on the model fields. 
Adding the angular momentum observations to the regularly-spaced temperature observations (E4) further increases agreement between the ensemble and the true state.  
This suggests that the angular momentum observations may contain additional information that complemetns the information in the temperature observations, though the difference between E3 (assimilating only temperature) and E4 (assimilating temperature and angular momentum) is small.  
