Figure \ref{fig:fit_to_ERPs} compares the modeled angular momentum functions for Experiments \NODA-\ERPRST (similar to \ref{fig:evalvariable_aam}, and again omitting $\chi_1$ for simplicity). 
The angular momentum components for each experiment are also compared to the ``true'' angular momentum in each case, which is assimilated in \ERPALL ~and \ERPRST.  

With no assimilation (\NODA, first column), the angular momentum functions again show how the ensemble spreads about the truth, and how the spread saturates after about one month.
If we assimilate the angular momentum functions alone (\ERPALL, second column), the ensemble predictably clusters close to the true angular momentum, and captures the day-to-day angular momentum variations. 
The difference between the first two columns indicates that assimilating the angular momentum has imposed some kind of constraint upon the wind, temperature, and surface pressure fields, though we do not yet know whether those fields have also moved closer to the true state. 

The ensemble clusters even more tightly around the truth when instead of the angular momentum functions we assimilate local temperature observations (\RST, third column), which is a much stronger constraint on the model fields. 
Finally, adding the angular momentum observations to the regularly-spaced temperature observations (\ERPRST, fourth column) slightly increases the agreement between the ensemble and the true state further.  
This suggests that the angular momentum observations may contain additional information that complemetns the information in the temperature observations, though the difference between \RST ~(assimilating only temperature) and \ERPRST ~(assimilating temperature and angular momentum) is small.  
