
\subsection{Error reduction in observation space}
In two of the four ensemble simulations outlined in the previous section (Table **), we assimilate the three atmospheric angular momentum functions [(\ref{eq:X1})-(\ref{eq:X3})] as additional observational constraints.
To illustrate the constraint that these observations impose, Figure \ref{fig:fit_to_ERPs} compares the four experiments in terms of their angular momentum components, in each case comparing the ensemble, its mean, and the true state.

For no assimilation (first column), the angular momentum components show how the ensemble spreads about the mean; the ensemble spread saturates after about one month.
When we assimilate the angular momentum components (second column), the ensemble clusters close to the truth, and the ensemble mean now captures the day-to-day angular momentum variations. 
Thee difference between the first two columns indicates that assimilating the angular momentum has imposed some kind of constraint upon the wind, temperature, and surface pressure fields, though we do not yet know whether those fields have also moved closer to the true state. 

Not surprisingly, the ensemble clusters even more tightly around the truth when instead of global integrals of the wind and pressure fields we assimilate local radiosonde temperatures (column 3), which is a much stronger constraint on the model fields. 
However, adding angular momentum observations to the radiosonde observations increases the agreement between the ensemble and the truth even further.
This suggests that the two observation types -- global angular momentum and local measurements -- complement each other.


