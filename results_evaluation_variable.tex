The three DART-WACCM simulations represent three ensembles with varying levels of observational constraint. 
We can evaluate the strength of the constraint imposed by the observations in each case by comparing the spread of the ensemble in each case. 
Figure \ref{fig:evalvariable_state} shows the global mean ensemble variance in the zonal wind in each case for the first four months of assimilation. 
Not surprisingly, the ensemble variance is lowest for global assimilation and highest for no assimilation.
The simulation with tropical assimilation only has much higher ensemble spread than the global assimilation case, indicating that it is much less certain. 
However, while the ensemble variance gives us an idea of global uncertainty, it does not tell us anything about accuracy. 
The global assimilation case shows a peak in ensemble variance on 17 December 2009, suggesting that the adaptive inflation algorithm has detected some divergenece of the filter from the observations; we could check whether more observations are rejected around this time, but that still does not tell us much about the accuracy of the state estimate. 
It is also unclear why the ensemble variance grows much more quickly in the two assimilation cases relative to the no-assimilation case, in the first week. 

Since Earth rotation parameters are proportional to the atmospheric angular momentum, they can be used to evaluate the accuracy of each simulation. 
Figure \ref{fig:evalvariable_aam} shows the three angular momentum components for each of the three WACCM ensembles, compared to the angular momentum implied by the observed Earth rotation parameter components [(\ref{eq:X12_to_PM})-(\ref{eq:X3_to_LOD})]. 
\hl{Describe what is wrong with X1 as a parameter -- it suffers from the IB approximation.}
\hl{Describe what is wrong with X3 as a parameter -- there's the problem if a constant of integration.}
This leaves $\chi_2$ as the best parameter to evaluate the performance of an ensemble data assimilation system. 

Comparing $\chi_2$ in observations and the ensemble (second row) between the three experiments shows not only the relative ensemble spread between the three cases, but also the relative accuracies of the ensemble mean. 
For example, we can see that assimilating in the tropics only actually results in a fairly accurate ensemble mean, despite the larger ensemble variance. 
The anomalously large ensemble variance on 17 Dec in the global-DA case does indeed follow a period of filter divergence, where the ensemble variance is persistently smaller than the distance to the observations. 

We can also see how ading the first observation at the initial time immediately corrects a strong bias in the model fields, which also increases the ensemble spread. 
\hl{[explain more clearly.]}

In summary, $\chi_2$, the equatorial component of angular momentum that \hl{[wait, which one is this again?]}, is a convenient way to check the fidelity of a data assimilation system, because it can indicate times of filter divergence and the overall accuracy of a data-constrained simulation. 
