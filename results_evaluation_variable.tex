In practice, it is often difficult to judge whether an assimilation system has actually succeeded in bringing the modeled state closer to the truth, simply because the true state isn't known. 
Typically the solution is to compare the analyzed model state to independent observations, but these can be difficult to obtain because observing systems often don't overlap sufficiently in time and space.
Here the atmospheric angular momentum may make a useful evaluation parameter, since it is easy to compute from the modeled wind and pressure fields, and is observed in the form of three parameters that are freely available at 12-hourly resolution, updated continuously, and available at high accuracy since about 1980.  

Experiments \WACCMNODA-\NCARFULL, which each run 40-member DART-WACCM ensembles, 
represent a stepwise increase of the constraint imposed by observations, and can be used to illustrate this idea.
Since we don't know the true state of the atmosphere, we could take the ensemble variance 
\begin{eqnarray}
S = 
\frac{1}{N-1}
\sum_{n=1}^N
\left(
	\left< x_{i}^{n} \right>-x_i^t
\right)^2,
\label{eq:spread}
\end{eqnarray}
(where $x_{i}$ represents a component of the state vector in the ensemble,  and $x_{i}^t$ and $x_{i}^n$ the corresponding component in the truth and individual ensemble members, respectively) as a measure of the constraint imposed by the observations in each case. 

Figure \ref{fig:evalvariable_state} compares 
the global mean ensemble variance of the zonal wind in each experiment as a function of time.  
As expected, ensemble variance is pushed down every time an observation comes in, in the ``saw'' pattern that is typical of sequential data assimilation. 
Fig.  \ref{fig:evalvariable_state} can give us an idea of where the state estimate is more or less certain, but can tell us nothing about whether the observational constraints actually bring the analyzed state closer to the truth. 
For example, while it is not surprising that the simulation with the most observations assimilated (\NCARFULL) eventually has the lowest ensemble variance, the ensemble variance in this case also shows a spike at about 20 days, which suggests that the effect of the addional SABER observations might be more complicated. 
\WACCMTROPICS~ (assimilating only in the tropics), doesn't see a reduced ensemble spread relative to \NODA~ (no assimilation) until about two weeks of assimilation, which suggests that the ensemble mean state in this case has not become more accurate --though, again, we don't know if this is indeed the case.  


Figure \ref{fig:evalvariable_aam} shows how the four simulations compare in terms of two of the three global components of angular momentum, $\chi_2$ and $\chi_3$ (here $\chi_1$ has been omitted because it looks qualitatively similar to $\chi_2$).  
For each experiment, the angular momentum components of the ensemble are compared  to the corresponding observed Earth rotation parameter. 
Figure  \ref{fig:evalvariable_aam} makes it easy to see how the constraint increases with more observations assimilated -- scanning the panels from left to right, the ensemble becomes more tightly clustered for both angular momentum components, and generally moves closer to the observed Earth rotation parameter. 
For example, while \WACCMTROPICS~ (assimilating only in the tropics) increases the ensemble spread relative to no assimilation in the first 20 days of assimilation, it still brings the ensemble angular momentum closer to the observed Earth roation parameters -- thus even while assimilation increases spread, it also improves the state estimate. 
In the rightmost panels, the peak on the ensemble variance at 20 days in \NCARFULL~ (adding SABER observations) reveals itself as a strong departure of the equatorial angular momentum from the polar motion.   
The difference between modeled $\chi_2$ and the observed polar motion in this case suggests that the entire ensemble might have developed a midlatitude mass field anomaly that is not present in the true state. 

The angular momentum has one shortcoming as an assimilation variable in that it most strongly weights the lowest model levels, since they carry the most mass. 
Thus, for example, the angular momentum diagnostic does not capture a major benefit of the SABER observations, which is that they reduce the error in the stratosphere. 
The observed Earth rotation parameters nevertheless make a useful tool for evaluating different data assimilation experiments. 
