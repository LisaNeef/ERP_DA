In non-idealized assimilation experiments, it is difficult to judge whether the assimilation of a given set of observations has actually succeeded in bringing the estimated state closer to the truth.
We can validate such an assimilation experiment by comparing to independent observations, but these can be difficult to obtain because observing systems often don't overlap sufficiently in time and space.


%---uncomment if needed
%GPS-RO measurements have been shown to add a high amount of information to numerical weather forecasts (Bonavita, 2013), due to their low systematic errors and high vertical resolution. It was shown by (Wang et al., 2013) that GPS-RO data give an unprecedented look into the vertical structure of the atmosphere because of their high vertical resolution, in this case showing a warming trend in the tropopause inversion layer that would not have been observed otherwise.

