It can be difficult to judge whether assimilation has succeeded in bringing a model closer to the truth, which isn't known. 
Independent observations that cover the same temporal and spatial domain as the assimilated observations often don't exist. 
Here the atmospheric angular momentum is a useful evaluation parameter, since it is easy to compute from the modeled wind and pressure fields, and is observed in the form of three parameters that are freely available at 12-hourly resolution, updated continuously, and available since the 1960s (and at high accuracy since about 1980).  

Experiments \WACCMNODA-\WACCMGLOBAL, which run 40-member DART-WACCM ensembles with an increasing data constraint, illustrate this idea in Figures \ref{fig:evalvariable_state}-\ref{fig:evalvariable_aam}.   
The ensemble variance,
\begin{eqnarray}
S_i = 
\frac{1}{N-1}
\sum_{n=1}^N
\left(
	\left< x_{i}^{n} \right>-x_i^t
\right)^2,
\label{eq:spread}
\end{eqnarray}
(where $x_{i}$ represents a component of the state vector in the ensemble,  and $x_{i}^t$ and $x_{i}^n$ the corresponding component in the truth and individual ensemble members, respectively), measures the constraint imposed by the observations in each case.
For example, comparing the global mean ensemble variance of the zonal wind in each experiment (Fig. \ref{fig:evalvariable_state}), we see that uncertainty in the state estimate is reduced at every observation time (in the ``saw'' pattern that is typical of sequential data assimilation), and that the experiments with more observations have a lower overall ensemble spread. 

Not knowing the true state, however, the ensemble variance only tells us about the uncertainty of the ensemble and not whether this uncertainty captures the true error. 
For example, ensemble spread in \WACCMTROPICS~(assimilating in the tropical band) is actually larger than in \WACCMNODA~(no assimilation), which suggests that the state estimate in this case might be less accurate. 
Likewise, increases in the ensemble spread in \WACCMGLOBAL around 5 and 32 days suggest that the state estimate might be worse here. 


Figure \ref{fig:evalvariable_aam} shows how  \WACCMNODA-\WACCMGLOBAL~compare in terms of two of the three components of angular momentum, $\chi_2$ and $\chi_3$ ($\chi_1$ has been omitted because it looks qualitatively similar to $\chi_2$),  
compared to the corresponding observed Earth rotation parameters. 
Figure  \ref{fig:evalvariable_aam} also shows how the constraint increases with more observations assimilated (scanning the panels from left to right, the ensemble becomes more tightly clustered for both angular momentum components, and moves closer to the observed Earth rotation parameter), but now also reveals when the filter does and does not diverge from the truth.  
For example, the large ensemble spread in the first two weeks of \WACCMTROPICS, and in the first 5 days of \WACCMGLOBAL, actually reflects the true error better than the ensemble spread in \WACCMNODA, and shows an ensemble that is closer to the true state. 
The increased ensemble spread in \WACCMGLOBAL~at 32 days coincides with divergence of the ensemble from the truth in $\chi_2$. 
This suggests that the adaptive inflation in DART has detected the budding filter divergence and increased the ensemble spread in order to prevent further divergence, although comparing the ensemble-estimated $\chi_2$ to the observed polar motion in this case suggests that filter divergence may be ongoing.  
