Figure \ref{fig:evalvariable_state} compares the global mean ensemble variance of the zonal wind between the free-running, 40-member WACCM ensemble, and the ensemble with 6-hourly assimilation. 
With assimilation, the ensemble variance, which measures the uncertainty in the state estimate, is reduced at every observation time in the ``saw'' pattern that is typical of sequential data assimilationd.
Not knowing the true state, however, the ensemble variance only tells us about the uncertainty of the ensemble and not whether this uncertainty captures the true error. 
At the beginning of the assimilation window ("A" in Fig.~\ref{fig:evalvariable_state}) the ensemble spread in the assimilation case is larger than it would be with no assimilation, indicating larger uncertainty. 
The ensemble spread reaches a steady state after about 20 days of assimilation, but still shows rises and falls, e.g. ensemble variance increases at 32 days ("B" in Fig.~\ref{fig:evalvariable_state}), suggesting that the state estimate might be less reliable there. 

To see what is actually going on during the assimilation, we can compare the angular momentum of the model ensemble in each case to the corresponding Earth rotation variations. 
\hl{[Correct text here after updating figure.]}
Figure \ref{fig:evalvariable_aam} compares the WACCM ensembles with and without assimilation in terms of two angular momentum components, $\chi_2$ and $\chi_3$ ($\chi_1$ has been omitted because it looks qualitatively similar to $\chi_2$),  
compared to the corresponding observed Earth rotation parameters. 
Figure  \ref{fig:evalvariable_aam} shows how the constraint increases with more observations assimilated (scanning the panels from left to right, the ensemble becomes more tightly clustered and moves closer to the corresponding observed Earth rotation parameters). 
Fig. \ref{fig:evalvariable_aam} now also reveals when the filter does and does not diverge from the truth.  
For example, while the ensemble with assimilation shows larger spread in the first five days, comparison to the Earth rotation parameters shows that the ensemble in this case is already closer to the truth than without assimilation. 
In the case with assimilation, an increased ensemble spread at about 30 days coincides with divergence of the ensemble mean from the truth, which can be seen in $\chi_2$ and $\chi_2$ . 
This suggests that the adaptive inflation in DART has detected the budding filter divergence at \~30 days, and increased the ensemble spread in order to prevent further divergence, though divergence from the truth continues well into the second month of assimilation.
\hl{[indicate this clearly on the figures.]}
This divergence would not have been detected by lookig at ensemble spread (Fig. \ref{fig:evalvariable_state}) alone. 
