%-------------------WACCM-------------------------
WACCM extends the dynamical core of CAM up to $5 \times 10^{-6}$hPa (circa 140 km, in the lower thermosphere) with 66 levels, and includes additional chemical and physical processes for these regions. 
The experiments here use WACCM version 4 with full interactive chemistry.
WACCM's three highest model levels (starting from about $2 \times 10^{-5}$hPa) have strong horizontal diffusion to absorb planetary waves. 
%% Kevin Reader's answer about the WACCM sponge layers: 
%The number of damped levels depends on the dissipation choice in the WACCM atm_in namelist variable div24del2flag. For value 2 (ldiv2 diffusion) there are 3 damped levels. For values 4 and 42 (ldiv4 and ldiv4+ldel2) there are 4 damped levels.
%In addition, the advection operators go to first order for the top 1/8 of the levels, which is 8 for the 66 level model.  I believe that this is a gentler
The DART-WACCM simulations shown here, like the CAM experiments in the previous section, have $1.9^{\text{o}} \times 2.5^{\text{o}}$ horizontal resolution.
%dissipation, so I think that we should set highest_state_pressure_Pa so that only levels 1-3 (or 4) feel no observations.


%-----WACCM experiments
Three DART-WACCM experiments are examined (Table \ref{tab:expts}); 
each runs a 40-member WACCM ensemble from 1 Oct 2009 to 31 Jan 2010. 
The first experiment is simply a WACCM ensemble with no assimilation, while the second 
assimilates the meteorological observations that are used in the NCEP/NCAR reanalysis, i.e. winds and temperatures from radiosondes and aircraft \citep{Saha2010}, as well as refractivity profiles measured via GPS radio-occultation by the COSMIC satellite constellation \citep{Anthes2008}.
The observations extend from the surface to about 2 hPa. 
A third experiment assimilates the same observations, but only in the tropical band (30S - 30N). 

In the cases with assimilation, the impact of the observations is localized using a  
Gaspari-Cohn function \citep{Gaspari1999} with a half-width of 0.2 radians in the horizontal, and 0.5 scale heights in the vertical.  
Adaptive ensemble inflation \citep{Anderson2009tellus} increases the ensemble spread when too many observations are rejected; the inflation factor varies in space and time and is proportional to the distance between the ensemble mean and the observation, given the uncertainties of each.  
No analysis increment is allowed above 0.1 hPa, in order to prevent the generation of spurious gravity waves resulting from a large ensemble spread due to the sparsity of upper level observations \citep{Polavarapu2005b}. 

%%---localization and inflation in Nick's 2014 paper: 
%Similar to Pedatella et al. [2013], the horizontal and vertical localizations are specified with a Gaspari-Cohn function [Gaspari and Cohn, 1999]
%with half widths of 0.2 radians and 0.5 in ln(p∕p0) coordinates, respectively. Spatially and temporally varying adaptive inflation is used to inflate the ensemble prior to the assimilation in order to prevent collapse of the model spread [Anderson, 2009].

%%---localization and inflation in Nick's 2013 paper: 
%The most notable change is that we specify the ver- tical localization using a Gaspari-Cohn function [Gaspari and Cohn, 1999] with a half width of 0.5 in ln(p/p0) coor- dinates, where p is pressure and p0 is the surface pressure. This differs from the vertical localization of 1000 hPa in p used by Raeder et al. [2012]. 

