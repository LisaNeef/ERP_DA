%-----------paper summary---------------------
% answer the question asked by this study 
We have evaluated the potential for using observations of Earth rotation parameters to evaluate data assimilation systems and constrain atmospheric models as assimilated variables.  
% a high-level summary of what my data show:
% (1) ER data is a convenient way to evaluate how close an analysis is getting to the truth
% (2) But if we want to assimilate ER data, we find that the covariance model doesn't work right because it is trying to capture tiny correlations 
% (2) if we constrain the ensemble more using other obs, it still doesn't really work because we end up inhibiting out ability to fit the conventional obs by insisting that it fit these small, flawed correlations. 
%
Earth rotation parameters are useful as three simple observables against which we can validate the angular momentum of an atmosphere simulation that is constrained by conventional observations.  
Advantages of the Earth rotation parameters are that they are freely available \citep{iers} at high accuracy and relatively high temporal resolution, and continuously updated in a timeseries starting in 1960. 
Comparing the angular momentum of a model ensemble to these observations is computationally straightforward, and 
gives a global view both of how well the modeled wind and mass fields approximate the truth.

However, it is extremely difficult to constrain the model state with angular momentum observations because they are global integrals, as shown by 
idealized assimilation experiments using synthetic observations generated from a prescribed truth.
\hl{[Below: indicate which figures show which result.]}
\hl{The ensemble filter used in this study generates a covariance model that is physically plausible and computes the statistically most likely update of each state space component, and indeed can help to constrain the model towards the truth if state-to-observation correlations have a large scale. 
Nevertheless, correlations between individual state components and the global angular momentum are much smaller than the sampling error of a finite ensemble, and it is common that 
the state update computed by the filter moves the ensemble and its mean farther away from the true state, while still satisfying the angular momentum observations. 
The global angular momentum observations do not impose a strong enough constraint to yield a reliable and robust correction to the model state.} 

One might therefore expect that global angular momentum observations can constrain the modeled state better if the ensemble is already largely constrained to the truth by more conventional, spatially-localized observations. 
We tested this idea by assimilating a global grid of synthetic temperature observations in addition to the global angular momentum observations. 
We found that, overall, adding the angular momentum observations worsens the analysis, 
because requiring that the state satisfy the observed angular momentum inhibits the analysis increment in the model state.  

% anticipation of any criticisms and caveats  
A common technique for avoiding divergence of an ensemble filter is to periodically inflate the ensemble about its mean. 
It is unlikely that ensemble inflation would help here, since we found that the filter diverges both when the ensemble spread is large (section \ref{sec:erpda}) and when it is very small (section \ref{sec:added_value}). 
We have seen in a set of short assimilation runs with ensemble inflation (omitted here for brevity), that this is indeed the case. 


% my result in the context of what others have done  
Even though this study focused on an atmosphere model, the results can likely be extended to ocean models, where the state vector, although smaller, is still many orders of magnitude larger than the observation vector, and this would also yield state-to-observation correlations that are much smaller than the sampling error. 
For the ocean, it is possible to adjust the freshwater flux into the model, rather than the prognostic state variables, as a control parameter (as in \citet{Saynisch2010} and \citet{Saynisch2012}), but this has no corollary in atmosphere models. 


% points for future research  
Overall, this work illustrates the difficulties of assimilating observations that represent integrals or averages of the model state. 
\citet{Dirren2005} proposed a variation of the ensemble filter to deal with observations that are temporal averages of the state; their alternative algorithm projects the observation increment only on the time-average of each ensemble member while keeping deviations from the average untouched. 
\citet{Huntley2009} showed that applying this method in a global atmosphere model reduces errors over a wide range of errors, and that one can even improve instantanous errors when only time-average observations are assimilated. 
It is conceivable that this approach could be extended to the assimilation of Earth rotation parameters, by updating the global-average contribution to the AAM at each gridpoint, and thereby improve the state estimate. 
However, for assimilation systems on the scale of DART-CAM and DART-WACCM, this would require major and fundamental changes in the overall code structure, which are beyond the scope of this study. 


Another alternative pathway may be to assimilate the rate-of-change of Earth rotation parameters, thereby constraining the total torque between the atmosphere and the solid Earth. 
The net torque is the sum of pressure gradients over topography (the so-called mountain torque), surface friction, and torque due to topographic gravity waves \citep{Lejenas1997}. 
Thus appropriate control variables could be the surface pressure, orographic gravity wave drag parameters, or surface friction. 
We defer exploration of this idea to future research. 


