Even when a given observation is unable to significantly constrain a modeled state by itself, it may still improve the assimilation of other, more conventional observations.  
Figure \ref{fig:fit_to_ERPs} column 3 shows that assimilating 
temperature observations alone into the ensemble is sufficient to achieve a good fit to the true angular momentum.
But then adding observations of the angular momentum improves the fit further, which implies that  
the angular momentum observations add information to the ensemble. 
If we assimilate angular momentum (or Earth rotation) observations along with local observations that bring the ensemble mean closer to the true state, the angular momentum observations could perhaps push the ensemble even closer to the truth 
by adding the additional requirement that the modeled angular momentum should match observations. 
But whether or not the small and error-frought state-to-observation covariances of the angular momentum really add value to the state estimate is not obvious, and here again we examine the true error in idealized experiments with 80-member CAM ensembles. 

% (a) what happens to the spread?  
Figure \ref{fig:added_value_MSE}a shows the increment (posterior-prior) in the global-average ensemble variance in the zonal wind, assimilating global temperatures with and without additional angular momentum observations. 
The increment in the ensemble spread is always negative [following (\ref{eq:sigma_a})], and adjusts from an initially weak value to a stronger steady-state value as the ensemble, which was initially constrained by both wind and temperature observations, adjusts to being constrained only by temperature observations. 
During this early period, adding the angular momentum observations causes a slightly stronger adjustment of the ensemble variance, indicating that the angular momentum observations add some information to the state estimate. 
% (b) but the error increments look really different -- here we see sampling error  
However, comparing the posterior-prior difference in the mean square error of the ensemble mean (Fig. \ref{fig:added_value_MSE}b) shows that adding the angular momentum observations actually results in a weaker error reduction during the adjustment period (1-5 Jan). 
Although the additional angular momentum observations do not add any additional certainty (Fig. \ref{fig:added_value_MSE}a) once the ensemble variance has reached steady state (after about 9 Jan), 
% (c) what does the total mean square error show? -- adding the ERPs increases error at worst and does nothing at best
the absolute posterior mean square error (Figure \ref{fig:added_value_MSE}c) is consistently larger when the angular momentum observations are added. 
Thus the additional observations increase the global mean error at worst (at the beginning of the assimilation), and have no impact at best (at steady-state). 
We therefore find no clear benefit of assimilating atmospheric angular momentum (or Earth rotation parameter) observations, either with or without the assimilation of conventional observations. 

