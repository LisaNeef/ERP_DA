
\section{Can Earth rotation observations add value?}

The previous sections showed that the assimilation of AAM observations can reduce error in the modeled wind and pressure fields, as long as the ensemble spread has not reached its saturation level. 
Thus it makes sense to test whether the assimilation of AAM is more successful if the AAM functions are assimilated together with (more conventional) spatially localized observations to keep the ensemble spread from growing.

Ccombining regularly-spaced observations with the AAM observations improved the estimated AAM [Fig. \ref{fig:fit_to_LOD}(d)], but worsened the estimated model state. 
\textcolor{alert}{[Insert figure showing that adding AAM to local temperature observations increases the error while decreasing the spread.]}

Figure \ref{fig:compare_divergence_added_value} is similar to Fig. \ref{fig:compare_divergence_basic}, but now comparing experiment RST to experiment RST+AAM.
Again, we see that the consequence of adding AAM components as an assimilation variables is that the filter diverges, and here the effect is much stronger than in the basic experiment (Fig. \ref{fig:compare_divergence_basic}).
 
