%The previous section showed that the assimilation of global angular momentum can reduce bias in the model state while the ensemble spread is still relatively small.
%This suggests that global integral observations like the anglar momentum might better constrain the state in the presence of other observations that keep the ensemble spread small. 
If angular momentum / Earth rotation parameter observations are assimilated in the presence of local observations, it is conceivable that an ensemble that is already clustered around the true state could perhaps be pushed even closer to the truth by adding the additional requirement that the global angular momentum should match observations. 
In this section, we compare experiments \RST~ and \ERPRST~ in Table \ref{tab:expts}, i.e. assimilating synthetic temperature observations with and without additional angular momentum observations.

Comparing these two experiments in terms of the fit to the angular momentum components (Fig. \ref{fig:fit_to_ERPs} columns 3-4), we see that the temperature observations alone already achieve a good fit to the true angular momentum, and that 
then adding observations of the angular momentum components further improves the fit.  
This means that the angular momentum observations add information to the ensemble, but the question remains as to whether the information is interpreted correctly by the model.

Figure \ref{fig:added_value_MSEincrement} shows zonal mean profiles of the increment in mean square error in zonal wind (i.e. the difference between posterior and prior) as a function of time, for 17 days of asssimilation in \RST~[assimilating temperatures, (a)] and \ERPRST~[assimilating temperatures and angular momentum, (b)].  
Both experiments show entirely negative values, meaning that the analysis increment at each time has reduced the difference between the ensemble mean and the true state.  
Adding the angular momentum observations in \ERPRST~(Fig. \ref{fig:added_value_MSEincrement}b) reduces the breadth of this error reduction, though the error reduction is also sometimes more intense in \ERPRST~than in \RST~(e.g. between 300 and 200 hPa towards the end of the assimilation period). 
This means that the added information from the observed angular momentum indeed sometimes improves the state estimate, but also sometimes inhibits the filter from finding the true state. 

To examine the overall "added value" of assimilating the angular momentum, Figure \ref{fig:added_value_MSE} shows profiles of the difference in zonal wind mean square error between \ERPRST~and \RST. 
The pervasive positive values indicate that adding the angular momentum observations (\ERPRST) almost always increases the distance to the true state (short instances of negative values can be found, but they are small and disappear in the ensemble mean). 
At the same time, adding the angular momentum observations very slightly decreases the ensemble spread (not shown here) everywhere.  
This indicates that the ensemble filter, in correcting the state to fit the observed angular momentum, tends to keep the ensemble from finding the true state by requiring it to fit the observed angular momentum.  

