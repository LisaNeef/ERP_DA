%------------------overview--------------------------
In this section we ask whether an atmosphere model could be constrained to the truth even more by assimilating these parameters. 
The experiments in this section assimilate synthetic observations of the angular momentum, as well as local temperature observations, into the CAM \citep{nealeetal2010}.

%-------------------CAM-------------------------
CAM forms the atmospheric component of the NCAR Community Earth System Model (CESM). 
We have run CAM version 5.0 with a finite-volume dynamical core, $1.9^{\text{o}} \times 2.5^{\text{o}}$ horizontal resolution, and  30 hybrid-coordinate vertical levels, with a top near 2 hPa.
% (from KR:) the highest model level is more like 3.6 hPa, while the model top (a staggered level) is ~2 hPa.  
The top three model levels (starting at about 14 hPa) constitute a ``sponge'' layer, where horizontal diffusion is applied to temperature, vorticity, and divergence in order to absorb vertically propagating planetary waves.  
%The diffusion has been tuned in order to give a reasonable strength of the stratospheric polar night jets \citep{nealeetal2010}.


%-----CAM5 experiments
Experiments \NODA-\ERPRST ~(Table \ref{tab:expts}) are performed using DART+CAM and assimilate synthetic observations of the three components of atmospheric angular momentum.  
An 80-member ensemble of CAM states was generated by selecting the 1 Jan restart files from an 80-year CAM simulation, and assimilating 6-hourly synthetic temperature and wind observations on a uniform grid (roughly 27000 observations at a time) for an entire year. 
% extra info: the obs are assimilation at about 1800 locations on the globe and 15 levels ranging from the surface to 5 hPa
This roughly simulates reanalysis-type observations, and yields a tightly-constrained starting ensemble on 1 Jan 2009.  

A reference CAM simulation, run from Jan - Feb 2009 with observed sea surface temperatures as a boundary condition, constitutes the ``truth'' from which we generate a set of synthetic observations for assimilation. 
From this truth, we generate synthetic observations of $\chi_1$, $\chi_2$, and $\chi_3$ every 24 hours, reflecting the observation frequency of the real Earth rotation data series published by the International Earth Rotation Services \citep{iers}, as well as the wind and temperature observations described above. 
The angular momentum observations are generated with zero observation error, to minimize sampling error effects and simulate an ideal situation where the Earth rotation parameters perfectly capture the atmospheric angular momentum. 
Separate experiments testing various values of observation error did not significantly change the results described below.  

\NODA ~integrates the CAM ensemble forward over Jan and Feb 2009 without data assimilation. 
In \ERPALL, the ensemble is run forward as before, but assimilating synthetic angular momentum observations every 24 hours.
\RST ~assimilates only the temperature observations (a separate experiment assimilating both wind and temperature observations was run but is not discussed here because the results are similar).
\ERPRST ~assimilates both the temperature observations and the three global angular momentum components.
\RST~ and \ERPRST~ only ran for 31 and 17 days, respectively, because this amount of integration time was found to be sufficient to compare the relative error reduction in the two experiments (see Fig. \ref{fig:fit_to_ERPs}). 

This analysis is similar to the studies of \citet{Saynisch2010,Saynisch2011} and \citet{Saynisch2012}, with two main differences:
(1) We use an atmosphere rather than an ocean model.
(2) We assimilate synthetic observations generated from a model simulation that serves as the ``true state''. 
This allows us to measure the true error, and 
also simulates an idealized situation where the atmosphere accounts for 100\% of Earth rotation variations, which means that our experiments do not require an estimate of the oceanic or hydrospheric contributions to Earth rotation variations.
